\documentclass[a4paper]{article}
\newif\ifpdf
\ifx\pdfoutput\undefined
  \pdffalse % we do not use PDFLaTeX
\else
  \pdfoutput=1 % we use PDFLaTeX
  \pdftrue
\fi
\usepackage{linuxdoc-sgml}
\usepackage{qwertz}
\usepackage{url}
\usepackage[latin1]{inputenc}
\ifpdf
  \usepackage[pdftex,colorlinks=true,urlcolor=blue,linkcolor=blue]{hyperref}
  \pdfcompresslevel=9
\else
  \usepackage{t1enc}
%%  \usepackage[dvips]{hyperref}
\fi
\usepackage[english]{babel}
\usepackage{epsfig}
\usepackage{null}
\def\addbibtoc{
\addcontentsline{toc}{section}{\numberline{\mbox{}}\relax\bibname}
}%end-preamble
\setcounter{page}{1}
\urldef{\aaaurl} \url{file:telafuncs.html}
\urldef{\aaburl} \url{file:telafuncsSectioned.html}
\urldef{\aacurl} \url{http://www.geo.fmi.fi/prog/tela/graphex/telagraphs.html}
\urldef{\aadurl} \url{http://www.geo.fmi.fi/}
\title{Tela Help File}
\author{Pekka Janhunen, {\ttfamily Pekka.Janhunen@fmi.fi}}
\date{v1.0, 2 August 1994}
\abstract{Tela ("Tensor Language") is a numerical computing environment.
This is the basic Tela help file. It is read online by the Tela help function.
You can also format and print it as a whole using the Linuxdoc-SGML 'format'
command. See also: 
{\em telafuncs.html} {\tt \aaaurl}
,

{\em telafuncsSectioned.html} {\tt \aaburl}
.

{\em Graphics examples} {\tt \aacurl}
 and 
{\em FMI/GEO home page} {\tt \aadurl}

are also available.
Send Tela bug reports to {\ttfamily tela-bugs@fmi.fi}.
Send suggestions to {\ttfamily tela-suggestions@fmi.fi}.}


\begin{document}
\maketitle
\tableofcontents

\section{Introduction}

Tela ("Tensor Language") is a numerical computing environment. A summary of its
basic features follows:

\begin{verbatim}
        - The syntax reminds C, coming sometimes close to Pascal
        - Some features have been borrowed from Matlab and other languages
        - Tela supports N-dimensional arrays (currently, N must be less than 5)
        - Arrays can be indexed and manipulated as in Matlab and Fortran-90
        - Six basic numerical data types: integer, real and complex arrays
          and scalars
        - Fast. Also scalar loops are pretty fast.
        - Graphics is currently handled using Kenny Toh's PlotMTV program
        - Uses HDF files as native data format
        - Besides in Tela itself, you can also write new functions using C-Tela,
          which is modified C++. Large parts of Tela has been written in C-Tela.
\end{verbatim}





\section{Flow of control statements}


\subsection{if, else\label{if}}

Reserved words: if, else

The if statement syntax:
\begin{tscreen}
\begin{verbatim}
    if (expr) stmt
\end{verbatim}
\end{tscreen}

or
\begin{tscreen}
\begin{verbatim}
    if (expr) stmt else stmt
\end{verbatim}
\end{tscreen}


The expression must evaluate to integer scalar or array. The condition is
considered true if the scalar is nonzero or if all the array elements are
nonzero. Examples:

\begin{tscreen}
\begin{verbatim}
    if (x!=0)
        format("x is nonzero, x=``\n",x)    // notice no semicolon
    else {
        x = 1;
        format("x was zero and is now 1\n");// this semicolon is optional
    };  // this semicolon separates this if stmt from the next stmt, if any
\end{verbatim}
\end{tscreen}





\subsection{for\label{for}}

Reserved word: for

The for statement syntax:

\begin{tscreen}
\begin{verbatim}
    for (init-stmt; continue-condition-expr; update-stmt) stmt
\end{verbatim}
\end{tscreen}


For example:

\begin{tscreen}
\begin{verbatim}
    for (i=1; i<=imax; i++) a[i] = 0.5*i;
\end{verbatim}
\end{tscreen}


The for statement is similar to that found in C language. The only exception
is that in Tela the 'init-stmt' and 'update-stmt' are statements, whereas in C
they are expressions. This is because the assignment '=' is a statement in
Tela, not an operator as in C. There is no comma operator in Tela. If you
want to have several statements as your init or update statement, you have
to use braces, for example:

\begin{tscreen}
\begin{verbatim}
    for ({i=1; j=10}; i<=10; {i++;j--}) format("i=``, j=``\n",i,j);
\end{verbatim}
\end{tscreen}


When expr is considered true, see \ref{if} {(if)}.




\subsection{foreach\label{foreach}}

Reserved word: foreach

The foreach statement syntax:

\begin{tscreen}
\begin{verbatim}
    foreach (i=A) stmt;
\end{verbatim}
\end{tscreen}

will loop over all components of array A, setting i equal to the component
on each iteration.

WARNING: This statement is currently included only for easing transition
from Matlab. The implementation is not optimal. In the future the statement
may even disappear. The Matlab to Tela translator m2t generates foreach
statements, which you should translate to ordinary for statements sooner
or later.

See also: \ref{for} {(for)}




\subsection{repeat, until\label{repeat}}

Reserved words: repeat, until

The repeat..until statement syntax:

\begin{tscreen}
\begin{verbatim}
    repeat stmt-sequence until expr
\end{verbatim}
\end{tscreen}


For example:

\begin{tscreen}
\begin{verbatim}
    i = 1;
    repeat
        format("i = ``\n",i);
        i++
    until i > 10;
\end{verbatim}
\end{tscreen}


This is exactly similar to Pascal, including the use of semicolons.
When expr is considered true, see \ref{if} {(if)}.

See also: \ref{while} {(while)}, \ref{if} {(if)}




\subsection{while\label{while}}

Reserved word: while

The while statement syntax:

\begin{tscreen}
\begin{verbatim}
    while (expr) stmt
\end{verbatim}
\end{tscreen}


For example:

\begin{tscreen}
\begin{verbatim}
    while (!found) {
        LookForIt();
        counter++
    };
\end{verbatim}
\end{tscreen}


The statement is executed until expr evaluates to false.
When expr is considered false or true, see \ref{if} {(if)}.

See also: \ref{repeat} {(repeat)}




\subsection{return\label{return}}

Reserved word: return

The return statement returns from the current function.
Any output variables must be assigned before calling
return, or they will remain undefined.

If called from the main level of a source'd file or
command line, stops execution of the source'd file.

\begin{tscreen}
\begin{verbatim}
    function result=AllPositive(A) {
        for (i=1; i<=length(A); i++)
                        if (A[i]<=0) {
                                result = 0;
                                return
                        };
                result = 1;
    };
\end{verbatim}
\end{tscreen}


The function returns 1 if all elements of vector A are
positive, and zero otherwise.

See also: \ref{function} {(function)}, \ref{break} {(break)}




\subsection{break, continue\label{break}}

Reserved words: break, continue

The break statement exits the surrounding loop.
It is analogous to the break statement of C.
The loop can be a for-loop, a while-loop or a repeat-
until loop.

The continue-statement effectively jumps to the end
of the body of the surrounding loop, causing the next
iteration to begin. It is analogous to the continue
statement of C.

For example,
\begin{tscreen}
\begin{verbatim}
    for (i=1; i<=10; i++) {
        if (i==5) break;
        disp i
    };
\end{verbatim}
\end{tscreen}

outputs the numbers 1,2,3 and 4.
On the other hand,
\begin{tscreen}
\begin{verbatim}
    for (i=1; i<=10; i++) {
        if (i==5) continue;
        disp i
    };
\end{verbatim}
\end{tscreen}

outputs the numbers from one to ten, excluding five.

See also: \ref{return} {(return)}, \ref{goto} {(goto)}.




\subsection{goto, label\label{goto}}

Reserved words: goto, label

The label statement defines a named jump address in instruction stream.
The goto statement jumps to a given label. The labels must be
identifiers. For example,

\begin{tscreen}
\begin{verbatim}
    for (i=1; i<=10; i++) for (j=1; j<=10; j++)
        if (M[i,j] < 0) {
            format("Negative matrix entry, exiting\n");
            goto exitloop;
        };
    label exitloop;
    // processing continues ...
\end{verbatim}
\end{tscreen}

would be equivalent to
\begin{tscreen}
\begin{verbatim}
    if (any(M<0)) format("Negative matrix entry, exiting\n");
\end{verbatim}
\end{tscreen}


The goto statements are local to a function. All goto addresses
must be defined as labels in the function. Gotos into blocks or
loops are currently allowed but may cause unpredictable results.

Think twice before you use goto. In almost all cases it can be
avoided by using the return, break, or continue statements.

See also: \ref{return} {(return)}, \ref{break} {(break)}.




\section{Other statements}


\subsection{call\label{call}}

Reserved word: call

Function call syntax:

\begin{tscreen}
\begin{verbatim}
    [y1,y2..] = f(x1,x2..)
\end{verbatim}
\end{tscreen}

calls f with input arguments x1,x2.. and outputs arguments y1,y2.. .

If f(x1,x2..) appears alone in an expression, it is effectively called
with one output argument, which becomes the value of the expression.
f(x1,x2..) is equivalent to
\begin{tscreen}
\begin{verbatim}
    call(f,x2,x2..).
\end{verbatim}
\end{tscreen}

This notation makes it possible to call functions indirectly through
variables, and to write functionals.

See also: \ref{function} {(function)}




\subsection{function\label{function}}

Reserved word: function
Also uses special tokens: {[} ] ( ) , ... ;

Function definition syntax:

\begin{tscreen}
\begin{verbatim}
    function [y1,y2..] = f(x1,x2..) LOCAL-DECL { stmt-sequence } ;
\end{verbatim}
\end{tscreen}

or
\begin{tscreen}
\begin{verbatim}
    function y = f(x1,x2..) LOCAL-DECL { stmt-sequence } ;
\end{verbatim}
\end{tscreen}

or
\begin{tscreen}
\begin{verbatim}
    function f(x1,x2..) LOCAL-DECL { stmt-sequence } ;
\end{verbatim}
\end{tscreen}

where yi are formal output arguments and xi are formal input arguments.
See local, global for what LOCAL-DECL may be.

By default, output arguments are optional and input arguments are
obligatory. This can be changed by using a semicolon in the argument list.
Identifiers before a semicolon become obligatory and identifiers after
the semicolon are declared optional.

Example:
\begin{tscreen}
\begin{verbatim}
    function [y,z;] = f(x) local(a) { /* ... */ };
\end{verbatim}
\end{tscreen}

declares x, y and z obligatory and 'a' a local variable.

An output argument should be declared obligatory if it is also used
as input in the function.

The ellipsis sign (...) may be appended to the input or output formal
argument list. It is thus possible to define functions with variable
number of arguments. See argin, argout, SetArgOut, Nargin, Nargout
for details.

You can define as many functions in one input file as you wish.
In order to call a Tela-function you must first source the input file
containing that function.

See also: \ref{local} {(local)}, \ref{return} {(return)},
\ref{argin} {(argin)}, \ref{Nargin} {(Nargin)},
\ref{special} {(special)}, \ref{package} {(package)}.




\subsection{local, global\label{local}}

Reserved words: local, global

Variables appearing inside functions, which are not input or output
arguments, are either local or global. If they are local, they are
similar to function arguments, except that they are initialized with
undefined value before entering the function.

The function definition is of the form
\begin{tscreen}
\begin{verbatim}
    function [out1,out2...] = f(in1,in2...) LOCAL-DECL { /* body */ };
\end{verbatim}
\end{tscreen}

where LOCAL-DECL has one of the following five forms:

1. LOCAL-DECL can be empty, in which case all variables are implicitly
local, except autoglobal variables such as pi and eps.

2. LOCAL-DECL can be the keyword 'local'. This is exactly similar to
case 1 above.

3. LOCAL-DECL can be the keyword 'global'. This makes all free variables
in the function body global.

4. LOCAL-DECL can be of the form 'local(a,b,...)'. This makes variables
a,b,... local, and all other free variables global. If an autoglobal
variables such as pi or eps is listed, as in 'local(pi)', it overrides
the autoglobal attribute, making 'pi' local and uninitialized in the
function body. Such practice is not recommended however.

5. LOCAL-DECL can be of the form 'global(a,b,...)'. This makes variables
a,b,... global, and all other free variables implicitly local. Autoglobal
variables remain global.

Examples:

\begin{tscreen}
\begin{verbatim}
function f() /*local*/ {y=2; disp y+pi};
\end{verbatim}
\end{tscreen}


Variable y is local, but pi is not since it is autoglobal. When called,
f() will output 5.14159, and global y has not been affected. Nothing
changes if you uncomment the keyword local.

\begin{tscreen}
\begin{verbatim}
    function V = ddx(u)
    global(imax,dx)
    {
        V = zeros(imax);
        coeff = 1/(2*dx);
        V[2:imax-1] = (u[3:imax] - u[1:imax-2])*coeff;
        V[1] = V[imax-1];
        V[imax] = V[2];
    };
\end{verbatim}
\end{tscreen}

This example computes the numerical derivative. The variables
coeff is local since it is not mentioned in the global list.

See also: \ref{function} {(function)}, \ref{package} {(package)}.




\subsection{package\label{package}}

Reserved word: package

The package mechanism is for hiding names local to a group of
functions and other statements from global access. Syntax:

\begin{tscreen}
\begin{verbatim}
package "mypackage" LOCAL-DECL { /* body */ };
\end{verbatim}
\end{tscreen}


Naming is optional: if the package is not named explicitly,
the current input file name is used. If you use one package per
one input file, you usually don't have to use any name.
LOCAL-DECL is analogous to function definition, see \ref{local} {(local)}.
Usually LOCAL-DECL will be of the form

\begin{tscreen}
\begin{verbatim}
package global(fn1,fn2,...) { /* body */ };
\end{verbatim}
\end{tscreen}


where fn1,fn2,... are the globally visible objects of the package.
They are typically functions, but they can also be variable names.
In the example above, all symbols in body which are not among
fn1,fn2,... are hidden from global reference. This is accomplished
by replacing the symbols with new symbols. The new symbol names
are obtained from the old ones by prepending a prefix "i:", where i
is a positive integer unique to each package name. Since a symbol
name cannot contain a colon, it is impossible to refer to these
symbols from outside. The hidden symbols are also excluded from
name completion and whos() function, for instance.

See also: \ref{function} {(function)}, \ref{local} {(local)}.




\subsection{disp\label{disp}}

Reserved word: disp

\begin{tscreen}
\begin{verbatim}
disp expr;
\end{verbatim}
\end{tscreen}

displays expression on standard output, in the same way as
typing its name on command line.

Disp is primarily a debugging aid. There are better ways to
create output, for example the standard function 'format'.




\section{Intrinsic functions}


\subsection{zeros, izeros, rzeros, czeros\label{zeros}}

Intrinsic functions: zeros, izeros, rzeros, czeros

zeros(nx) returns a real zeroed vector of length nx, zeros(nx,ny)
returns a zeroed real matrix and so on. You can use it to "dimension"
arrays that are indexed later, and to set the initial value to
zero. The companion functions izeros and czeros create integer and
complex zeroed arrays, respectively. rzeros is a synonym for zeros.




\subsection{voids\label{voids}}

Intrinsic function: voids

voids(nx) returns an empty vector of Tela objects; voids(nx,ny)
returns an empty matrix and so on.  A void object can hold any other
Tela object in every position, i.e., voids can represent heterogenous
collections like lists in lisp, though they are indexed like vectors
or arrays.

Voids can be nested.  This is a special case of a void holding another
void.




\subsection{abs\label{abs}}

Intrinsic function: abs

abs(x) returns the absolute value of x, which may be of any numeric
object.  If x is an array, the operation is performed componentwise.
If x is a string, an integer vector of the character ASCII codes
is returned.




\subsection{min, max\label{min}}

Intrinsic functions: min, max

min(x) returns the smallest element of array x, or x itself if x is
scalar.

{[}m,p] = min(x) returns the minimum in m and the minimum position in p.

min(x,y) returns the smaller of x and y. If both are arrays, their
dimensions must agree. If one is scalar, it is promoted to array if
needed.

min(x,y,...,z) effectively expands to min(x,min(y,...z)...).

The function max works similarly. The arguments may not be complex.




\subsection{Nargin, Nargout\label{Nargin}}

Intrinsic functions: Nargin, Nargout

Nargin() returns the number of optional arguments passed to current
function.  Optional arguments are specified using the ellipsis (...)
notation in the formal argument list. You may not use Nargin() in
functions whose argument list does not end with an ellipsis.

Nargout() works similarly. For example, the following function f can
be called with any number of input/output arguments; it displays the
numbers:

\begin{tscreen}
\begin{verbatim}
    function [...] = f(...) {
        format("`` input args, `` output args.\n",Nargin(),Nargout())
    };
\end{verbatim}
\end{tscreen}


See also: \ref{argin} {(argin)}




\subsection{argin, argout, SetArgOut\label{argin}}

Intrinsic functions: argin, argout, SetArgOut

The function argin(n) returns the n-th optional input argument. The
first optional argument has n=1. Similarly, argout(n) returns the
value of the n-th optional output argument.

SetArgOut(n,x) sets the n-th optional output argument to x.

Please notice that the functions argin, argout and SetArgOut do not
count the preceeding 'normal' arguments, but the first optional always
has index n=1.

See also: \ref{Nargin} {(Nargin)}




\subsection{nop\label{nop}}

Intrinsic function: nop

nop() generates a 'no-operation' instruction. It is useful (?) for
benchmarking purposes. For example,

nop(); nop(); nop(); nop(); nop();

generates five NOP instructions in the flatcode stream.




\section{Operators}


\subsection{operators\label{operators}}

The Tela operators, from lower to higher precedence:

\begin{tscreen}
\begin{verbatim}
Operator       Association   Meaning
--------       -----------   -------
:              -             Range creation, ex: 1:10, -5:2.3:7
||             left          Logical OR, similar to C
&&             left          Logical AND, similar to C
!              right         Logical NOT
==, !=         left          Equality and nonequality
>, >=, <, <=   left          Comparisons
+, -           left          Addition and subtraction
*, /, mod      left          Multiplication, real division, modulus
**             left          Matrix multiplication
-, +           right         Unary minus and plus
^              right         Raising to power
\end{verbatim}
\end{tscreen}


The special symbols '++' and '--' are not actually operators but
statements. See help \ref{++} {(++)}.

See also the help for each individual operators:
\ref{:} {(:)}, \ref{||} {({$|$}{$|$})}, \ref{&&} {(\&\&)}, \ref{==} {(==)},
\ref{comparison} {(comparison)}, \ref{+} {(+)}, \ref{-} {(-)},
\ref{*} {(*)}, \ref{/} {(/)}, \ref{mod} {(mod)}, \ref{**} {(**)},
\ref{power} {(power)}, \ref{!} {(!)}.
See also: \ref{special} {(special)} (type help special).




\subsection{special\label{special}}

\begin{tscreen}
\begin{verbatim}
The Tela special characters
---------------------------
\end{verbatim}
\end{tscreen}


\begin{tscreen}
\begin{verbatim}
Char(s)   Meanings, usages
-------   ----------------
( )       Parentheses, expression grouping       (a+b)*(a-c)
[ ]       Brackets, array indexing               A[i,2:N-1]
          Function call                          [a,b] = f(x,y,z);
          Function definition                    function [a,b] = f(x,y,z) {};
<[ ]>     Mapped array indexing                  A<[ivec,jvec]>
#( )      Array constructor                      #(1,2; 3,4)
{ }       Statement grouping                     {i++; j++};
++        Incrementation                         i++;
--        Decrementation                         i--;
;         Statement separator                    {i++; j++};
          Separator in for statement             for (i=1; i<=imax; i++) {};
          Separator of obligatory/optional args  function y=f(x; y) {};
          Separator in array constructor         #(1,2; 3,4)
,         Separator of parameters                x = f(x,y) + 3;
          Separator in array constructor         #(1,2; 3,4)
\end{verbatim}
\end{tscreen}


More information available:
\ref{[} {([)}, \ref{Mapped indexing} {({$<$}[)},
\ref{++} {(++)}.

See also: \ref{operators} {(operators)}.




\subsection{mod\label{mod}}

Reserved word: mod

a mod b gives the modulus, a modulo b. a and b may be also complex; in
that case the modulus is taken separately for the real and imaginary
parts. The modulus operator has the same precedence as the pointwise
multiplication '*' and real division '/'.

See also: \ref{*} {(*)}, \ref{/} {(/)}, \ref{operators} {(operators)}




\subsection{+\label{+}}

Special symbol: +

a + b is the normal addition operator. If both a and b are scalars,
the result is a scalar. If one of them is array, the result is an
array of the same size. If both are arrays, their dimensions must
agree and the result is an array.

Unary plus (+a) returns 'a' as is.

See also: \ref{-} {(-)}, \ref{++} {(++)}, \ref{operators} {(operators)}




\subsection{*\label{*}}

Special symbol: *

a * b is the normal (pointwise) multiplication operator. If both a and
b are scalars, the result is a scalar. If one of them is array, the
result is an array of the same size. If both are arrays, their
dimensions must agree and the result is an array.

See also: \ref{**} {(**)}, \ref{operators} {(operators)}




\subsection{/\label{/}}

Special symbol: /

a / b is the normal (pointwise) division operator. If both a and b are
scalars, the result is a scalar. If one of them is array, the result
is an array of the same size. If both are arrays, their dimensions
must agree and the result is an array.

See also: \ref{*} {(*)}, \ref{mod} {(mod)}, \ref{operators} {(operators)}




\subsection{-\label{-}}

Special symbol: -

a - b is the subtraction operator. If both a and b are scalars, the
result is a scalar. If one of them is array, the result is an array of
the same size. If both are arrays, their dimensions must agree and the
result is an array.

Unary minus (-a) is the negative of a.

See also: \ref{+} {(+)},\ref{++} {(++)},\ref{operators} {(operators)}




\subsection{**\label{**}}

Special symbol: **

T**U is the generalized matrix product of T and U.
If T has components T{[}i,..,j,k] and U has components U{[}a,b,c,...],
then (T**U){[}i,..,j,b,c,...] = sum(T{[}i,..,j,k]*U{[}k,b,c,...],k=1:kmax),
i.e. it is a contraction of tensors T and U with respect to the
innermost dimensions, which must agree. In case of matrices ** thus
gives the ordinary matrix product. If one or both operands are scalars,
T**U is the same as T*U (pointwise multiplication).

See also: \ref{*} {(*)}, \ref{operators} {(operators)}




\subsection{: void range\label{:}}

Special symbol: ':'

a:step:b creates a vector of values \#(a,a+step,a+2*step,...) such that
all elements are less or equal than b. a:b creates a range using unit
step. A lone ':' stands for the Void value. When used as an array
subscript, it stands for the entire range, for example A{[}:,3] refers
to the third column of matrix A.

See also: \ref{operators} {(operators)}




\subsection{++, --\label{++}}

Special symbols: ++, --

The statement a++ increments a by one. The statement a-- decrements a
by one. Notice that these are not operators but statements in Tela.

See also: \ref{+} {(+)}




\subsection{or $|$$|$\label{||}}

Special symbol: $|$$|$

a $|$$|$ b is the logical OR of a and b. The operands must be integer
valued.  For array operand(s), the OR operation is applied
componentwise.

See also: \ref{&&} {(\&\&)}, \ref{operators} {(operators)}




\subsection{and \&\&\label{&&}}

Special symbol: \&\&

a \&\& b is the logical AND of a and b. The operands must be integer
valued.  For array operand(s), the AND operation is applied
componentwise.

See also: \ref{||} {({$|$}{$|$})},\ref{operators} {(operators)}




\subsection{not !\label{!}}

Special symbol: !, operators (type help operators)

!a is the logical NOT of a. The operand must be integer valued. If the
operand is an integer array, the result is integer array also,
otherwise the result is integer scalar.

On command prompt, '!' executes an operating system command, if it is
the first character on line. In t-files this extra meaning of '!' does
not exist, but you must use the 'system' function explicitly.




\subsection{equal, ==, notequal, !=\label{==}}

Special symbols: ==, !=

a == b is the equality test operator. If both operands are scalars,
the result is either 1 or 0. If one operand is numeric array and the
other one is scalar, the result is an integer array of 0's and 1's of
the same size as the array operand. If both operands are arrays of the
same size, the result is again an integer array of 0's and 1's. But if
the operands are arrays of different size, the result is 0 (integer
scalar). Strings are handled as integer arrays of their ASCII codes,
according to their internal representation. For other types of
objects, for example functions, the result is 1 only if the objects
are exactly equal.

a != b is the 'not-equal' operator. It is analogous to '=='.

Unlike order-related comparison operators ({$<$}, {$>$}, {$<$}=, {$>$}=), the '==' and
'!=' operators never generate an error message for any operands.

See also: \ref{comparison} {(comparison)}, \ref{operators} {(operators)}




\subsection{comparison: lt, gt, le, ge\label{comparison}}

Special symbols: $<$, $>$, $<$=, $>$=

These operators obviously test whether the first operand is less than,
greater than, less or equal than, or greater or equal than the second
operand. If both operands are scalars, the result is scalar (1 or 0).
If one operand is array and the other one is scalar, the result is an
integer array of 0's and 1's. If both operands are arrays, their
dimensions must agree. The result is then again an integer array of
0's and 1's.

The operands may not be nonnumeric, nor they may include complex
numbers.

See also: \ref{==} {(==)}, \ref{operators} {(operators)}




\subsection{power\label{power}}

Special symbol: \^{}

a \^{} b is a-raised-to-the-power-b. If both a and b are scalars, the
result is a scalar. If one of them is array, the result is an array of
the same size. If both are arrays, their dimensions must agree and the
result is an array.

See also: \ref{*} {(*)}, \ref{**} {(**)}, \ref{operators} {(operators)}




\subsection{$ {]}\label{[}
{\rm 
\begin{tscreen}
\begin{verbatim}
Special symbols: [, ]

If A is an array, it can be indexed using A[i,j,...,k]. The number of
indices must be equal to rank(A), except that all arrays can be
indexed using just single index, in which case the array is indexed in
flattened form. The flattened indexing is useful e.g. in connection
with find(), because find() returns a flattened index vector. Flat
indexing is also generally faster than ordinary indexing for
multidimensional arrays.

The indices may be of three types. 1) integer scalars, 2) integer vectors,
3) Void value (notation ':'), which means 'whole range'. The rank of the
result is equal to rank(A) - n, where n is the number of integer scalar
indices.
\end{verbatim}
\end{tscreen}


See also: \ref{Mapped indexing} {({$<$}[)}


\subsection{Mapped indexing\label{Mapped indexing}}

\begin{tscreen}
\begin{verbatim}
Special symbols: <[, ]>

Besides ordinary array indexing, accomplished with [ ], you can use
mapped indexing using <[ ]>. Assume A is an array with N = rank(A).
Assume that I1...IN are integer arrays, and that their dimensions
mutually agree. Then A<[I1,I2,...,IN]> is a collection of A's
components, and its size is equal to the size of each Ik.

Unlike ordinary array indexing, in mapped indexing the size of the
result is not determined by A, but the size of the index
arrays. Mapped indexing can not easily be returned to ordinary
indexing, hence it is included as a separate operation in Tela.
\end{verbatim}
\end{tscreen}


See also: \ref{[} {([)}.


\section{Other Tela-related material}


\subsection{telakka\label{telakka}}



telakka (TeLa Kernel Konstruction Accessory) is a program that
generates new Tela kernels. It is implemented as a shell script. It is
used much as a C compiler. C-tela files (.ct files) written by you can
be compiled and linked into Tela using telakka. For example, if
mystuff.ct contains your own function named myfunction,

\begin{tscreen}
\par
\addvspace{\medskipamount}
\nopagebreak\hrule
\begin{verbatim}
unix> cat mystuff.ct
[y] = myfunction(x)
/* This does something really stupid */
{
    cout << "in myfunction, x = " << x << "\n";
    y = 3.14;
    Add(y,y,x);
    return 0;    // successful exit to Tela
}
unix> telakka -o mytela mystuff.ct
unix> ./mytela
This tela is a tensor language, Version 0.5g.
Type  ?help  for help.
->Try: docview(), source("demo")
>help myfunction
This does something really stupid 
>myfunction(42)
in myfunction, x = 42
45.14
>
\end{verbatim} 
\nopagebreak\hrule 
\addvspace{\medskipamount}
\end{tscreen}


The executable mytela is a full Tela plus the C-tela functions from
mystuff.ct. The help command finds the C-style comment /* ... */
following the function header automatically. Also name completion will
recognize myfunction. C-tela code is C++ code with one syntactic
extention: the function header is simpler and follows Tela
conventions. There is a preprocessor, named ctpp, which converts
C-tela to ordinary C++ by transforming function headers. Telakka calls
ctpp, the system C++ compiler and the linker automatically as
needed. You can pass other object files, libraries and C compilation
switches to telakka as you need.

On systems that support DLD dynamic linking there is a faster method
to bring your own code in Tela. Just compile the .ct file with telakka
-c to produce an .o file. Then use the link function in Tela to bring
the functions in Tela executable; in this way you don't have to
generate a full new copy of the kernel. The link function does not
exist on systems that do not have DLD.




\subsection{m2t\label{m2t}}



m2t is Matlab to Tela translator. Usage is as follows:

\begin{tscreen}
\par
\addvspace{\medskipamount}
\nopagebreak\hrule
\begin{verbatim}
       m2t [-h?]                       (give help)
   or: m2t [-sd] <file.m >file.t       (basic usage)
   or: m2t [-FSmsd] [files.m] >file.t  (many files at once)

List of possible command line options:

    -F        Ignore script files (process Functions only)
    -S        Ignore function files (process Scripts only)
    -m        Multi-file mode: generate .t files using .m file basenames
    -s        Silent mode, suppress all warnings
    -d        Suppress matrix division warnings
    -h, -?    Produce this message

Examples:

    m2t -F *.m >funcs.t    Compile all function files into "funcs.t"
    m2t -Sm *.m            Compile all scripts in separate t-files
\end{verbatim} 
\nopagebreak\hrule 
\addvspace{\medskipamount}
\end{tscreen}


Many Tela functions can reside in one source file, therefore m2t
by default writes to standard output. If you want to stick to Matlab
convention and use only one function per file, you use the -m flag.

The -F and -S arguments are useful tools for selecting either only
function M-files or script M-files. The default is to process both
types of files.

It is important to realize that m2t is not a full-fledged translator.
You almost always have to edit its input in order to run it suffesfully
in Tela. Despite this shortcoming, many people have found m2t extremely
useful. Particularly, m2t has difficulties in recognizing vectors built
without commas. For example, in Matlab is is legal to write

\begin{tscreen}
\par
\addvspace{\medskipamount}
\nopagebreak\hrule
\begin{verbatim}
a = [1 2 a+3]
\end{verbatim} 
\nopagebreak\hrule 
\addvspace{\medskipamount}
\end{tscreen}


instead of 

\begin{tscreen}
\par
\addvspace{\medskipamount}
\nopagebreak\hrule
\begin{verbatim}
a = [1,2,a+3].
\end{verbatim} 
\nopagebreak\hrule 
\addvspace{\medskipamount}
\end{tscreen}


However, m2t properly recognizes only the second form except in some
trivial cases. Keep this in mind when writing new M-files. The other weak
point is in deciding which symbol is a function and which symbol is a matrix.
In Matlab, both function calls and array references use (..) parentheses.
In Tela array refererence must use {[}..] brackets. m2t does what is possible
to guess what is function and what is variable, but it sometimes guesses
wrong.

It is also important to realize that m2t can be misused, i.e. it can be
used to translate copyrighted M-files to Tela. This is true with any
translator. You as a user are completely responsible for ensuring that
copyright is not violated when using m2t.







}$}

\end{document}
