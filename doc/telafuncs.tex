\documentclass[a4paper]{article}
\newif\ifpdf
\ifx\pdfoutput\undefined
  \pdffalse % we do not use PDFLaTeX
\else
  \pdfoutput=1 % we use PDFLaTeX
  \pdftrue
\fi
\usepackage{linuxdoc-sgml}
\usepackage{qwertz}
\usepackage{url}
\usepackage[latin1]{inputenc}
\ifpdf
  \usepackage[pdftex,colorlinks=true,urlcolor=blue,linkcolor=blue]{hyperref}
  \pdfcompresslevel=9
\else
  \usepackage{t1enc}
%%  \usepackage[dvips]{hyperref}
\fi
\usepackage[english]{babel}
\usepackage{epsfig}
\usepackage{null}
\def\addbibtoc{
\addcontentsline{toc}{section}{\numberline{\mbox{}}\relax\bibname}
}%end-preamble
\setcounter{page}{1}
\title{Builtin Tela functions}
\author{Pekka Janhunen, {\ttfamily Pekka.Janhunen@fmi.fi}}
\date{Version 2.02, Fri May 20 13:26:28 EEST 2005}
\abstract{This document has been automatically generated from the online
help messages of builtin Tela functions.}


\begin{document}
\maketitle
\tableofcontents

\section{INTRODUCTION}

Negative error codes describe fatal errors, whereas positive error
codes are warnings only. In case of fatal error, the function
output arguments have undefined values. If a function has no output
arguments, it typically has only positive return codes.

The error code descriptions do normally not appear in the Tela
help messages. If an error occurs, Tela will however find and
display the proper textual message.




\subsection{BatchMode\label{BatchMode}}

\begin{tscreen}
\begin{verbatim}
[y] = BatchMode()
 BatchMode() returns 1 if this Tela process is in batch mode
   (command line switch -b), otherwise 0.
\end{verbatim}

See also: \ref{SilentMode} {(SilentMode)}, \ref{VerboseMode} {(VerboseMode)}, \ref{UsingReadline} {(UsingReadline)}.
\end{tscreen}





\subsection{ChebyshevT\label{ChebyshevT}}

\begin{tscreen}
\begin{verbatim}
[w] = ChebyshevT(n,x)
 ChebyshevT(n,x) gives the nth Chebyshev polynomial of the first kind,
   ChebyshevT(n,cos(theta)):= cos(n*theta).
   Error codes:
   -1: First arg not numerical
   -2: Second arg not numerical
   -5: Argument dimensions do not match
 
\end{verbatim}
\end{tscreen}





\subsection{CheckReadOnlyMode\label{CheckReadOnlyMode}}

\begin{tscreen}
\begin{verbatim}
[;y] = CheckReadOnlyMode(;x)
 CheckReadOnlyMode() returns 1 if this Tela process checks that function
   input arguments are not modified by the function. The check is done by
   default, but if it causes problems (bugs in Tela) you can turn it off
   using CheckReadOnlyMode(off).
   CheckReadOnlyMode returns the old mode setting.
\end{verbatim}

See also: \ref{SilentMode} {(SilentMode)}, \ref{BatchMode} {(BatchMode)}, \ref{VerboseMode} {(VerboseMode)}.
\begin{verbatim}
   Error codes:
   1: Argument not integer 
\end{verbatim}
\end{tscreen}





\subsection{DebugQueryMode\label{DebugQueryMode}}

\begin{tscreen}
\begin{verbatim}
[;y] = DebugQueryMode(;x)
 DebugQueryMode() returns 1 if the debug query mode
   has been set, otherwise 0.
   DebugQueryMode(on) and DebugQueryMode(off) change
   the current setting.

   If debug query mode is on, the program will ask whether
   the user wants to enter debug mode if control-C is
   pressed (INT signal sent). Otherwise the program simply
   stops on pressing control-C.
   Error codes:
   1: Argument not integer 
\end{verbatim}
\end{tscreen}





\subsection{EPOCHbreakdown\label{EPOCHbreakdown}}

\begin{tscreen}
\begin{verbatim}
[year,month,day,hour,minute,second,msec] = EPOCHbreakdown(epoch)
 [year,month,day,hour,minute,second,msec] = EPOCHbreakdown(epoch)
   breaks down the epoch value (number of milliseconds since
   01-Jan-0000 00:00:00.000) into year (4-digit), month (1-12), day (1-31),
   minute (0-59), second (0-59) and millisecond (0-999) values (integers).
   The argument must be a real scalar or array.
\end{verbatim}

See also: \ref{computeEPOCH} {(computeEPOCH)}, \ref{parseEPOCH} {(parseEPOCH)}, \ref{encodeEPOCH} {(encodeEPOCH)}.
\begin{verbatim}
   Error codes:
   -1: Argument not real scalar or array
\end{verbatim}
\end{tscreen}





\subsection{FFT\label{FFT}}

\begin{tscreen}
\begin{verbatim}
[f] = FFT(u; dim)
 FFT(u) gives the complex Fast Fourier Transform of u.
   If u's rank is more than one, the transform is computed
   only along the first dimension (many independent 1D
   transforms).

   FFT(u,dim) computes the FFT along the specified dimension.
   The first dimension is labeled 1 and so on.

   For vector u, f=FFT(u) is equivalent with

   n = length(u); f = czeros(n);
   for (j=1; j<=n; j++)
       f[j] = sum(u*exp(-(j-1)*(0:n-1)*2i*pi/n));

   All Fourier transform functions in Tela can take the transform
   along any dimension in a multidimensional array, and the transform
   length is not restricted. The function FFT should be used only in
   case of complex input data. Use realFFT for real input array.

   Functions FFT, realFFT, sinqFFT, cosFFT and their inverses
   are the most efficient when the transform length n is a product
   of small primes.

   Functions sinFFT and invsinFFT are efficient when n+1 is
   a product of small primes
           
   Functions cosFFT and invcosFFT are efficient when n-1 is
   a product of small primes
           
\end{verbatim}

See also: \ref{invFFT} {(invFFT)}, \ref{realFFT} {(realFFT)}, \ref{sinFFT} {(sinFFT)}, \ref{cosFFT} {(cosFFT)}, \ref{sinqFFT} {(sinqFFT)}, \ref{cosqFFT} {(cosqFFT)}, \ref{Lomb} {(Lomb)}.
\begin{verbatim}
   Error codes:
   -1: First argument not a numeric array
   -2: Second argument not integer
   -3: Second argument out of range
   
\end{verbatim}
\end{tscreen}





\subsection{Gauss2F1\label{Gauss2F1}}

\begin{tscreen}
\begin{verbatim}
[w] = Gauss2F1(a,b,c,z)
 Gauss2F1(a,b,c,z) is Gauss' 2F1 hypergeometric function
   F(a,b;c;z), defined by the power series
      F(a,b;c;z) = sum(((a)_n*(b)*n/(c)_n)*z^n/n!,n=0..Inf)
   where (a)_n is the Pochhammer symbol poch(a,n),
      (a)_n = a*(a+1)*...*(a+n-1) = Gamma(a+n)/Gamma(a).
   The function has a singularity at z=1 and a branch cut
   in Re(z) > 1, Im(z) == 0. The function is evaluated for
   any numerical arguments and is threaded automatically over
   array arguments. The method of evaluation is power series
   expansion for small |z|, a transformation formula for certain
   large |z|, and direct integration of the defining diff. equation
   in other cases.
\end{verbatim}

See also: \ref{Kummer1F1} {(Kummer1F1)}.
\begin{verbatim}
   Error codes:
   -1: Args must be complex scalars
   -2: Second arg not numerical
   -3: Third arg not numerical
   -4: Fourth arg not numerical
   -5: Argument dimensions do not match
\end{verbatim}
\end{tscreen}





\subsection{GetInstructionData\label{GetInstructionData}}

\begin{tscreen}
\begin{verbatim}
[Ninstructions,Noperations] = GetInstructionData(p,mnemo)
 [Ninstr,Nops] = GetInstructionData(p,"mnemonic") returns
   the number of specific instructions executed when p=perf() was
   gathered, and the number of (floating point) operations associated
   with the instruction.
   The mnemonic must be a string, and it may be any of the names
   that appear in the disasm(f) listing. In addition, mnemonic may
   be "flop", which includes all instructions that may perform floating
   point arithmetic, or it may be "*", which includes all instructions.
   Error codes:
   -1: Bad first argument, must be obtained from perf()
   -2: Second arg not a string
   -3: Second arg not a recognized string
   
\end{verbatim}
\end{tscreen}





\subsection{HDFNewMode\label{HDFNewMode}}

\begin{tscreen}
\begin{verbatim}
[;y] = HDFNewMode(;x)
 HDFNewMode(1) causes subsequent save calls to write the data
   in "new" style HDF files, using the SD rather than SDS interface,
   which is nowadays the default.
   The old style SDS interface is used by subsequent saves after
   calling HDFNewMode(0).
   
   This function works only if you have compiled Tela with HDF4.0b1
   or newer and Tela version 1.23 or later. Whether this is the case
   can be tested by calling NewHDFSupported().
   
\end{verbatim}

See also: \ref{NewHDFSupported} {(NewHDFSupported)}, \ref{save} {(save)}.
\begin{verbatim}
   Error codes:
   1: Argument not integer
   2: New HDF mode not supported by this Tela
   
\end{verbatim}
\end{tscreen}





\subsection{HeavisideTheta\label{HeavisideTheta}}

\begin{tscreen}
\begin{verbatim}
[y] = HeavisideTheta(x)
 y = HeavisideTheta(x) returns 1 if x>=0 and 0 if x<0.
   x must be real. If x is array, the operation is applied componentwise.
\end{verbatim}

See also: \ref{sign} {(sign)}.
\begin{verbatim}
   Error codes:
   -1: Complex or nonnumeric input argument 
\end{verbatim}
\end{tscreen}





\subsection{Im\label{Im}}

\begin{tscreen}
\begin{verbatim}
[y] = Im(x)
 y = Im(x) takes the imaginary part of a complex quantity x.
   If x is real or integer, the result is zero.
   If x is an array, the operation is applied componentwise.
\end{verbatim}

See also: \ref{Re} {(Re)}, \ref{conj} {(conj)}, \ref{arg} {(arg)}.
\begin{verbatim}
   Error codes:
   -1: Nonnumeric input argument 
\end{verbatim}
\end{tscreen}





\subsection{IncompleteGammaP\label{IncompleteGammaP}}

\begin{tscreen}
\begin{verbatim}
[w] = IncompleteGammaP(a,x)
 IncompleteGammaP(a,x) is the regularized incomplete gamma function
     IncompleteGammaP(a,x) := integrate(exp(-t)*t^(a-1),t=0..x)/Gamma(a).
   The evaluation is reliable for Re(a) > 0 only, although a result
   is returned also for Re(a) <= 0.
   The notation conforms with Numerical Recipes.
   Error codes:
   -1: First arg not numerical
   -2: Second arg not numerical
   -5: Argument dimensions do not match
 
\end{verbatim}
\end{tscreen}





\subsection{IncompleteGammaQ\label{IncompleteGammaQ}}

\begin{tscreen}
\begin{verbatim}
[w] = IncompleteGammaQ(a,x)
 IncompleteGammaQ(a,x) is the regularized incomplete gamma function
     IncompleteGammaQ(a,x) := 1 - IncompleteGammaP(a,x).
   The evaluation is reliable for Re(a) > 0 only, although a result
   is returned also for Re(a) <= 0.
   The notation conforms with Numerical Recipes.
   Error codes:
   -1: First arg not numerical
   -2: Second arg not numerical
   -5: Argument dimensions do not match
 
\end{verbatim}
\end{tscreen}





\subsection{Kummer1F1\label{Kummer1F1}}

\begin{tscreen}
\begin{verbatim}
[w] = Kummer1F1(a,c,z)
 Kummer1F1(a,c,z) is the 1F1 confluent hypergeometric function,
   defined by
     1F1(a,c,z) = sum(((a)_n/(c)_n)*z^n/n!,n=0..Inf)
   where (a)_n is the Pochhammer symbol poch(a,n),
      (a)_n = a*(a+1)*...*(a+n-1) = Gamma(a+n)/Gamma(a).
   The function has no singularities or branch cuts for any z,
   but is infinite when c is a non-positive integer.
   The method of evaluation is power series expansion for
   small and moderate |z| and direct integration of the defining
   diff. equation in other cases.
\end{verbatim}

See also: \ref{Kummer1F1Regularized} {(Kummer1F1Regularized)}, \ref{Gauss2F1} {(Gauss2F1)}.
\begin{verbatim}
   Error codes:
   -1: First arg not numerical
   -2: Second arg not numerical
   -3: Third arg not numerical
   -5: Argument dimensions do not match
\end{verbatim}
\end{tscreen}





\subsection{Kummer1F1Regularized\label{Kummer1F1Regularized}}

\begin{tscreen}
\begin{verbatim}
[w] = Kummer1F1Regularized(a,c,z)
 Kummer1F1Regularized(a,c,z) is the regularized confluent
   hypergeometric function, defined by
   Kummer1F1Regularized(a,c,z):= Kummer1F1(a,c,z)/Gamma(c).
   This function is finite for all finite values of the parameters.
   The evaluation method works also when c is negative integer.
\end{verbatim}

See also: \ref{Kummer1F1} {(Kummer1F1)}.
\begin{verbatim}
   Error codes:
   -1: First arg not numerical
   -2: Second arg not numerical
   -3: Third arg not numerical
   -5: Argument dimensions do not match
\end{verbatim}
\end{tscreen}





\subsection{LU\label{LU}}

\begin{tscreen}
\begin{verbatim}
[L;U,P] = LU(A)
 [L,U,P] = LU(A) computes the LU factorization of matrix A.
   The factorization is A = P**L**U, where P is a permutation
   matrix, L is lower triangular with unit diagonal and U is
   upper triangular.
   [lu] = LU(A) leaves the factors L and U packed in one matrix.
   [lu,p] = LU(A) returns also the pivoting info vector p.
   (Notice that this p is related to the permutation matrix P
   but is not the same. You need this form of LU if you want to
   use LUbacksubst later on.)
\end{verbatim}

See also: \ref{LUbacksubst} {(LUbacksubst)}, \ref{linsolve} {(linsolve)}, \ref{inv} {(inv)}, \ref{chol} {(chol)}, \ref{SVD} {(SVD)}.
\begin{verbatim}
   Error codes:
   1: Singular matrix (==> zero in U's diagonal)
   -1: Input arg not an array
   -2: Input arg not a rank-2 array (matrix)
   
\end{verbatim}
\end{tscreen}





\subsection{LUbacksubst\label{LUbacksubst}}

\begin{tscreen}
\begin{verbatim}
[x] = LUbacksubst(lu,p,b)
 LUbacksubst(lu,p,b) solves the linear system A**x == b,
   where A has been previously LU-decomposed using LU:
   [lu,p] = LU(A).
\end{verbatim}

See also: \ref{linsolve} {(linsolve)}, \ref{LU} {(LU)}.
\begin{verbatim}
   Error codes:
   -1: First input arg is not an array
   -2: First input arg is not a square matrix
   -3: Second input arg is not an integer vector
   -4: Third input arg is not an array
   -5: Third input arg is not a vector or matrix
   -6: Incompatible dimensions in first/third args
   -7: Singular matrix
   -8: Incompatible dimensions in first/second args
   -9: Internal error
\end{verbatim}
\end{tscreen}





\subsection{LegendreP\label{LegendreP}}

\begin{tscreen}
\begin{verbatim}
[w] = LegendreP(n,m;x)
 LegendreP(n,x) is the nth Legendre polynomial P_n(x)
   evaluated at x.
   LegendreP(n,m,x) is the associated Legendre function
   P_n^m(x).
   The Arfken definition is in use, which is different from
   e.g. what Mathematica uses.
   Error codes:
   -1: First arg not numerical
   -2: Second arg not numerical
   -3: Third arg not numerical
   -5: Argument dimensions do not match
\end{verbatim}
\end{tscreen}





\subsection{Lomb\label{Lomb}}

\begin{tscreen}
\begin{verbatim}
[freq,ampl;normalization,prob] = Lomb(x,y; dx,oversampling)
   [freq,ampl] = Lomb(t,u)
   The Lomb periodogram is power spectrum analysis of unevenly sampled data.
   The argument t must be monotonically increasing "time" axis and u the
   corresponding real data vector. The output freq is the frequency axis and
   ampl the normalized Lomb power spectrum. The result is normalized so that
   sum(ampl)*(freq[2]-freq[1]) approximates the mean squared signal mean(u^2)
   and in case of evently spaced t, ampl is close to the FFT power spectrum P:
   P=2*(t[2]-t[1])*abs2(realFFT(u))/length(u); P[length(P)]*= 0.5; P[1]*= 0.5.

     [freq,ampl] = Lomb(t,u,dt,oversampling)
   sets the finest wanted t resolution dt (default is min(diff(dt))) and the
   oversampling factor (default 4.0).

     [freq,ampl,normalization,prob] = Lomb(...)
   returns also normalized and prob, where normalization is a scalar which is
   such that normalization*ampl gives the dimensionless Lomb spectrogram
   conforming to Numerical Recipes convention. The fourth output arg 'prob' is
   the probability that the spectral power would occur by chance under the
   assumption of Gaussian noise; if 'prob' is small at obtained peak, the peak
   is statistically significant. Lomb uses a fast n*log(n) algorithm due to
   Press and Rybicki. The implementation is loosely based on the Numerical
   Recipes book. Lomb returns also zero frequency and the corresponding
   amplitude which is related to mean(u) while the N.R. routine does not return
   the zero frequency part.
   Example:
      n = 100; t = sort(n*rand(n)); u = 2*sin(128*pi*t/n);
      [f,a,norm,prob] = Lomb(t,u,0.25); plot(f,a);
   The spectral peak is clear, although not easily seen plot(t,u). The
   probability that the peak occurs by chance is seen from plot(f,prob); it is
   about 1e-18 in this case.
\end{verbatim}

See also: \ref{FFT} {(FFT)}, \ref{realFFT} {(realFFT)}.
\begin{verbatim}
   Error codes:
   -1: First input arg t is not real vector of length at least 2
   -2: Second input arg u is not real vector
   -3: Lengths of input vectors are different
   -4: Third input arg 'dt' is not positive real scalar
   -5: Fourth input arg 'oversampling' is not positive real scalar
\end{verbatim}
\end{tscreen}





\subsection{NewHDFSupported\label{NewHDFSupported}}

\begin{tscreen}
\begin{verbatim}
[y] = NewHDFSupported()
 NewHDFSupported() returns 1 if this Tela was compiled with
   HDF4.0b1 or later. If this is the case, then the following
   become possible:
   - HDFNewMode(1) can be used to save HDF data in "new" format
     (SD rathern than SDS interface). This is now the default!
   - Unidata netCDF files can be loaded/imported in the same
     way as HDF SDS/SD files
   
   The main benefit of using the new format is that no precision
   loss occurs because 64-bit numbers are used for real data
   (the old HDF interface only supports 32-bit data).
   
   The main drawback of using the new format is that old Tela
   versions and Tela versions which are still compiled with
   HDF3.1r5 can not read the files. The old style files can be
   read by all Tela versions.
   
\end{verbatim}

See also: \ref{HDFNewMode} {(HDFNewMode)}, \ref{save} {(save)}.
\end{tscreen}





\subsection{Re\label{Re}}

\begin{tscreen}
\begin{verbatim}
[y] = Re(x)
 y = Re(x) takes the real part of a complex quantity x.
   If x is real or integer, it is returned as such.
   If x is an array, the operation is applied componentwise.
\end{verbatim}

See also: \ref{Im} {(Im)}, \ref{conj} {(conj)}, \ref{arg} {(arg)}.
\begin{verbatim}
   Error codes:
   -1: Nonnumeric input argument 
\end{verbatim}
\end{tscreen}





\subsection{SVD\label{SVD}}

\begin{tscreen}
\begin{verbatim}
[U;S,V] = SVD(A)
 [U,S,V] = SVD(A) computes the singular value
   decomposition of matrix A: A = U**S**V'.
   U and V are unitary and S is diagonal.
   SVD(A) as such returns the vector of singular values.
\end{verbatim}

See also: \ref{LU} {(LU)}, \ref{chol} {(chol)}.
\begin{verbatim}
   Error codes:
   1: No convergence
   -1: Input arg not an array
   -2: Input array is not a matrix
   -3: Internal error
   -4: Two output arg case not supported (must be 1 or 3)
   
\end{verbatim}
\end{tscreen}





\subsection{SilentMode\label{SilentMode}}

\begin{tscreen}
\begin{verbatim}
[y] = SilentMode()
 SilentMode() returns 1 if this Tela process is in silent mode
   (command line switch -s), otherwise 0.
\end{verbatim}

See also: \ref{BatchMode} {(BatchMode)}, \ref{VerboseMode} {(VerboseMode)}, \ref{CheckReadOnlyMode} {(CheckReadOnlyMode)}.
\end{tscreen}





\subsection{UsingReadline\label{UsingReadline}}

\begin{tscreen}
\begin{verbatim}
[y] = UsingReadline()
 UsingReadline() returns 1 if this Tela was compiled
   to use the GNU readline library, otherwise 0.
\end{verbatim}

See also: \ref{BatchMode} {(BatchMode)}, \ref{SilentMode} {(SilentMode)}, \ref{VerboseMode} {(VerboseMode)}.
\end{tscreen}





\subsection{VerboseMode\label{VerboseMode}}

\begin{tscreen}
\begin{verbatim}
[;y] = VerboseMode(;x)
 VerboseMode() returns 1 if this Tela process is in verbose mode
   (command line switch -v), otherwise 0.
   VerboseMode(on) and VerboseMode(off) set the verbose mode on
   and off, respectively. They return the old mode setting.
\end{verbatim}

See also: \ref{SilentMode} {(SilentMode)}, \ref{BatchMode} {(BatchMode)}.
\begin{verbatim}
   Error codes:
   1: Argument not integer 
\end{verbatim}
\end{tscreen}





\subsection{abs2\label{abs2}}

\begin{tscreen}
\begin{verbatim}
[y] = abs2(x)
 y = abs2(x) computes the square of the absolute value of x.
   If x is real or integer, the result is just the square of x.
   If x is complex, the result is equal to x*conj(x), except that
   the latter would produce a complex result with zero or tiny
   (because of possible roundoff error) imaginary part.
   If x is an array, the operation is applied componentwise.
   Error codes:
   -1: Nonnumeric input argument 
\end{verbatim}
\end{tscreen}





\subsection{accum\label{accum}}

\begin{tscreen}
\begin{verbatim}
[u] = accum(...)
 [u] = accum(i,j,...,v)
   is the same as u<[i,j,...]>+= v
   except that repeated indices are handled by sequential
   increments rather than the last assignment only being effective
   as in u<[i,j,...]>+= v. Notice that the v argument must be
   an array of same size as i,j,..., a scalar v is not accepted.
   (Hint: to promote scalar into array, add zeros array to it.)
   Error codes:
   -30: Too few input args to accum (min is 2)
   -1: I/O arg u (left-hand-side) is non-numeric
   -2: I/O arg u is scalar but last input arg is not
   -3: I/O arg u is scalar but one of the indices is not 1, : or #(1)
   -4: Mismatch between rank of I/O arg u and number of indices
   -5: Scalar index out of range
   -6: Invalid IntArray used as index, should be one-dimensional
   -7: Vector index out of range
   -8: Invalid index, should be integer or IntArray
   -9: Accumulating non-integer into integer array
   -10: Accumulating non-real into real array
   -11: Accumulating non-scalar into complex array
   -12: Too large dimensional accum, not yet implemented (internal error)
   -14: Type mismatch
   -22: Index vector length disagrees with 'RHS' length
\end{verbatim}
\end{tscreen}





\subsection{acos\label{acos}}

\begin{tscreen}
\begin{verbatim}
[y] = acos(x)
 y = acos(x) computes the arc cosine of x.
   If x is complex, the result is complex, otherwise real.
   If x is an array, the operation is applied componentwise.
\end{verbatim}
\end{tscreen}





\subsection{all\label{all}}

\begin{tscreen}
\begin{verbatim}
[y] = all(x)
 all(x) returns 1 if all elements of x are nonzero,
   and 0 otherwise.
   x must be an integer array or scalar.

   Conditional statements (if, while, for, until) assume
   implicit all, for example saying
       if (A > 0) ...
   is equivalent of saying
       if (all(A > 0)) ... ,
   thus you will need all less frequently than you need any.
   
\end{verbatim}

See also: \ref{any} {(any)}.
\begin{verbatim}
   Error codes:
   1: Argument not integer or IntArray 
\end{verbatim}
\end{tscreen}





\subsection{aniclose\label{aniclose}}

\begin{tscreen}
\begin{verbatim}
[] = aniclose(ap)
 aniclose(ap) closes a previously opened animation file.
   Error codes:
   1: Argument not a valid open animation handle
\end{verbatim}
\end{tscreen}





\subsection{aniopen\label{aniopen}}

\begin{tscreen}
\begin{verbatim}
[ap] = aniopen(fn,rw...)
 aniopen("outfile","w") opens an MPEG or MNG animation file for writing.
   It returns an animation handle that can be passed to aniput
   for writing the frames.

   You can also give option-value pairs as additional arguments,
   the following are recognized (Example values are the defaults):

   aniopen("outfile","w",
     "animtype","mpeg-1",    // "MPEG-1", "MPEG-2" or "MNG" (case insensitive)
     "bytes_per_frame",12e3, // bytes per frame to write, approximately
     "framerate",7,          // frames per second to show when showing
     "I_frames_only",0,      // if flag is 1, write I-frames only
     "verbose",0)            // give some progress report to stderr

   The rates must be positive reals or ints. The flags must be passed as
   0 or 1. Set "I_frames_only" to 1 if you want to view all frames with
   Xanim. Most widely available tools currently show only MPEG-1 animations,
   not MPEG-2 so that default type is still set to "MPEG-1".

   If animtype is "MNG", all other options are ignored (except being
   syntactically checked). MNG animation always uses lossless compression.
   This version writes a VLC (Very Low Complexity) MNG animation, which
   uses adaptive filtering and the default ZLIB compression level.
   Future Tela versions may offer additional options for controlling
   the type of the MNG output (e.g., Delta-PNG animations).

   Error codes:
   -1: First argument not a string
   -2: Second argument not a string
   -3: Cannot open animation file
   -4: Second argument not equal to "w" or "r"
   5: Error with animation options
\end{verbatim}

See also: \ref{aniput} {(aniput)}, \ref{aniclose} {(aniclose)}.
\end{tscreen}





\subsection{aniput\label{aniput}}

\begin{tscreen}
\begin{verbatim}
[] = aniput(ap,r,g,b)
 aniput(ap,r,g,b) writes a new frame to a previously
   opened animation (ap=handle returned by aniopen).
   The first frame defines the image size of the animation.
   Error codes:
   -1: First argument not a valid open animation handle
   -2: Second argument not an int matrix
   -3: Third argument not an int matrix
   -4: Fourth argument not an int matrix
   -5: r,g,b arguments not of the same size with each other
   -6: Image size does not match with first frame
\end{verbatim}

See also: \ref{aniopen} {(aniopen)}, \ref{aniclose} {(aniclose)}.
\begin{verbatim}
\end{verbatim}
\end{tscreen}





\subsection{annotate\label{annotate}}

\begin{tscreen}
\begin{verbatim}
[] = annotate(primitive...)
 annotate("primitive"[,options]) adds MTV annotations to the previous
   graph. The plot command(s) and the annotate command(s) must appear
   inside hold(on) ... hold(off) in order to work correctly.
   
   --------------------------------------------
   Here may be the proper place to list the most frequently used
   PlotMTV options. These are used by giving them as optional args
   (option-value pairs) to graphics functions.  See the PlotMTV data
   format definition (DataFormat.ps) for all the available options.
   
   Option name(s)      Possible value(s)       Explanation
   "xlabel"            string                  X-axis label
   "ylabel"            string                  Y-axis label
   "zlabel"            string                  Z-axis label
   "toplabel"          string                  Plot title
   "subtitle"          string                  Text below title
   "comment"           string                  Text on right corner
   "xmin","xmax"       real number             X-axis min,max
   "ymin","ymax"       real number             Y-axis min,max
   "zmin","zmax"       real number             Z-axis min,max
   "xgrid"             real vector             X gridpoints (nonuniform)
   "ygrid"             real vector             Y gridpoints (nonuniform)
   "cmin","cmax"       real number             Contour min/max
   "nsteps"            integer                 Number of contours
   "cstep"             real number             Contour spacing
   "contstyle"         1: normal contours, 2: colored (pcolor),
                       3: 3D surface
   "hiddenline"        "true": colored 3D surface, "false": wireframe
   "linecolor"         color value (integer)
   "linewidth"         integer
   "linestyle"         solid,dashed etc.; integers; 0 is no line
   "markertype"        integers; 0 is no marker (the default)
   "markersize"        real number
   "markercolor"       color value (integer)
   
   Some color (red,green,blue,pink etc.) names have been defined
   in telainit.t. Use them for clarity if possible.
   
   The most usual PlotMTV command line options (always start with
   minus sign):
   -3d                  Initially view in 3D
   -colorps             Produce color PostScript
   -landscape           Produce landscape (rotated) PostScript
   -scale s             PostScript scale factor, default 1
   -nodate              Drop the date from PostScript figure
   -title 'my title'    PlotMTV window title
   
   For an example in using the annotations, see e.g. the file
   "3windows.t", usual location is /usr/local/lib/tela/t/.
   
\end{verbatim}

See also: \ref{plotopt} {(plotopt)}, \ref{hold} {(hold)}, \ref{plot} {(plot)}, \ref{pcolor} {(pcolor)}, \ref{mesh} {(mesh)}, \ref{contour} {(contour)}, \ref{vplot} {(vplot)}, \ref{colormap} {(colormap)}.
\begin{verbatim}
   Error codes:
   1: Could not open temporary MTV file
   2: First argument not a string
   3: Syntax error in graph options 
\end{verbatim}
\end{tscreen}





\subsection{any\label{any}}

\begin{tscreen}
\begin{verbatim}
[y] = any(x)
 any(x) returns 1 if at least one element of x is nonzero,
   and 0 otherwise.
   x must be a integer array or scalar.

   Conditional statements (if, while, for, until) default
   to implicit all, thus you sometimes need to use any
   with these statements.
   
\end{verbatim}

See also: \ref{all} {(all)}.
\begin{verbatim}
   Error codes:
   1: Argument not integer or IntArray 
\end{verbatim}
\end{tscreen}





\subsection{applyfilter\label{applyfilter}}

\begin{tscreen}
\begin{verbatim}
[U] = applyfilter(u,c,d)
 U = applyfilter(u,c,d)
   applies a IIR (Infinite Impulse Response) digital filter
   to time series real vector u, producing another vector U.
   The filter coefficients c,d must be real vectors.
   The filter formulas is
     U[n] = c**u[n-k] + d**U[n-j]
   where k=0:length(c)-1, j=1:length(d)
   and starts effects are handled by repeating the first
   component backwards.
   Error codes:
   -1: First input arg must be real vector
   -2: Second input arg must be real vector
   -3: Third input arg must be real vector
\end{verbatim}
\end{tscreen}





\subsection{arg\label{arg}}

\begin{tscreen}
\begin{verbatim}
[phi] = arg(z)
 arg(z) returns the argument of a complex quantity
   (in radians). The result is between -pi and pi.
   If z is a complex array, the operation is applied
   componentwise.
\end{verbatim}

See also: \ref{Re} {(Re)}, \ref{Im} {(Im)}, \ref{conj} {(conj)}.
\begin{verbatim}
   Error codes:
   -1: Argument has undefined value
   1: Nonnumeric argument 
\end{verbatim}
\end{tscreen}





\subsection{asin\label{asin}}

\begin{tscreen}
\begin{verbatim}
[y] = asin(x)
 y = asin(x) computes the arc sine of x.
   If x is complex, the result is complex, otherwise real.
   If x is an array, the operation is applied componentwise.
\end{verbatim}
\end{tscreen}





\subsection{atan\label{atan}}

\begin{tscreen}
\begin{verbatim}
[y] = atan(x)
 y = atan(x) computes the arc tangent of x.
   If x is complex, the result is complex, otherwise real.
   If x is an array, the operation is applied componentwise.
\end{verbatim}

See also: \ref{atan2} {(atan2)}.
\end{tscreen}





\subsection{atan2\label{atan2}}

\begin{tscreen}
\begin{verbatim}
[z] = atan2(y,x)
 z = atan2(y,x) computes the arcus tangent of y/x using the signs
   of both arguments to determine the quadrant of the return value.
   The input argument must be integer or real scalars and the
   return value is real. They can also be real arrays; in this case
   both must be arrays of the same length.
\end{verbatim}

See also: \ref{atan} {(atan)}.
\begin{verbatim}
   Error codes:
   -1: Bad input argument
   -2: Input arrays must be of same length
\end{verbatim}
\end{tscreen}





\subsection{autoglobal\label{autoglobal}}

\begin{tscreen}
\begin{verbatim}
[] = autoglobal(...)
 autoglobal("sym1","sym2",...) sets the 'autoglobal' attribute
   to given symbols. The symbols are then globally accessible
   from inside packages and functions without need to explicitly
   declaring them global. Constants such as pi carry the autoglobal
   attribute automatically, but with this function it is possible
   for the user to define more autoglobal symbols.

   Warning: This function is for experts only. It is not advisable
   to say autoglobal("x"), for instance, because many existing code
   may then break down.
\end{verbatim}

See also: \ref{hide} {(hide)}.
\begin{verbatim}
   Error codes:
   1: Argument not a string 
\end{verbatim}
\end{tscreen}





\subsection{autosource\label{autosource}}

\begin{tscreen}
\begin{verbatim}
[] = autosource(fn...)
 autosource("file.t","name1","name2",...) tags symbols
   name1, name2,... such that the command source("file.t")
   is effectively executed when any of the symbols name
   is used. This is load-on-demand.
\end{verbatim}

See also: \ref{source} {(source)}.
\begin{verbatim}
   Error codes:
   1: Argument not a string 
\end{verbatim}
\end{tscreen}





\subsection{avoid\_constructors\label{avoid_constructors}}

\begin{tscreen}
\begin{verbatim}
[] = avoid_constructors(flag)
 avoid_constructors(on) affects the operation
   of subsequent t2ct function calls.
   When this flag is on, the generated C-tela (C++)
   code avoids constructing class objects on the stack
   but instead declares these objects as pointers and
   initializes them using the 'new' operator.
   This mode compiles on ATT Cfront versions that do not
   support goto statements in blocks having constructors.

   However, in some rare cases avoid_constructors(on)
   is known to produce incorrect code. The default is
   avoid_constructors(off).
\end{verbatim}

See also: \ref{t2ct} {(t2ct)}.
\begin{verbatim}
   Error codes:
   1: Argument not an integer
\end{verbatim}
\end{tscreen}





\subsection{axpy\label{axpy}}

\begin{tscreen}
\begin{verbatim}
[y] = axpy(a,x)
 [y] = axpy(a,x) is the equivalent of y = y + a*x
   for scalar a and (real or complex) arrays x and y.
   axpy calls a fast BLAS-1 routine if available.
   Error codes:
   -1: Input/output arg y is not real/complex array
   -2: First input arg a is not a scalar
   -3: Second input arg x is not real/complex array
   -4: Second input arg x has size incompatible with y
   -5: If y is real a and x must also be real
   -6: If y is complex x must also be complex
\end{verbatim}
\end{tscreen}





\subsection{bar\label{bar}}

\begin{tscreen}
\begin{verbatim}
[] = bar(x,y,z...)
 bar(grouplabels,data,barnames[,options]) produces a bar chart.
   Example 1:
   
   bar(strmat("DEC","HP"),#(111,150.6),"Speed")
   
\end{verbatim}

See also: \ref{plot} {(plot)}, \ref{hist} {(hist)}.
\begin{verbatim}
   Error codes:
   1: First argument not int nor real array
   2: First argument not a vector nor matrix
   3: Second argument not int nor real array
   4: Second argument not a vector nor matrix
   5: Dimension mismatch between first and second arg
   6: If second arg is vector, third arg must be a string
   7: Third arg must be a string or a string matrix
   8: If second arg is matrix, third arg must be a string matrix
   9: Dimension mismatch between second and third arg
   10: Syntax error in graph options
   11: First arg must not be real matrix
   12: Could not open temporary MTV file
   
\end{verbatim}
\end{tscreen}





\subsection{bsearch\label{bsearch}}

\begin{tscreen}
\begin{verbatim}
[i] = bsearch(x,X)
 i = bsearch(x,X)
   binary-searches for value X in ordered vector x and returns index i
   such that x[i] <= X <= x[i+1] if x is monotonically increasing,
   or x[i] >= X >= x[i+1] if x is monotonically decreasing.
   The first argument x must be monotonic int or real vector,
   and second argument X can be either int or real scalar or array.
   The result value i is integer-valued, and of the same size as X.
   It satisfies 1<=i<=N-1 where N=length(x) in cases where X is
   in the range min(x[1],x[N])..max(x[1],x[N]), and zero otherwise
   (point out of range).
   The execution time of bsearch is O(log(length(x))*length(X)).
\end{verbatim}

See also: \ref{interp} {(interp)}.
\begin{verbatim}
   Error codes:
   -1: First arg not int or real vector
   -2: Second arg not int or real
\end{verbatim}
\end{tscreen}





\subsection{cd\label{cd}}

\begin{tscreen}
\begin{verbatim}
[] = cd(fn)
 cd("pathname") will change the current directory to "pathname".
   cd("") will change to home directory.
   cd("~/my/dir") will change to directory "my/dir" in the
   home directory.
\end{verbatim}

See also: \ref{getenv} {(getenv)}.
\begin{verbatim}
   Error codes:
   1: Input argument not a string
   2: Directory not found
   3: This system does not support cd 
\end{verbatim}
\end{tscreen}





\subsection{ceil\label{ceil}}

\begin{tscreen}
\begin{verbatim}
[y] = ceil(x)
 ceil(x) returns the smallest integer which is larger than x.
   x must be integer or real scalar or array. If it is an array,
   the operation is applied componentwise.
\end{verbatim}

See also: \ref{floor} {(floor)}, \ref{round} {(round)}, \ref{trunc} {(trunc)}.
\begin{verbatim}
   Error codes:
   -1: Complex or nonnumeric input argument 
\end{verbatim}
\end{tscreen}





\subsection{chol\label{chol}}

\begin{tscreen}
\begin{verbatim}
[B] = chol(A)
 B=chol(A) returns the Cholesky decomposition of a
   positive definite square matrix A: B'**B == A.
   A may be integer, real or complex valued.
   Only the upper half of A is referenced, the lower half
   is assumed to be Hermitian symmetric.
\end{verbatim}

See also: \ref{linsolve} {(linsolve)}, \ref{LU} {(LU)}, \ref{matprod} {(matprod)}, \ref{det} {(det)}, \ref{eig} {(eig)}, \ref{inv} {(inv)}.
\begin{verbatim}
   Error codes:
   -1: Input arg is not a matrix
   -2: Input matrix is not square
   -3: Input matrix is not positive definite 
\end{verbatim}
\end{tscreen}





\subsection{clear\label{clear}}

\begin{tscreen}
\begin{verbatim}
[] = clear(...)
 clear() removes all variables that are visible from
   the whos() function from the workspace. Function definitions
   are not cleared. Variables in loaded packages are also not affected.

   clear("var1","var2",...) removes only the specified variables.
   If a string is not a variable name, no warning is given.
\end{verbatim}

See also: \ref{whos} {(whos)}, \ref{hide} {(hide)}, \ref{unhide} {(unhide)}.
\begin{verbatim}
   Error codes:
   1: Bad argument, not a string
\end{verbatim}
\end{tscreen}





\subsection{closefig\label{closefig}}

\begin{tscreen}
\begin{verbatim}
[] = closefig(n)
 closefig(n) (n>1) closes the nth graphics window.
   The active window is not changed, so if n is the active
   window, subsequent plot commands will reopen it.
   If the window has never been opened or has already been
   closed, closefig is silent about it.

   closefig("all") closes all figures.
\end{verbatim}

See also: \ref{figure} {(figure)}.
\begin{verbatim}
   Error codes:
   1: Argument not an integer
   2: Argument not positive
   3: This Tela installation does not support the closefig command
   4: No n>1 figures are currently open (and X11 initialization has failed)
   
\end{verbatim}
\end{tscreen}





\subsection{colormap\label{colormap}}

\begin{tscreen}
\begin{verbatim}
[] = colormap(r;g,b)
 colormap(r,g,b) sets the colormap (palette) for the next
   opened graphics window. The arguments r,g,b must be real vectors
   with equal lengths and with entries in the range 0..1. If they
   are outside range, they are silently truncated.

   The colormap has an effect only in pcolor plots, and in contour
   and mesh plots if suitable options are used. The colormap is reset
   to its default value after each plot.

   colormap(c), where c is a Nx3 matrix, is also accepted.
   
   NOTICE: you need PlotMTV1.4.2t or later for this feature
   to work. If you try to use it with older PlotMTV's, a warning
   message about unknown option will be displayed.
   
\end{verbatim}

See also: \ref{plot} {(plot)}, \ref{annotate} {(annotate)}.
\begin{verbatim}
   Error codes:
   1: Argument is not real vector
   2: Arguments have unequal lengths
   3: Argument length less than 2 is not allowed
   4: You must give either 1 or 3 input arguments
   5: Single argument is not a Nx3 real matrix
\end{verbatim}
\end{tscreen}





\subsection{computeEPOCH\label{computeEPOCH}}

\begin{tscreen}
\begin{verbatim}
[epoch] = computeEPOCH(year,month,day,hour,minute,second,msec)
 computeEPOCH(year,month,day,hour,minute,second,msec)
   returns the number of milliseconds since 01-Jan-0000 00:00:00.000
   as a real number.
   The arguments must be integer scalars or arrays.
   The arrays must be of the same sizes, but a mixture of arrays and
   scalars is allowed. The result is array if any of the inputs is.
   If an illegal component is detected the result(ing component) will
   be -1.0.
\end{verbatim}

See also: \ref{EPOCHbreakdown} {(EPOCHbreakdown)}, \ref{parseEPOCH} {(parseEPOCH)}, \ref{encodeEPOCH} {(encodeEPOCH)}.
\begin{verbatim}
   Error codes:
   -1: Argument not integer
   -2: Arguments of incompatible dimensions
\end{verbatim}
\end{tscreen}





\subsection{conj\label{conj}}

\begin{tscreen}
\begin{verbatim}
[y] = conj(x)
 y = conj(x) computes the complex conjugate of x.
   Real and integer arguments are returned as such.
   If x is an array, the operation is applied componentwise.
\end{verbatim}

See also: \ref{Re} {(Re)}, \ref{Im} {(Im)}, \ref{arg} {(arg)}.
\end{tscreen}





\subsection{contour\label{contour}}

\begin{tscreen}
\begin{verbatim}
[] = contour(z...)
 contour(z[,options]) plots the matrix z as a filled contour plot.
   contour(z,"xgrid",<x-grid-vector>,"ygrid",<y-grid-vector>[,other opts])
   specifies a nonuniform grid in X and Y.
\end{verbatim}

See also: \ref{contour3} {(contour3)}, \ref{annotate} {(annotate)}, \ref{plot} {(plot)}, \ref{mesh} {(mesh)}, \ref{pcolor} {(pcolor)}, \ref{vplot} {(vplot)}.
\begin{verbatim}
   Error codes:
   1: Could not open temporary MTV file
   2: First argument is not a numeric 2D array
   3: Syntax error in graph options
   4: Write error in MTV file - file system full? 
\end{verbatim}
\end{tscreen}





\subsection{contour3\label{contour3}}

\begin{tscreen}
\begin{verbatim}
[] = contour3(z...)
 contour3(z[,options]) plots the 3D array as a
   "volume" plot. Currently this only means that all
   six faces of the volume are contoured and colored
   according to options.
\end{verbatim}

See also: \ref{annotate} {(annotate)}, \ref{plot} {(plot)}, \ref{mesh} {(mesh)}, \ref{pcolor} {(pcolor)}, \ref{vplot} {(vplot)}.
\begin{verbatim}
   contour.
   Error codes:
   1: Could not open temporary MTV file
   2: First argument is not a real 3D array
   3: Syntax error in graph options
   4: Write error in MTV file - file system full? 
\end{verbatim}
\end{tscreen}





\subsection{cos\label{cos}}

\begin{tscreen}
\begin{verbatim}
[y] = cos(x)
 y = cos(x) computes the cosine function of x.
   If x is complex, the result is complex, otherwise real.
   The argument must be in radians.
   If x is an array, the operation is applied componentwise.
\end{verbatim}
\end{tscreen}





\subsection{cosFFT\label{cosFFT}}

\begin{tscreen}
\begin{verbatim}
[f] = cosFFT(u; dim)
 cosFFT(u) gives the cosine Fast Fourier Transform of array u.
   If u's rank is more than one, the transform is computed
   only along the first dimension (many independent 1D
   transforms).

   cosFFT(u,dim) computes the FFT along the specified dimension.
   The first dimension is labeled 1 and so on.

   For vector u, f=cosFFT(u) is equivalent with

   n = length(u); f = zeros(n);
   for (j=1; j<=n; j++)
       f[j] = u[1] - (-1)^j*u[n] + 2*sum(u[2:n-1]*cos((1:n-2)*(j-1)*pi/(n-1)));

   Note that cosFFT is most efficient when n-1 is a product of small
   primes, where n is the transform length.
           
\end{verbatim}

See also: \ref{invcosFFT} {(invcosFFT)}, \ref{sinFFT} {(sinFFT)}, \ref{cosqFFT} {(cosqFFT)}, \ref{sinqFFT} {(sinqFFT)}, \ref{realFFT} {(realFFT)}, \ref{FFT} {(FFT)}.
\begin{verbatim}
   Error codes:
   -1: First argument not a real array
   -2: Second argument not integer
   -3: Second argument out of range
   
\end{verbatim}
\end{tscreen}





\subsection{cosh\label{cosh}}

\begin{tscreen}
\begin{verbatim}
[y] = cosh(x)
 y = cosh(x) computes the hyperbolic cosine function of x.
   If x is complex, the result is complex, otherwise real.
   If x is an array, the operation is applied componentwise.
\end{verbatim}
\end{tscreen}





\subsection{cosqFFT\label{cosqFFT}}

\begin{tscreen}
\begin{verbatim}
[f] = cosqFFT(u; dim)
 cosqFFT computes the quarter-wave cosine Fourier transform.
   Except for the quarter-wave cosine character, it works similarly to cosFFT.
   
   For vector u, f=cosqFFT(u) is equivalent with

   n = length(u); f = zeros(n);
   for (j=1; j<=n; j++)
       f[j] = u[1] + 2*sum(u[2:n]*cos((2*j-1)*(1:n-1)*pi/(2*n)));
   
   cosqFFT is most efficient when the transform length is a product
   of small primes.
           
\end{verbatim}

See also: \ref{invcosqFFT} {(invcosqFFT)}, \ref{realFFT} {(realFFT)}, \ref{sinqFFT} {(sinqFFT)}, \ref{FFT} {(FFT)}.
\begin{verbatim}
   Error codes:
   -1: First argument not a real array
   -2: Second argument not integer
   -3: Second argument out of range
   
\end{verbatim}
\end{tscreen}





\subsection{cot\label{cot}}

\begin{tscreen}
\begin{verbatim}
[y] = cot(x)
 y = cot(x) is the cotangent function
   cot(x) = 1/tan(x) = cos(x)/sin(x). 
\end{verbatim}
\end{tscreen}





\subsection{cputime\label{cputime}}

\begin{tscreen}
\begin{verbatim}
[t] = cputime(;p)
 cputime() returns the CPU time in seconds used by the current
   tela session. cputime(p) returns CPU time from performance data
   array p, previously obtained from perf().
\end{verbatim}

See also: \ref{tic} {(tic)}, \ref{toc} {(toc)}, \ref{perf} {(perf)}.
\begin{verbatim}
   Error codes:
   -1: Bad argument, must be obtained from perf() 
\end{verbatim}
\end{tscreen}





\subsection{csc\label{csc}}

\begin{tscreen}
\begin{verbatim}
[y] = csc(x)
 y = csc(x) is the cosecant function
   csc(x) = 1/sin(x). 
\end{verbatim}
\end{tscreen}





\subsection{cumprod\label{cumprod}}

\begin{tscreen}
\begin{verbatim}
[y] = cumprod(x)
 y = cumprod(x) forms an array of partial products
   y = #(x[1],x[1]*x[2],...,prod(x[1:n]),...).
   The result y has the same size and type as x. Unlike
   prod, cumprod does not try to avoid integer overflows.
   The array x is logically flattened, to apply
   along a specified dimension use map(cumsum,...).
   If x is scalar, it is returned as such.
\end{verbatim}

See also: \ref{prod} {(prod)}, \ref{cumsum} {(cumsum)}, \ref{map} {(map)}.
\begin{verbatim}
   Error codes:
   -1: Nonnumeric input arg 
\end{verbatim}
\end{tscreen}





\subsection{cumsum\label{cumsum}}

\begin{tscreen}
\begin{verbatim}
[y] = cumsum(x)
 y = cumsum(x) forms an array of partial sums
   y = #(x[1],x[1]+x[2],...,sum(x[1:n]),...).
   The result y has the same size and type as x.
   The array x is logically flattened, to apply
   along a specified dimension use map(cumsum,...).
   If x is scalar, it is returned as such.
\end{verbatim}

See also: \ref{sum} {(sum)}, \ref{cumprod} {(cumprod)}, \ref{map} {(map)}.
\begin{verbatim}
   Error codes:
   -1: Nonnumeric input arg 
\end{verbatim}
\end{tscreen}





\subsection{curve\label{curve}}

\begin{tscreen}
\begin{verbatim}
[] = curve(x,y...)
 curve(x,y) adds a curve in the current SPPC panel.
   SPPC option-value pairs can be added as further arguments.
   Do 'help sppcopts' for list of possible options.
\end{verbatim}

See also: \ref{sppcmark} {(sppcmark)}, \ref{pseudo} {(pseudo)}, \ref{panel} {(panel)}, \ref{endpanels} {(endpanels)}, \ref{sppcopts} {(sppcopts)}, \ref{xlabels} {(xlabels)}.
\begin{verbatim}
   Error codes:
   1: Could not open temporary MTV file
   2: First argument (x) not a real vector
   3: Second argument (y) not a real vector
   4: First two arguments of different lengths
   5: Odd number of arguments
   6: Option name not a string
   7: Unknown option
   8: Bad option value
\end{verbatim}
\end{tscreen}





\subsection{det\label{det}}

\begin{tscreen}
\begin{verbatim}
[d] = det(A)
 det(A) returns the determinant of a square matrix A.
   A may be integer, real or complex valued.
   If A is scalar, it is returned as such.
   Error codes:
   -1: Nonnumeric input arg
   -2: Input array is not a matrix
   -3: Input matrix is not square
   -4: Singular matrix 
\end{verbatim}
\end{tscreen}





\subsection{diag\label{diag}}

\begin{tscreen}
\begin{verbatim}
[y] = diag(x; offset)
 diag(V, K) (V is a vector) returns a square diagonal matrix, with
   vector V on the main diagonal (K == 0, default), the K-th super
   diagonal (K > 0) or the K-th sub-diagonal (K < 0).

   diag(M, K) (M is a matrix) returns the main diagonal (K == 0,
   default), the K-th super diagonal (K > 0), or the K-th sub-diagonal
   (K < 0) of M as a vector.  M need not necessarily be square.

   Error codes:
   -1: Input array X (arg 1) not vector or matrix
   -2: Expecting offset K (arg 2) to be integral
   -3: Internal error
    1: Requested diagonal out of range
\end{verbatim}
\end{tscreen}





\subsection{disasm\label{disasm}}

\begin{tscreen}
\begin{verbatim}
[] = disasm(fn)
 disasm(f) produces disassembly listing of function f.
   Error codes:
   1: Argument not a Tela function 
\end{verbatim}
\end{tscreen}





\subsection{eig\label{eig}}

\begin{tscreen}
\begin{verbatim}
[D;V] = eig(A)
 eig(A) returns the eigenvalues of a square matrix A.
   [D,V] = eig(A) returns the eigenvalues in D and the
   right eigenvectors as columns of V. The eigenvectors
   satisfy A**V == D*V.
\end{verbatim}

See also: \ref{inv} {(inv)}, \ref{LU} {(LU)}.
\begin{verbatim}
   Error codes:
   1: Failed to converge
   -1: Input arg is not an array
   -2: Input array is not a matrix
   -3: Input matrix is not square
   -4: Internal error
   
\end{verbatim}
\end{tscreen}





\subsection{encodeEPOCH\label{encodeEPOCH}}

\begin{tscreen}
\begin{verbatim}
[s] = encodeEPOCH(epoch)
 encodeEPOCH(epoch) returns the character string
   encoding of the given epoch value (number of milliseconds
   since 01-Jan-0000 00:00:00.000) as dd-mmm-yyyy hh:mm:ss:ccc.
\end{verbatim}

See also: \ref{parseEPOCH} {(parseEPOCH)}, \ref{computeEPOCH} {(computeEPOCH)}, \ref{EPOCHbreakdown} {(EPOCHbreakdown)}.
\begin{verbatim}
   Error codes:
   -1: Argument not a real scalar
\end{verbatim}
\end{tscreen}





\subsection{endpanels\label{endpanels}}

\begin{tscreen}
\begin{verbatim}
[] = endpanels(...)
 endpanels() finishes the current SPPC page by drawing
   the pending panels. Do 'help sppcopts' for list of valid options.
   
   Additionally, endpanels recognizes option "output". If set to non-empty string,
   it stores the SPPC code in the named file instead of calling ghostview,
   e.g.: endpanels("output","myfile.sppc").
   You can then do e.g.
     sppc -papertype 8x14cm -landscape <myfile.sppc >myfile.ps
   See sppc -help for possible options. This enables "blind" batch mode runs.
   
\end{verbatim}

See also: \ref{curve} {(curve)}, \ref{sppcmark} {(sppcmark)}, \ref{pseudo} {(pseudo)}, \ref{panel} {(panel)}, \ref{sppcopts} {(sppcopts)}, \ref{xlabels} {(xlabels)}.
\begin{verbatim}
   Error codes:
   1: Calling endpanels before calling curve or pseudo
   5: Odd number of arguments
   6: Option name not a string
   7: Unknown option
   8: Bad option value
\end{verbatim}
\end{tscreen}





\subsection{error\label{error}}

\begin{tscreen}
\begin{verbatim}
[] = error(;msg)
 error("message") writes message and returns to the
   Tela main level.
   error() uses a default message.
\end{verbatim}

See also: \ref{exit} {(exit)}.
\begin{verbatim}
   Error codes:
   -1: Argument not a string
\end{verbatim}
\end{tscreen}





\subsection{eval\label{eval}}

\begin{tscreen}
\begin{verbatim}
[] = eval(s)
 eval("string") executes string as a Tela command, as it had been
   typed from the keyboard.
   The evaluation is done in global context. The symbols appearing
   in the string refer to the global ones.
\end{verbatim}

See also: \ref{evalexpr} {(evalexpr)}.
\begin{verbatim}
   Error codes:
   1: Argument not a string 
\end{verbatim}
\end{tscreen}





\subsection{evalexpr\label{evalexpr}}

\begin{tscreen}
\begin{verbatim}
[y] = evalexpr(s)
 y = evalexpr("expression") executes string as a Tela command,
   returning its value in y.
   The evaluation is done in global context. The symbols appearing
   in the string refer to the global ones.
\end{verbatim}

See also: \ref{eval} {(eval)}.
\begin{verbatim}
   Error codes:
   1: Argument not a string 
\end{verbatim}
\end{tscreen}





\subsection{exit\label{exit}}

\begin{tscreen}
\begin{verbatim}
[] = exit(;rv)
 exit() stops Tela, returning zero (success) exit code
   to the calling process.

   exit(n) returns exit code n instead of zero.
   n must be an int scalar.

   Function quit is a synonym for exit.
\end{verbatim}

See also: \ref{error} {(error)}.
\begin{verbatim}
Error codes:
   1: expecting integer scalar for EXIT_CODE (arg 1).
\end{verbatim}
\end{tscreen}





\subsection{exp\label{exp}}

\begin{tscreen}
\begin{verbatim}
[y] = exp(x)
 y = exp(x) computes the exponent function of x.
   If x is complex, the result is complex, otherwise real.
   If x is an array, the operation is applied componentwise.
\end{verbatim}
\end{tscreen}





\subsection{export\_CDF\label{export_CDF}}

\begin{tscreen}
\begin{verbatim}
[] = export_CDF(fn...)
 export_CDF("fyle"[,flags][,"varables"]...) writes the named global
   variables in CDF file "fyle.cdf". The arguments after the file name
   must be either strings (global variable names) or integer scalars giving
   optional flags.
   export_CDF("fyle"[,flags]) writes all global variables.
   Integer arrays are written using 1-byte or 2-byte signed or unsigned
   storage, if possible.
   Strings and chars are written as CDF_CHAR type if in range -127..127.

   Supported symbolic flags (to combine, use addition):
     CDF_FLOAT    Write reals using 4 bytes
     CDF_DOUBLE   Write reals using 8 bytes (the default)
     CDF_RECORD   Turn first dimension to 'record dimension'
     CDF_NORECORD Do not (the default)
   For example:
     export_CDF("fyle.cdf",CDF_FLOAT+CDF_RECORD,"A",CDF_NORECORD,"B")
   stores global variables A and B using floats, if they are real arrays,
   and turning the first dimension of A into a record dimension.
   Thus the flags are effective for those variables that follow, unless
   turned off by other flags. The flags are not remembered between calls.
\end{verbatim}

See also: \ref{import_CDF} {(import\_CDF)}.
\begin{verbatim}
   Error codes:
   -1: First argument not a string
   -2: Fatal CDF library error
   -5: Rest of the args are not strings or int scalars
\end{verbatim}
\end{tscreen}





\subsection{export\_PNG\label{export_PNG}}

\begin{tscreen}
\begin{verbatim}
[] = export_PNG(fn,r;g,b,a)
 export_PNG("file.png",u)
   writes integer matrix u into grayscale PNG file.

   export_PNG("file.png",u,a) writes grayscale and alpha data.

   export_PNG("file.png",r,g,b) writes RGB colour data.

   export_PNG("file.png",r,g,b,a) writes RGB and alpha data.

   The bit depth will be 8 or 16, depending on whether maxval
   is less than 256 or not. Negative values are silently
   replaced by zero.
   
   Error codes:
   1: First arg not a string
   2: Cannot open output file
   3: Cannot write PNG header to output file
   4: Second input arg not integer matrix
   5: Third input arg not integer matrix
   6: Third input arg is different size matrix than second arg
   7: Fourth input arg not integer matrix
   8: Fourth input arg is different size matrix than second arg
   9: Fifth input arg not integer matrix
   10: Fifth input arg is different size matrix than second arg
\end{verbatim}
\end{tscreen}





\subsection{export\_PNM\label{export_PNM}}

\begin{tscreen}
\begin{verbatim}
[] = export_PNM(fn,r;g,b)
 export_PNM("file.pnm",x) writes integer matrix x in
   PNM (Portable aNyMap) file "file.pnm". If no element is
   larger than 255, a 8-bit "raw" file is written, otherwise an ASCII file.
   If all elements are equal to 0 or 1, a black and white 1-bit PBM
   (Portable BitMap) file will be written.

   export_PNM("file.pnm",r,g,b) creates a 24-bit color-PPM file.
   Matrix entries are truncated in the range 0..255 in this case.
   The integer matrices r,g and b represent the red, green
   and blue values. They must have equal dimensions.

   If the system supports pipes (has the popen function),
   you can use export_PNM("!...",..) to write to a pipe instead
   of a file (for example, export_PNM("!cjpeg >x.jpeg",r,g,b)).
   
\end{verbatim}

See also: \ref{import_PNM} {(import\_PNM)}.
\begin{verbatim}
   Error codes:
   1: First arg not a string
   2: Argument is not integer matrix
   3: export_PNM must be called with 2 or 4 input args
   4: Dimensions of R,G,B matrices must be equal
   5: Could not open file for output
\end{verbatim}
\end{tscreen}





\subsection{export\_RIS8\label{export_RIS8}}

\begin{tscreen}
\begin{verbatim}
[] = export_RIS8(fn,x)
 export_RIS8("file.hdf",x) writes (appends) integer matrix x
   to HDF file "file.hdf" as a raster-8 image (Raster Image Set, RIS).
   When the file is opened with default settings using the usual
   HDF viewers such as Spyglass products, the first dimension grows
   vertically downward and the second dimension grows from left to
   right.

   The RIS8 HDF files can be read back using import.
\end{verbatim}

See also: \ref{export_matlab} {(export\_matlab)}, \ref{save} {(save)}, \ref{import} {(import)}.
\begin{verbatim}
   Error codes:
   1: First argument not a string
   2: Second argument not an integer matrix
   
\end{verbatim}
\end{tscreen}





\subsection{export\_matlab\label{export_matlab}}

\begin{tscreen}
\begin{verbatim}
[] = export_matlab(fn...)
 export_matlab("file") saves all variables in Tela
   workspace in "file". Any previous contents of "file"
   are overwritten. The data are written in MATLAB
   binary format. Hidden Tela variables are not saved.

   export_matlab("file","var1","var2"...) saves only the
   specified variables. Notice that you have to give the
   variable names as strings.

   The resulting MAT-file can be read using the
   MATLAB 'load' command.
   
   Limitations (bugs): It is not possible to export
   local variables. If you try, the global ones will
   be written, if they have numeric values. Use
   export_matlab2 to achieve this, and to have explicit
   control of variable naming.
   
\end{verbatim}

See also: \ref{export_matlab2} {(export\_matlab2)}, \ref{save} {(save)}, \ref{load} {(load)}, \ref{import} {(import)}.
\begin{verbatim}
   Error codes:
   1: Too few arguments
   2: Argument not a string
   3: Write error on file
   
\end{verbatim}
\end{tscreen}





\subsection{export\_matlab2\label{export_matlab2}}

\begin{tscreen}
\begin{verbatim}
[] = export_matlab2(fn...)
 export_matlab2("file", var1,"name1", var2,"name2"...)
   saves objects var1,var2... in MATLAB binary format in "file".
   The objects will be named "name1", "name2"... .
   Any previous contents of "file" are overwritten.
   workspace in "file". Any previous contents of "file"
   is overwritten.

   The resulting MAT-file can be read using the
   MATLAB 'load' command.
   
\end{verbatim}

See also: \ref{export_matlab} {(export\_matlab)}, \ref{save} {(save)}, \ref{load} {(load)}, \ref{import} {(import)}.
\begin{verbatim}
   Error codes:
   1: First arg not a string
   2: Even number of arguments
   3: Write error on file
   4: The 'name' argument is not a string
   
\end{verbatim}
\end{tscreen}





\subsection{export\_netCDF\label{export_netCDF}}

\begin{tscreen}
\begin{verbatim}
[] = export_netCDF(fn...)
 export_netCDF("file",...) behaves exactly like
   save but it exports the variables in netCDF rather
   than HDF format. The function is operational only
   if NewHDFSupported() returns nonzero.
   
\end{verbatim}

See also: \ref{save} {(save)}, \ref{HDFNewMode} {(HDFNewMode)}.
\begin{verbatim}
   Error codes:
   1: Too few arguments
   2: Argument not a string or char
   3: Unexpected HDF error
   4: Cannot open new-style HDF SD file
   5: Cannot create netCDF file
   6: netCDF not supported by this Tela, use HDF4.0b1 or later
\end{verbatim}
\end{tscreen}





\subsection{export\_sound\_CDR\label{export_sound_CDR}}

\begin{tscreen}
\begin{verbatim}
[] = export_sound_CDR(fn,L,R)
 export_sound_CDR("file.cdr",left,right)
   takes integer-valued vectors left and right and writes them in
   raw audio-CD format in "file.cdr". The file will consist of 16-bit
   signed sample pairs (left,right).
   If the filename starts with "!", it is interpreted as an output pipe.
   Error codes:
   1: First arg not a (nonempty) string
   2: Second arg not an int vector
   3: Third arg not an int vector
   4: Second and third args of unequal lengths
   5: Cannot open output file
   6: This Tela implementation has sizeof(short)!=2, cannot do export_sound_CDR, sorry
   7: Not enough memory for work arrays
   8: File write incomplete, maybe filesystem full
\end{verbatim}
\end{tscreen}





\subsection{eye\label{eye}}

\begin{tscreen}
\begin{verbatim}
[A] = eye(n)
 eye(n) returns the (integer) unit matrix of order n.
   n must be a non-negative scalar integer.
   eye(V) where V is a two-element integer vector with
   both elements equal and positive works also, thus
   you can also use eye(size(A)).
\end{verbatim}

See also: \ref{ones} {(ones)}, \ref{inv} {(inv)}.
\begin{verbatim}
   Error codes:
   -1: Argument not an integer or IntArray
   -2: Negative dimension
   -3: IntArray rank not 1
   -4: IntArray length not 2
   -5: Integer vector elements are unequal
   
\end{verbatim}
\end{tscreen}





\subsection{fclose\label{fclose}}

\begin{tscreen}
\begin{verbatim}
[] = fclose(fnum)
 fclose(fnum) closes file with given identification number.
   The fnum must have been previously obtained from fopen.
\end{verbatim}

See also: \ref{fopen} {(fopen)}, \ref{fformat} {(fformat)}, \ref{fread} {(fread)}, \ref{fwrite} {(fwrite)}, \ref{fgetc} {(fgetc)}, \ref{fgetc} {(fgetc)}.
\begin{verbatim}
   Error codes:
   -1: Bad argument: not integer
   -2: Bad argument: outside range
   3: File was not open
\end{verbatim}
\end{tscreen}





\subsection{feof\label{feof}}

\begin{tscreen}
\begin{verbatim}
[result] = feof(fnum)
 feof(fnum) checks whether end of file has been reached
   on previously opened file with identification number fnum.
   Return value is 1 in case of EOF and 0 otherwise.
   Return value is -1 if the file is not open.
\end{verbatim}

See also: \ref{fgetc} {(fgetc)}, \ref{fgets} {(fgets)}, \ref{fopen} {(fopen)}, \ref{fread} {(fread)}, \ref{fwrite} {(fwrite)}.
\begin{verbatim}
   Error codes:
   -1: Bad argument: not integer
   -2: Bad argument: outside range
\end{verbatim}
\end{tscreen}





\subsection{fformat\label{fformat}}

\begin{tscreen}
\begin{verbatim}
[] = fformat(fnum,str...)
 fformat(fnum,"format-string",arg1,arg2,...) is similar to format,
   except that it does not output to stdout but to opened file.
\end{verbatim}

See also: \ref{format} {(format)}, \ref{sformat} {(sformat)}, \ref{fopen} {(fopen)}.
\begin{verbatim}
   Error codes:
   -1: First argument not integer
   -2: First argument not a valid file number
   -3: Second argument not a string or char
   4: File is not open
   
\end{verbatim}
\end{tscreen}





\subsection{fgetc\label{fgetc}}

\begin{tscreen}
\begin{verbatim}
[ch] = fgetc(fnum)
 fgetc(fnum) returns the next character from previously
   opened file with identification number fnum, or VOID
   value if end of file has been reached.
\end{verbatim}

See also: \ref{fgets} {(fgets)}, \ref{fopen} {(fopen)}, \ref{feof} {(feof)}, \ref{fread} {(fread)}, \ref{fwrite} {(fwrite)}.
\begin{verbatim}
   Error codes:
   -1: Bad argument: not integer
   -2: Bad argument: out of range
   -3: File was not open
\end{verbatim}
\end{tscreen}





\subsection{fgets\label{fgets}}

\begin{tscreen}
\begin{verbatim}
[s;endletter] = fgets(fnum;endletters)
 s=fgets(fnum) reads a string from previously opened
   file with identification number fnum. The string
   is terminated with a newline, which is removed from
   the stream but not returned.
   s=fgets(fnum,t) where t is a string uses characters
   in t as terminators, the default for t is "\n".
   [s,t1]=fgets(..) also returns the terminating character
   in t1.
\end{verbatim}

See also: \ref{fgetc} {(fgetc)}, \ref{fopen} {(fopen)}, \ref{feof} {(feof)}, \ref{fread} {(fread)}, \ref{fwrite} {(fwrite)}.
\begin{verbatim}
   Error codes:
   -1: Bad first argument: not integer
   -2: Bad first argument: outside range
   -3: File was not open
   -4: Bad second argument: not a string
   
\end{verbatim}
\end{tscreen}





\subsection{figure\label{figure}}

\begin{tscreen}
\begin{verbatim}
[] = figure(n)
 figure(n) causes subsequent plot commands to use
   window (figure) number n (n=1,2,...). Figures with
   n>1 are drawn over by each new plot command.
   
   figure(-1) restores the default behavior, which is
   to create standalone windows. These plots are never
   replaced by new graphics and are quit only by the user.
   
   Using this command for n>1 requires that you use the
   Tela-compatible PlotMTV version.
   
\end{verbatim}

See also: \ref{closefig} {(closefig)}.
\begin{verbatim}
   Error codes:
   1: Argument is not an integer
   2: Figure number zero is reserved, don't use it
   3: This Tela installation does not support the figure command
   4: Failed X11 initialization, cannot synchronize with PlotMTV
   
\end{verbatim}
\end{tscreen}





\subsection{find\label{find}}

\begin{tscreen}
\begin{verbatim}
[y] = find(x)
 I=find(V) returns the index vector I=(i) for which
   V[i] is nonzero. V must be an integer array. The length
   of I is equal to the number of nonzeros in V.
   If V is multidimensional, it is used in flattened form.

   Example: If V=#(1,-2.3,4,5,-8.2), find(V < 0) returns
   #(2, 5).
   
\end{verbatim}

See also: \ref{any} {(any)}, \ref{all} {(all)}, \ref{flatten} {(flatten)}.
\begin{verbatim}
   Error codes:
   1: Argument not integer vector 
\end{verbatim}
\end{tscreen}





\subsection{flatten\label{flatten}}

\begin{tscreen}
\begin{verbatim}
[y] = flatten(;x)
 flatten(x) returns the array x flattened to a vector.
   If x is not an array, it is returned as is.

   Example: flatten(#(1,2; 3,4)) returns #(1,2,3,4).
   
   x = flatten() flattens x "in place". This is much faster
   (it works in constant time) than to do x = flatten(x),
   since no data movement is involved. 
\end{verbatim}
\end{tscreen}





\subsection{flip\label{flip}}

\begin{tscreen}
\begin{verbatim}
[y] = flip(x;d)
 flip(A) reverses the first dimension of array A.
   flip(A,d) reverses the d'th dimension. For example,
   flip(#(1,6,2; 7,9,2.3),2) returns #(2,6,1; 2.3,9,7).
   If A is non-array it is returned as such, regardless of d.
\end{verbatim}

See also: \ref{transpose} {(transpose)}.
\begin{verbatim}
   Error codes:
   -1: Second argument not an integer scalar
   -2: Second argument out of range 
\end{verbatim}
\end{tscreen}





\subsection{floor\label{floor}}

\begin{tscreen}
\begin{verbatim}
[y] = floor(x)
 floor(x) returns the largest integer which is smaller than x.
   x must be integer or real scalar or array. If it is an array,
   the operation is applied componentwise.
\end{verbatim}

See also: \ref{ceil} {(ceil)}, \ref{round} {(round)}, \ref{trunc} {(trunc)}.
\begin{verbatim}
   Error codes:
   -1: Complex or nonnumeric input argument 
\end{verbatim}
\end{tscreen}





\subsection{fopen\label{fopen}}

\begin{tscreen}
\begin{verbatim}
[fnum] = fopen(name,mode)
 fopen("filename",mode) opens a file and returns
   its identifier (integer). The mode parameter can be
   "r", "w" or "a" for reading, writing and appending,
   respectively. If the open is not succesful, -1 is returned.
\end{verbatim}

See also: \ref{fformat} {(fformat)}, \ref{fclose} {(fclose)}, \ref{fgetc} {(fgetc)}, \ref{fgets} {(fgets)}, \ref{fread} {(fread)}, \ref{fwrite} {(fwrite)}.
\begin{verbatim}
   Error codes:
   -1: First arg not a string
   -2: Second arg not a string
   -3: Too many open files
   -4: Bad string for second arg
\end{verbatim}
\end{tscreen}





\subsection{format\label{format}}

\begin{tscreen}
\begin{verbatim}
[] = format(str...)
 format("format-string",arg1,arg2,...) prints "format-string"
   to standard output, replacing occurrences of `format-spec`
   with consecutive args. `Format-spec` is either empty, i.e. ``,
   or of the form

       `[-]w[.d]`.

   Here w is the field width (unsigned integer) and d is the number
   of significant digits, also unsigned integer. By default the
   argument is printed left-justified, but the optional minus sign
   dictates right justification. The backquote character `  can be
   produced by writing it three times: ```.
   
   Hint: You can add any number of spaces before the closing backquote,
   for example `20.7    `.
   These spaces do not affect the output. This feature can be used
   to justify source code lines.
\end{verbatim}

See also: \ref{fformat} {(fformat)}, \ref{sformat} {(sformat)}.
\begin{verbatim}
   Error codes:
   1: First argument not a string or char 
\end{verbatim}
\end{tscreen}





\subsection{fparse\label{fparse}}

\begin{tscreen}
\begin{verbatim}
[n...] = fparse(fnum,controlstr)
 [n,a,b,c,...] = fparse(fnum,"contolstring") scans input file
   with identification number fnum (obtained with fopen).
   Occurrences of `` in "controlstring" denote objects to be
   read and placed to output variables a,b,c,... in order.
   The number of objects succesfully read added by one is
   placed in n in the normal case.
   The notation `i`, `r`, `z` and `c` can be used to read
   integer, real or complex numbers, or single characters.

   For example,

   fnum = fopen("inputfile","r");
   [n,c,i,z] = fparse(fnum,"N`c` = `i`, a = `z`;");
   if (n != 4) error("...");

   would accept the input

   N3=  35,a   =3.4;

   after which c would be '3', i would be 35 and z would
   be 3.4. White space characters in controlstring are special,
   they match any number (including zero) of whitespace characters
   in input. Other characters in controlstring must appear
   literally in input. The variable n would be assigned the value
   4 in this case. Every time an object is succesfully read,
   the return value is incremented. If the end of the control string
   after the last `` item matches the input, the return value is
   incremented once more. Thus, in the above example, n==1 would
   mean that error occurred between `c` and `i``, n==1 would
   indicate error between `i` and `z` and n==3 would indicate a
   missing semicolon after `z`. Perfectly correct input always
   produced n equal to the total number of output arguments,
   four in the above example.

   Currently there is no way to quote the ` character
   in control string. To read `, read it as `c` and later
   check that it really was a backquote.
   
\end{verbatim}

See also: \ref{fopen} {(fopen)}, \ref{feof} {(feof)}, \ref{fgets} {(fgets)}, \ref{fgetc} {(fgetc)}, \ref{fread} {(fread)}, \ref{fwrite} {(fwrite)}.
\begin{verbatim}
   Error codes:
   -1: First input arg not an integer
   -2: Second input arg not a string
   -3: Bad first argument: outside range
   -4: File is not open
   -5: Unended `` item in control string
   -6: Too few output arguments as compared to control string
   -7: Bad `` item: must be `c`, `i`, `r` or `z`
   -8: `x` item does not end with backquote character
\end{verbatim}
\end{tscreen}





\subsection{fprintf\label{fprintf}}

\begin{tscreen}
\begin{verbatim}
[] = fprintf(fnum,formatstr...)
 fprintf(fnum,"format-string",arg1,arg2,...) is an interface to the C
   fprintf function. The format string should have a percent slot
   for every arg. The args may be integer or real scalars or strings.
   The file identifier fnum must have been obtained from fopen.

   Notice: The stream is not flushed after every fprintf operation,
   but a flush occurs whenever you switch from using fprintf to
   fformat on the same file. Therefore avoid mixing fprintf and fformat
   on the same file if performance is an issue for you!
   
\end{verbatim}

See also: \ref{fopen} {(fopen)}, \ref{printf} {(printf)}, \ref{sprintf} {(sprintf)}, \ref{format} {(format)}.
\begin{verbatim}
   Error codes:
   1: Bad argument type
   2: Second arg not a string
   3: First argument not an integer
   4: Bad file identifier: out of range
   5: File is not open
   6: Internal error: fdopen failed
   
\end{verbatim}
\end{tscreen}





\subsection{fread\label{fread}}

\begin{tscreen}
\begin{verbatim}
[s] = fread(fnum,n)
 fread(fnum,n) reads next n characters (bytes) from
   previously opened file with identification number fnum,
   and returns the result as a string.
   If EOF is reached during read, the read is terminated earlier,
   resulting in length(s) being less than n (possibly zero).
\end{verbatim}

See also: \ref{fwrite} {(fwrite)}, \ref{fgets} {(fgets)}, \ref{fopen} {(fopen)}, \ref{feof} {(feof)}, \ref{fgetc} {(fgetc)}.
\begin{verbatim}
   Error codes:
   -1: Bad first arg, not integer
   -2: Bad first arg, out of range
   -3: Bad second arg, not integer
   -4: File was not open
   -5: Number of bytes to be read is negative
\end{verbatim}
\end{tscreen}





\subsection{fwrite\label{fwrite}}

\begin{tscreen}
\begin{verbatim}
[] = fwrite(fnum,arr)
 fwrite(fnum,arr) writes the elements of integer array arr
   as unsigned bytes in a previously opened file with
   identification number fnum. The elements of arr are clamped
   to the range 0..255 before writing them in the file.
\end{verbatim}

See also: \ref{fread} {(fread)}, \ref{fgets} {(fgets)}, \ref{fopen} {(fopen)}, \ref{feof} {(feof)}, \ref{fgetc} {(fgetc)}.
\begin{verbatim}
   Error codes:
   -1: Bad first arg, not integer
   -2: Bad first arg, out of range
   -3: Bad second arg, not integer array
   -4: File was not open
\end{verbatim}
\end{tscreen}





\subsection{gaussrand\label{gaussrand}}

\begin{tscreen}
\begin{verbatim}
[x] = gaussrand(...)
 gaussrand() returns a Gaussian-distributed random real.
   The distribution function is f(x) = (1/(2*pi))*exp(-0.5*x^2)
   so that the mean is zero and the standard deviation is unity.
   gaussrand(N) returns a vector of length N,
   gaussrand(N,M) returns a matrix of size N x M, etc.
   Example: To test gaussrand, plot it together with the distribution
   function, as follows:

   N=50000; x=-4:0.1:4; y=(N/(2*pi))*exp(-0.5*x^2);
   hold(on); hist(gaussrand(N),20,-4,4); plot(x,y); hold(off);
\end{verbatim}

See also: \ref{rand} {(rand)}, \ref{srand} {(srand)}.
\begin{verbatim}
   Error codes:
   -1: Tried to create too high rank array
   -2: Argument not an integer
   -3: Non-positive integer argument
\end{verbatim}
\end{tscreen}





\subsection{geopack\_coordtrans\label{geopack_coordtrans}}

\begin{tscreen}
\begin{verbatim}
[X,Y,Z] = geopack_coordtrans(fromto,x,y,z,epoch)
 [X,Y,Z] = geopack_coordtrans("fromto",x,y,z,epoch)
   transforms Cartesian coordinates (x,y,z) to another system.
   The first argument specifies the transformation, it must be a 4-6-character
   string containing two coordinate system symbols. Valid coordinate
   systems are GEO,GSM,GSE,GEI,MAG and SM. For example, "GEOGSM","GEISM",
   "SMSM" and "MAGGSE" are valid first arguments where e.g. GEOGSM specifices
   that (x,y,z) is in the GEO system and is to be transformed to the GSM system
   (X,Y,Z). The last argument (epoch) is the number of milliseconds since
   January 1, year 0000, see functions computeEPOCH, EPOCHbreakdown, parseEPOCH
   and encodeEPOCH how to manipulate and construct epoch values.

   The arguments x,y,z,epoch can be scalars or arrays. If they are arrays,
   the must be of the same size. Scalars and arrays can be mixed, scalars
   are implicitly promoted to arrays.

   This function calls the GEOPACK Fortran routines by N. Tsyganenko et al.
   The GEOPACK package operates in single precision.
   Error codes:
   -1: First arg not a string
   -2: Second arg (x) must be int or real scalar or array
   -3: Third arg (y) must be int or real scalar or array
   -4: Fourth arg (z) must be int or real scalar or array
   -5: Fifth arg (epoch) must be int or real scalar or array
   -6: First arg not a string with 4, 5 or 6 characters
   -7: First arg not a valid string (allowed coordsys: GEO,GSM,GSE,GEI,MAG,SM)
   -8: Incompatible array dimensions
 
\end{verbatim}
\end{tscreen}





\subsection{geopack\_model\label{geopack_model}}

\begin{tscreen}
\begin{verbatim}
[Bx,By,Bz] = geopack_model(model,params,x,y,z,epoch; tiltangle)
 [Bx,By,Bz] = geopack_model(model,params,x,y,z,epoch;tiltangle)
   computes a Tsyganenko model magnetic field.
   Inputs:
     model:     string specifying model to use
         params:    parameter vector for the model
         x,y,z:     GSM coordinates in units of R_E (6371.2 km)
         epoch:     Epoch used in IGRF part (and external, if
                      tiltangle is not given)
         tiltangle: (optional) Dipole tilt angle in radians,
                      do not use with IGRF
   Outputs:
     Bx,By,Bz:  GSM Cartesian B-field in nanotesla

   Format of model specification string: either "MODEL"
   or "MODEL,MODEL". Valid MODELs are DIP,IGRF,T89,T96,T96_NOFAC and NONE.
   Empty string is synonym for NONE. The values are case
   insensitive, i.e. both "T96" and "t96" can be used.
   Examples of valid strings: "IGRF", "IGRF,T96", ",DIP",
   "T89,", "None,t89". The T96_NOFAC is T96 from which Birkeland
   currents have been left out; this switch was added by P.Janhunen.

   Parameter vector for T96 and T96_NOFAC is #(Pdyn,Dst,By,Bz)
   where Pdyn is solar wind dynamic pressure (nanopascal),
   Dst the Dst index (nanotesla), By and Bz the IMF GSM
   components (nanotesla).

   Parameter vector for T89 is #(Kp), i.e., the vector contains
   just one element, the Kp index. For DIP (dipole) and IGRF
   models, the parameter vector is ignored (it must still be a
   real or int vector, e.g., zeros(0) suffices).
   
   If tiltangle is given as seventh input argument, it is used
   in the external field model, and epoch only affects the internal
   part. If tiltangle is not given, it is computed automatically
   from the epoch.

   This function calls the GEOPACK Fortran routines by N. Tsyganenko et al.
   The GEOPACK package operates in single precision. Using sophisticated
   methods such as Bulirsch-Stoer or quality-controlled Runge-Kutta for
   field-line tracing using geopack_model may therefore be problematic;
   the function geopack_trace that is provided for this purpose uses
   simpler methods, for this reason. This is a warning to those who would
   like to improve the speed of geopack_trace by implementing a better
   ODE integrator.
   Error codes:
   -1: First arg (model) not a string
   -2: Second arg (params) is not a real vector
   -3: Third arg (x) is not real scalar or array
   -4: Fourth arg (y) is not real scalar or array
   -5: Fifth arg (z) is not real scalar or array
   -6: Sixth arg (epoch) is not real scalar or array
   -7: Seventh arg (tiltangle) not real scalar
   -8: First arg (model) is an invalid string
   -9: Two Tsyganenko models cannot be combined (the other must be IGRF)
   -10: Second arg (params) is of improper length (must be 1 for T89, 4 for T96/T96_NOFAC)
   1: tiltangle setting ignored since IGRF model in use (use DIP instead)
\end{verbatim}
\end{tscreen}





\subsection{geopack\_trace\label{geopack_trace}}

\begin{tscreen}
\begin{verbatim}
[x,y,z; X,Y,Z,success] = geopack_trace(model,params,domain,x0,y0,z0,epoch...)
 [x,y,z; X,Y,Z,success] = geopack_trace(model,params,domain,x0,y0,z0,
                                          epoch[,optional-args])
   traces a field line in a model magnetic field, starting from (x0,y0,z0).
   The model and params arguments are the same as for geopack_model.
   Coordinates are GSM and measured in Earth radius R_E (6371.2 km).
   The domain argument defines the region where the tracing is performed.
   The tracing is stopped when the point moves outside the domain.
   The meaning of the epoch argument is the same as in geopack_model.

   Format of domain string: "r=1.01,r=4.5,z=0", i.e., a comma-separated
   list of equations. Valid left-hand-sides are r,x,y,z (r=sqrt(x^2+y^2+z^2)).
   Coordinates are GSM and units in R_E also here. Any real value is a
   valid inequality right-hand-side. Up to 8 conditions may be used.

   Optional arguments are var,value pairs where var is a string.
   Supported are:
     var         value       meaning
     "sign"      +1 or -1    +1:trace along B, -1:trace antiparallel to B
     "tiltangle" real        dipole tilt angle (radians), if not given,
                               computed from epoch, can be used only if
                               internal field model is DIP or none.
     "maxstep"   real        maximum step size in R_E (default: 0.3)
         "maxsteps"  pos.int     maximum steps to take (default: 2000)

   This function calls the GEOPACK Fortran routines by N. Tsyganenko et al.
   The GEOPACK package operates in single precision. Therefore, this function
   uses a simple and robust midpoint method for integrating the field line.
   Using more sophisticated methods (Bulirsch-Stoer or quality-controlled
   Runge-Kutta) was tried, but did not work well in this case, because of
   lack of precision. This is a warning to those who wouldlike to improve
   the speed of geopack_trace by implementing a better ODE integrator.

   With the default arguments, the error when mapping from x=-13 to
   ionosphere is about 1 km in the ionospheric plane. By decreasing maxstep
   from the default (0.3), the accuracy can be improved.

   The optional output arguments X,Y,Z, if given, become vectors containing
   the full traced trajectory. The seventh output argument (success) is
   returned as 1 unless the number of maxsteps was exceeded, in which case
   it will be returned as 0.
   
   Notice that tracing can fail sometimes in the sense that the point
   starts to circling around in the magnetosphere, never hitting the
   defined boundaries. To avoid such cases from taking too long time
   you can set the parameter "maxsteps" to smaller value than its default
   2000 and quaery the success argument.

   Error codes:
   -1: First arg (model) not a string
   -2: Second arg (params) is not a real vector
   -3: Third arg (domain) is not a string
   -4: Fourth arg (x0) is not int or real scalar
   -5: Fifth arg (y0) is not int or real scalar
   -6: Sixth arg (z0) is not int or real scalar
   -7: Seventh arg (epoch) is not int or real scalar
   -8: Odd number of optional args
   -9: Optional args must be var,value,var,value,... where var is a string
   -10: First arg (model) is an invalid string
   -11: Two Tsyganenko models cannot be combined (the other must be IGRF)
   -12: Second arg (params) is of improper length (must be 1 for T89, 4 for T96/T96_NOFAC)
   -13: Invalid third arg (domain): must be "{R|X|Y|Z}={realvalue},..."
   -14: Invalid optional argument (supported are: sign,tiltangle,maxstep)
   -15: Invalid value for optional argument "sign": must be +1 or -1
   -16: Invalid value for optional argument "tiltangle" (must be int or real scalar)
   -17: Invalid value for optional argument "maxstep" (must be int or real scalar)
   -18: Invalid value for optional argument "maxsteps" (must be positive int)
   1: tiltangle setting ignored since IGRF model in use (use DIP instead)
\end{verbatim}
\end{tscreen}





\subsection{getenv\label{getenv}}

\begin{tscreen}
\begin{verbatim}
[y] = getenv(varname)
 getenv("envvar") returns the value of environment variable
   "envvar", or VOID if such variable is not defined in
   the UNIX environment.

   For example, getenv("LOGNAME") returns the login name of
   the owner of the Tela prcess.
   (If Unix functions are not available, getenv will always
   return VOID.)
\end{verbatim}

See also: \ref{putenv} {(putenv)}, \ref{getpid} {(getpid)}, \ref{cd} {(cd)}.
\begin{verbatim}
   Error codes:
   -1: Argument not a string
   
\end{verbatim}
\end{tscreen}





\subsection{getpid\label{getpid}}

\begin{tscreen}
\begin{verbatim}
[y] = getpid()
 getpid() returns the process ID of the Tela process.
   This can be used e.g. to generate unique temporary
   file names. If Unix functions are not available,
   getpid() returns 0. 
\end{verbatim}
\end{tscreen}





\subsection{grid\label{grid}}

\begin{tscreen}
\begin{verbatim}
[X,Y] = grid(x,y)
 [X,Y] = grid(x,y) produces matrices X,Y that are formed from vectors x,y
   such that X[i,j] = x[i] for all j, and Y[i,j] = y[j] for all i.
\end{verbatim}

See also: \ref{grid3} {(grid3)}.
\begin{verbatim}
   Error codes:
   -1: Input argument is array but not a vector
   -2: Input argument is not real array 
\end{verbatim}
\end{tscreen}





\subsection{grid3\label{grid3}}

\begin{tscreen}
\begin{verbatim}
[X,Y,Z] = grid3(x,y,z)
 [X,Y,Z] = grid3(x,y,z) produces 3D arrays X,Y,Z that ar
   formed from vectors x,y,z such that
   X[i,j,k] = x[i] for all j,k,
   Y[i,j,k] = y[j] for all i,k, and
   Z[i,j,k] = z[k] for all i,j.
\end{verbatim}

See also: \ref{grid} {(grid)}.
\begin{verbatim}
   Error codes:
   -1: Input arg is array but not a vector
   -2: Input arg not a real array 
\end{verbatim}
\end{tscreen}





\subsection{help\label{help}}

\begin{tscreen}
\begin{verbatim}
[] = help(;fn)
 help(function-name) or help("help-item") displays the help information
   associated with a given function or a given help item. On command line
   you may use the abbreviation

      ?help-item
   or
      help help-item

   These forms are translated to help("help-item") before parsing.

   First tries:
   help operators
   help special
   help if
   help for
   help function
   ...
   
   Error codes:
   1: Item not found
   2: Cannot open help file 
\end{verbatim}
\end{tscreen}





\subsection{herm\label{herm}}

\begin{tscreen}
\begin{verbatim}
[B] = herm(A; P)
 herm(A) is the same as conj(transpose(A)).
   herm(A,P) is the same as conj(transpose(A,P)).
   You can abbreviate "herm(A)" as "A'".
\end{verbatim}

See also: \ref{transpose} {(transpose)}, \ref{flip} {(flip)}, \ref{conj} {(conj)}.
\begin{verbatim}
   Error codes:
   -1: Permutation argument not integer array
   -2: Permutation argument of bad rank or size
   -3: Permutation argument contains invalid integers 
\end{verbatim}
\end{tscreen}





\subsection{hide\label{hide}}

\begin{tscreen}
\begin{verbatim}
[] = hide(...)
 hide("sym-name",...) sets the 'hidden' attribute to
   specified symbols.
\end{verbatim}

See also: \ref{whos} {(whos)}, \ref{unhide} {(unhide)}, \ref{autoglobal} {(autoglobal)}.
\begin{verbatim}
   Error codes:
   1: Argument not a string
   2: Argument does not name a symbol
   
\end{verbatim}
\end{tscreen}





\subsection{hist\label{hist}}

\begin{tscreen}
\begin{verbatim}
[;ndata,xdata] = hist(x...)
 hist(x) produces a histogram of vector x.
   The range min(x)..max(x) is divided in bins, and the
   number of x values in each bin is counted. The count determines
   the height of each bin.
   
   hist(x) uses 10 bins, hist(x,n) uses n bins.
   hist(x,n,a) starts from x=a.
   hist(x,n,a,b) ignores x values outside interval a..b.
   Rest of the args may contain other MTV options.
   [ndata,xdata] = hist(args) returns the count and abscissa vectors
   but does not draw anything. bar(xdata,ndata,"lab") can be used to draw
   the histogram later. Graphics options are ignored in this case.
\end{verbatim}

See also: \ref{plot} {(plot)}, \ref{bar} {(bar)}.
\begin{verbatim}
   Error codes:
   1: First argument not a (real) vector
   2: Second arg not a positive integer
   3: Third arg not a (real) scalar
   4: Fourth arg not a (real) scalar
   5: Third arg greater of equal than fourth arg
   6: Could not open temporary MTV file
   7: Syntax error in graph options
   
\end{verbatim}
\end{tscreen}





\subsection{hold\label{hold}}

\begin{tscreen}
\begin{verbatim}
[] = hold(flag)
 hold(on) and hold(off) set the graphics hold mode on and off.
   When hold is on, all graphics commands will be accumulated and
   performed only until hold(off).
   If hold(on) is called many times in succession, also hold(off)
   must be called as many times until the plots are produced.
   For example, if

   function f() {hold(on); plot1(); plot2(); hold(off)};

   and it is called as

   hold(on); f(); plot3(); hold(off);

   then all three plots are actually combined in one plot.

   hold(-1) can be used to reset the internal counter.
   Use it in emergency.
   
\end{verbatim}

See also: \ref{plot} {(plot)}, \ref{holdmode} {(holdmode)}.
\begin{verbatim}
   Error codes:
   1: Argument not an integer 
\end{verbatim}
\end{tscreen}





\subsection{holdmode\label{holdmode}}

\begin{tscreen}
\begin{verbatim}
[result] = holdmode(;flag)
 holdmode(overlay) causes held plots to be combined in same figure.
   holdmode(paging) shows them as sequential pages in one window.
   holdmode(stacking) draws them as subplots in one window.
   holdmode() returns the current setting.
\end{verbatim}

See also: \ref{hold} {(hold)}.
\begin{verbatim}
   Error codes:
   1: Argument not an integer
   2: Bad value for argument 
\end{verbatim}
\end{tscreen}





\subsection{import\label{import}}

\begin{tscreen}
\begin{verbatim}
[] = import(filename; appendix)
 import("file") tries to load the contents of "file" in
   Tela workspace. All files accepted by load are also accepted
   by import. In addition, import accepts more general HDF files
   (SDS and 8-bit raster image files) as well as MATLAB binary
   files (MAT-files).
   
   import("file","app") appends the string "app" to the name
   of every variable imported.
   
   Restrictions:
   1) If file is HDF-file, import first tries to read all SDSs.  Only
   if none was found, it tries to read all RIS8 datasets.  The
   variable names are taken from the HDF labels, if the labels have
   been set. If there are no labels, the variables are named
   "Dataset1", "Dataset2", ... and "Image1", "Image2",... for SDS and
   RIS8 imports, respectively.
   2) Only read Matlab version 4 format (Actually Level 1.0, as
   written by matlab version <= 4 and on request on newer versions).
   
   For filename conventions, see load.
\end{verbatim}

See also: \ref{load} {(load)}, \ref{save} {(save)}, \ref{import1} {(import1)}, \ref{export_matlab} {(export\_matlab)}, \ref{export_RIS8} {(export\_RIS8)}, \ref{import_PNM} {(import\_PNM)}.
\begin{verbatim}
   
   (The difference between import and import1 is that import1 reads
   only one object and returns it, whereas import reads several
   objects and assigns them directly to workspace variables.)
   Error codes:
   1: First arg not string
   2: First arg is not an HDF file
   3: File not found
   4: Cannot import file
   5: Unused error message
   6: Cannot import file, internal Matlab file error
   7: Cannot import this Matlab file, unknown data format
   8: Cannot import this Matlab file, it is a sparse matrix. Make it full before saving.
   10: Bad Matlab binary file, premature end of file
   11: Second argument not a string
   12: Cannot import this Matlab file, it contains VAX or CRAY floating point
   
\end{verbatim}
\end{tscreen}





\subsection{import1\label{import1}}

\begin{tscreen}
\begin{verbatim}
[x] = import1(filename; label)
 import1("file") reads one object from "file". The imported
   object is returned. "File" can be one of the following:

   1) HDF file, in which case the first Scientific Data Set (SDS)
      is imported. The form import1("file.hdf","label") reads SDS
      with label "label", which is not necessarily the first one.

   2) D-style ASCII file of the following format:

        (line 1)   D=Nt dim1 dim2 ... dimN
        (any number M of blank lines or lines starting with '#')
        (line M+2) data1 data2 ....

       where N is the rank of the dataset and t is an optionial
       type specification letter: t may be either 'r', 'i', or 'c'
       for real, integer and complex data, respectively. If t is
       missing, real data are asssumed.

   3) Plain ASCII file of nrows x ncols real numbers. If nrows or
      ncols is 1, it will be returned as a vector, otherwise as a
      matrix. Missing entries are treated as zeros. This format
      is similar to Matlab's load for ASCII, except that '#'
      comments are accepted (for vector files, '#' comments are
          accepted only in the beginning).
       
\end{verbatim}

See also: \ref{import} {(import)}, \ref{load} {(load)}, \ref{save} {(save)}.
\begin{verbatim}
   See import for a difference between import and import1.
   For filename conventions, see load.
   Error codes:
   -1: Input arg not a char or string
   2: File not found
   3: Unknown format in ASCII file
   4: Unknown format in ASCII file
   5: Too high rank ASCII data
   6: Syntax error in ASCII file dimension specification
   7: Syntax error when reading D-style ASCII data
   -8: Internal error
   -9: Cannot import file
   -10: Second arg not a string
   11: Specified label not found
   12: Premature end of file in D-style ASCII
   13: Syntax error when reading plain ASCII data
   14: DFSDgetdims error
   15: DFSDgetdata error
   
\end{verbatim}
\end{tscreen}





\subsection{import\_CDF\label{import_CDF}}

\begin{tscreen}
\begin{verbatim}
[] = import_CDF(fn...)
 import_CDF("fyle") reads the variables in CDF file "fyle.cdf"
   into Tela workspace. Notice that you can leave out the suffix .cdf.

   import_CDF("fyle","var1","var2",...) reads only variables with
   specified names.

   CDF 'record dimensions' are turned into the first dimension of the
   Tela arrays. Likewise, CDF 'number of elements', which is used for
   character strings, is turned into the last dimension. Thus you get
   all the data stored in the file in all cases.
   
   Use import_CDF if you want to load all variables and their names
   suit your purposes and it is OK to load them into global variables.
   import_CDF allows you to control which variables to read, but not
   change their names or load them into local variables.
   See read_CDF if you rather want to load only some variables and
   you want to give them new names or load them into local variables.
   
   Extra services:
   - Uncompresses and reads *.cdf.gz files automatically (using scratch file)
\end{verbatim}

See also: \ref{read_CDF} {(read\_CDF)}, \ref{export_CDF} {(export\_CDF)}.
\begin{verbatim}
   Error codes:
   1: Loading INT2/UINT2 CDF data, but this Tela kernel has sizeof(short int) != 2
   2: Loading INT4/UINT4 CDF data, but this Tela kernel has sizeof(int) != 4
   3: Loading FLOAT CDF data, but this Tela kernel has sizeof(float) != 4
   4: Loading DOUBLE CDF data, but this Tela kernel has sizeof(double) != 8
   5: Empty variable name in CDF file - ignoring
   6: Too high rank in CDF file for this Tela kernel (variable(s) skipped)
   -1: First argument not a string
   -2: Fatal CDF library error
   -3: Unsupported data type
   -5: Function called with more than one argument and the rest are not all strings
   -6: Could not generate temporary file
   -10: Internal error
\end{verbatim}
\end{tscreen}





\subsection{import\_PNG\label{import_PNG}}

\begin{tscreen}
\begin{verbatim}
[r;g,b,a] = import_PNG(fn)
 [r,g,b] = import_PNG(fn)
   reads in a PNG file and places the red, green and blue components
   in matrices r,g and b, respectively. If the file is grayscale,
   r, g and b will have the same value.

   u = import_PNG(fn) reads a grayscale image into u.
   If file has colour information, the average of R, G and B
   components is placed in matrix u.

   [r,g,b,a] = import_PNG(fn) reads also the alpha channel.
   If no alpha channel is stored in the file, a will be set to 255.

   Limitations/bugs: Reading interlaced PNG files is not yet supported.
   
   Error codes:
   1: Cannot open input file
   2: Input file is not PNG file
   3: Bad format inside input PNG file
   -1: Input arg not a string
 
\end{verbatim}
\end{tscreen}





\subsection{import\_PNM\label{import_PNM}}

\begin{tscreen}
\begin{verbatim}
[r;g,b] = import_PNM(fn)
 x = import_PNM("file.pnm") reads PNM (Portable aNyMap) format image file.
   x will become integer matrix.
   [r,g,b] = import_PNM("file.pnm") reads a color image
   and assigns the red, green and blue components to
   matrices r,g,b.
   All six PNM formats (P1-P6) are recognized. If a color
   image (P3 or P6) is loaded using only one output argument,
   the average of color components is computed and assigned
   to x. If a greyscale image is loaded using three output
   args, only the first (r) output arg will be filled,
   g and b are assigned the VOID value. You might use
   a code like

   [r,g,b] = import_PNM("file.pnm");
   if (isvoid(g)) {g=r; b=r};

   to continue processing in 24-bit mode.

   If the system supports pipes (has the popen function),
   you can do import_PNM("!...") to read from a pipe.
   For example, import_PNM("djpeg x.jpg").
   
\end{verbatim}

See also: \ref{import} {(import)}, \ref{export_PNM} {(export\_PNM)}.
\begin{verbatim}
   Error codes:
   -1: First arg not a string
   2: Could not open input file
   3: Input file is not PNM file
   4: Width or height not positive - improper PNM file
   5: Invalid color range
   -6: Two output args, give 1 or 3
   7: Error when reading ASCII data
   8: Pixel not 0 or 1 in 1-bit image
   10: Could not read all binary data
   
\end{verbatim}
\end{tscreen}





\subsection{info\label{info}}

\begin{tscreen}
\begin{verbatim}
[] = info()
 info() shows information about various class sizes for this Tela implementation.
   It also prints the total counts of Tnode, Tprg and Tobject objects at the moment. 
\end{verbatim}
\end{tscreen}





\subsection{input\_string\label{input_string}}

\begin{tscreen}
\begin{verbatim}
[s] = input_string(;prompt)
 input_string() waits for an input line from the keyboard
   and returns it as a string. The newline is not included
   in the result.
   If the string is enclosed in double quotes, they are removed.

   input_string("prompt") displays prompt first.
   Error codes:
   -1: EOF encountered. 
\end{verbatim}
\end{tscreen}





\subsection{intpol\label{intpol}}

\begin{tscreen}
\begin{verbatim}
[y] = intpol(A...)
 intpol(A,index1,index2...) is a general interpolation
   function. A must be an array from which values are interpolated.
   The rank of A must equal the number of index arguments.
   Each index argument may be a real scalar or real array.
   All index arguments must mutually agree in type and rank.
   The array A may also be complex. The result y is of same
   rank and size as each of the index arguments.

   intpol(A,i,j,...) is a generalization of mapped indexing
   A<[i,j,...]> for non-integral indices. The function benefits
   from vectorization even more than most other Tela functions.

   Currently intpol uses linear interpolation.
   Error codes:
   -1: First arg not a numerical array
   -2: Rank of first arg does not match number of index args
   -3: Non-real index arg
   -4: Dissimilar index args
   -6: Range overflow
\end{verbatim}
\end{tscreen}





\subsection{inv\label{inv}}

\begin{tscreen}
\begin{verbatim}
[B] = inv(A)
 inv(A) returns the inverse of a square matrix A.
   A may be integer, real or complex valued.
   A may also be a scalar, in which case its reciprocal
   is returned.
\end{verbatim}

See also: \ref{linsolve} {(linsolve)}, \ref{LU} {(LU)}, \ref{chol} {(chol)}, \ref{matprod} {(matprod)}, \ref{det} {(det)}, \ref{eig} {(eig)}.
\begin{verbatim}
   Error codes:
   -1: Nonnumeric input arg
   -2: Input array is not a matrix
   -3: Input matrix is not square
   -4: Singular matrix
   -5: Singular matrix 
\end{verbatim}
\end{tscreen}





\subsection{invFFT\label{invFFT}}

\begin{tscreen}
\begin{verbatim}
[f] = invFFT(u; dim)
 invFFT() is the inverse of FFT().
   For vector f, u=invFFT(f) is equivalent with

   n = length(f); u = czeros(n);
   for (j=1; j<=n; j++)
       u[j] = (1/n)*sum(f*exp((j-1)*(0:n-1)*2i*pi/n));

   Differences with FFT: sign of i is plus, scale factor 1/n.
   
\end{verbatim}

See also: \ref{FFT} {(FFT)}.
\begin{verbatim}
   Error codes:
   -1: First argument not a numeric array
   -2: Second argument not integer
   -3: Second argument out of range
   
\end{verbatim}
\end{tscreen}





\subsection{invcosFFT\label{invcosFFT}}

\begin{tscreen}
\begin{verbatim}
[f] = invcosFFT(u; dim)
 invcosFFT() is the inverse of cosFFT().
   Actually invcosFFT differs from cosFFT only by normalization,
   but it is provided as a separate function for convenience.
   
   For vector f, u=invcosFFT(f) is equivalent with

   n = length(f); u = zeros(n);
   for (j=1; j<=n; j++)
       u[j] = (f[1] - (-1)^j*f[n] + 2*sum(f[2:n-1]*cos((1:n-2)*(j-1)*pi/(n-1))))/(2*n-2)
   
\end{verbatim}

See also: \ref{cosFFT} {(cosFFT)}.
\begin{verbatim}
   Error codes:
   -1: First argument not a real array
   -2: Second argument not integer
   -3: Second argument out of range
   
\end{verbatim}
\end{tscreen}





\subsection{invcosqFFT\label{invcosqFFT}}

\begin{tscreen}
\begin{verbatim}
[f] = invcosqFFT(u; dim)
 invcosqFFT() is the inverse of cosqFFT()
   (inverse quarter-wave cosine Fourier transform).
   
   For vector f, u=invcosqFFT(f) is equivalent with

   n = length(f); u = zeros(n);
   for (j=1; j<=n; j++)
       u[j] = (1/n)*sum(f*cos((2*(1:n)-1)*(j-1)*pi/(2*n)));
   
\end{verbatim}

See also: \ref{cosqFFT} {(cosqFFT)}.
\begin{verbatim}
   Error codes:
   -1: First argument not a real array
   -2: Second argument not integer
   -3: Second argument out of range
   
\end{verbatim}
\end{tscreen}





\subsection{invrealFFT\label{invrealFFT}}

\begin{tscreen}
\begin{verbatim}
[f] = invrealFFT(u; dim,oddevenspec)
 invrealFFT() is the inverse of realFFT().
   invrealFFT(u,dim,"even") and invrealFFT(u,dim,"odd") specifies
   even or odd transform length, respectively.
   invrealFFT(u,dim,N) uses the same evenness as the integer N has.

   If the evenness is not specified explicitly, the imaginary parts
   of the highest frequency components are tested. If they are all zero
   the transform length is even, otherwise odd. However, this automatic
   method will fail if the imaginary parts are not EXACTLY zero. If you
   use multiple FFTs to solve a PDE, for example, you should probably
   specify the evenness explicitly.
\end{verbatim}

See also: \ref{realFFT} {(realFFT)}.
\begin{verbatim}
   Error codes:
   -1: First argument not a complex array
   -2: Second argument not integer
   -3: Second argument out of range
   -4: Third argument not "even", "odd" or an integer
   
\end{verbatim}
\end{tscreen}





\subsection{invsinFFT\label{invsinFFT}}

\begin{tscreen}
\begin{verbatim}
[f] = invsinFFT(u; dim)
 invsinFFT() is the inverse of sinFFT().
   Actually invsinFFT differs from sinFFT only by normalization,
   but it is provided as a separate function for convenience.
   
   For vector f, u=invsinFFT(f) is equivalent with

   n = length(f); u = zeros(n);
   for (j=1; j<=n; j++)
       u[j] = (1/(n+1))*sum(f*sin((1:n)*j*pi/(n+1)));
   
\end{verbatim}

See also: \ref{sinFFT} {(sinFFT)}.
\begin{verbatim}
   Error codes:
   -1: First argument not a real array
   -2: Second argument not integer
   -3: Second argument out of range
   
\end{verbatim}
\end{tscreen}





\subsection{invsinqFFT\label{invsinqFFT}}

\begin{tscreen}
\begin{verbatim}
[f] = invsinqFFT(u; dim)
 invsinqFFT() is the inverse of sinqFFT()
   (inverse quarter-wave sine Fourier transform).
   
   For vector f, u=invsinqFFT(f) is equivalent with

   n = length(f); u = zeros(n);
   for (j=1; j<=n; j++)
       u[j] = (1/n)*sum(f*sin((2*(1:n)-1)*j*pi/(2*n)));
   
\end{verbatim}

See also: \ref{sinqFFT} {(sinqFFT)}.
\begin{verbatim}
   Error codes:
   -1: First argument not a real array
   -2: Second argument not integer
   -3: Second argument out of range
   
\end{verbatim}
\end{tscreen}





\subsection{isCfunction\label{isCfunction}}

\begin{tscreen}
\begin{verbatim}
[y] = isCfunction(x)
 isCfunction(x) returns 1 if x is a C-tela function.
   and 0 otherwise.
\end{verbatim}

See also: \ref{isfunction} {(isfunction)}, \ref{isTfunction} {(isTfunction)}.
\end{tscreen}





\subsection{isTfunction\label{isTfunction}}

\begin{tscreen}
\begin{verbatim}
[y] = isTfunction(x)
 isTfunction(x) returns 1 if x is a function written in Tela
   and 0 otherwise.
\end{verbatim}

See also: \ref{isfunction} {(isfunction)}, \ref{isCfunction} {(isCfunction)}.
\end{tscreen}





\subsection{isarray\label{isarray}}

\begin{tscreen}
\begin{verbatim}
[y] = isarray(x)
 isarray(x) returns 1 if x is an array and 0 if it is not.
\end{verbatim}

See also: \ref{isscalar} {(isscalar)}, \ref{isvector} {(isvector)}, \ref{ismatrix} {(ismatrix)}.
\end{tscreen}





\subsection{ischar\label{ischar}}

\begin{tscreen}
\begin{verbatim}
[y] = ischar(x)
 ischar(x) returns 1 if x is a character and 0 otherwise.
\end{verbatim}

See also: \ref{isstring} {(isstring)}, \ref{isstr} {(isstr)}.
\end{tscreen}





\subsection{iscomplex\label{iscomplex}}

\begin{tscreen}
\begin{verbatim}
[y] = iscomplex(x)
 iscomplex(x) returns 1 if x is a complex array or scalar,
   and 0 if it is real or integer or a nonnumeric object.
\end{verbatim}

See also: \ref{isreal} {(isreal)}, \ref{isfloat} {(isfloat)}, \ref{isint} {(isint)}.
\end{tscreen}





\subsection{isdefined\label{isdefined}}

\begin{tscreen}
\begin{verbatim}
[y] = isdefined(x)
 isdefined(x) returns 1 if x is not undefined and 0 if
   it is undefined. Optional function arguments are undefined
   if not assigned by the caller; isdefined can be used
   inside the function to test whether this is the case.
\end{verbatim}

See also: \ref{isundefined} {(isundefined)}.
\end{tscreen}





\subsection{isfinite\label{isfinite}}

\begin{tscreen}
\begin{verbatim}
[y] = isfinite(x)
 isfinite(x) returns 1 if x is a finite number and 0 otherwise.
   If x is array, the operation is applied componentwise.
   If x is non-numeric it is considered not finite.
   Integer and consequently strings and chars are always finite. 
\end{verbatim}
\end{tscreen}





\subsection{isfloat\label{isfloat}}

\begin{tscreen}
\begin{verbatim}
[y] = isfloat(x)
 isfloat(x) returns 1 if x is a floating point array
   or scalar, and 0 otherwise. Notice the difference between
   isfloat and isreal. isreal(x) is 1 for integer objects,
   while isfloat(x) is 0.
\end{verbatim}

See also: \ref{isreal} {(isreal)}, \ref{isint} {(isint)}, \ref{iscomplex} {(iscomplex)}.
\end{tscreen}





\subsection{isfunction\label{isfunction}}

\begin{tscreen}
\begin{verbatim}
[y] = isfunction(x)
 isfunction(x) returns 1 if x is a function
   (Tela-function, C-tela function or intrinsic function)
   and 0 otherwise.
\end{verbatim}

See also: \ref{isCfunction} {(isCfunction)}, \ref{isTfunction} {(isTfunction)}.
\end{tscreen}





\subsection{isint\label{isint}}

\begin{tscreen}
\begin{verbatim}
[y] = isint(x)
 isint(x) returns 1 if x is integer scalar or array
   and 0 if it is not.
\end{verbatim}

See also: \ref{isreal} {(isreal)}, \ref{isfloat} {(isfloat)}, \ref{iscomplex} {(iscomplex)}.
\end{tscreen}





\subsection{islist\label{islist}}

\begin{tscreen}
\begin{verbatim}
[y] = islist(x)
 islist(x) returns 1 if x is a heterogenous collection and 0 otherwise. 
\end{verbatim}
\end{tscreen}





\subsection{ismatrix\label{ismatrix}}

\begin{tscreen}
\begin{verbatim}
[y] = ismatrix(x)
 ismatrix(x) returns 1 if x is a matrix (2D array)
   and 0 if it is not.
\end{verbatim}

See also: \ref{isscalar} {(isscalar)}, \ref{isvector} {(isvector)}, \ref{isarray} {(isarray)}.
\end{tscreen}





\subsection{isreal\label{isreal}}

\begin{tscreen}
\begin{verbatim}
[y] = isreal(x)
 isreal(x) returns 1 if x is numerical non-complex
   array or scalar, and 0 otherwise.
\end{verbatim}

See also: \ref{isfloat} {(isfloat)}, \ref{isint} {(isint)}, \ref{iscomplex} {(iscomplex)}.
\end{tscreen}





\subsection{isscalar\label{isscalar}}

\begin{tscreen}
\begin{verbatim}
[y] = isscalar(x)
 isscalar(x) returns 1 if x is scalar and 0 if it is not.
\end{verbatim}

See also: \ref{isvector} {(isvector)}, \ref{ismatrix} {(ismatrix)}, \ref{isarray} {(isarray)}.
\end{tscreen}





\subsection{isstr\label{isstr}}

\begin{tscreen}
\begin{verbatim}
[y] = isstr(x)
 isstr(x) returns 1 if x is a character or string
   and 0 otherwise.
\end{verbatim}

See also: \ref{isstring} {(isstring)}, \ref{ischar} {(ischar)}.
\end{tscreen}





\subsection{isstring\label{isstring}}

\begin{tscreen}
\begin{verbatim}
[y] = isstring(x)
 isstring(x) returns 1 if x is a string and 0 otherwise.
\end{verbatim}

See also: \ref{ischar} {(ischar)}, \ref{isstr} {(isstr)}.
\end{tscreen}





\subsection{isundefined\label{isundefined}}

\begin{tscreen}
\begin{verbatim}
[y] = isundefined(x)
 isundefined(x) returns 1 if x is not undefined and 0 if
   it is undefined. Optional function arguments are undefined
   if not assigned by the caller; isdefined can be used
   inside the function to test whether this is the case.
\end{verbatim}

See also: \ref{isdefined} {(isdefined)}.
\end{tscreen}





\subsection{isvector\label{isvector}}

\begin{tscreen}
\begin{verbatim}
[y] = isvector(x)
 isvector(x) returns 1 if x is a vector and 0 if it is not.
\end{verbatim}

See also: \ref{isscalar} {(isscalar)}, \ref{ismatrix} {(ismatrix)}, \ref{isarray} {(isarray)}.
\end{tscreen}





\subsection{isvoid\label{isvoid}}

\begin{tscreen}
\begin{verbatim}
[y] = isvoid(x)
 isvoid(x) returns 1 if x is a void value and 0 otherwise. 
\end{verbatim}
\end{tscreen}





\subsection{length\label{length}}

\begin{tscreen}
\begin{verbatim}
[L] = length(x)
 length(x) returns the total number of elements in array x.
   If x is scalar, length(x) is 1. If x is undefined, an error
   results. Notice that if x is e.g. a matrix, its length is
   equal to the product of the row and column dimensions,
   which is different from e.g. the Matlab convention.
\end{verbatim}

See also: \ref{size} {(size)}, \ref{rank} {(rank)}.
\begin{verbatim}
   Error codes:
   -1: Argument has undefined value 
\end{verbatim}
\end{tscreen}





\subsection{limit\label{limit}}

\begin{tscreen}
\begin{verbatim}
[y] = limit(x,a,b)
 limit(x,a,b) limits x to the range [a,b].
   x may be an integer or real array or scalar,
   the a and b arguments must be integer or real
   scalars. The result type is integer only
   if all x,a,b are integers, otherwise it is real.
   limit(x,a,b) is equivalent to min(b,max(x,a)).

   Notice: limit is builtin function in >=1.23
   version. Before 1.23 it was a t-file function.

   (Usually a<=b, if a>b then limit(x,a,b) has the
   value of b but this behavior should not be trusted.)
   Error codes:
   -1: First arg not int or real
   -2: Second arg not int or real scalar
   -3: Third arg not int or real scalar
\end{verbatim}
\end{tscreen}





\subsection{link\label{link}}

\begin{tscreen}
\begin{verbatim}
[] = link(filename)
 link("file.o") makes C-tela functions in "file.o" available
   to Tela. "file.o" must be compiled from a C-tela file
   (usually "file.ct").
   Error codes:
   1: Cannot initialize DLD
   2: Cannot link-load file
   3: Argument not string or char
   4: Cannot find fninfo pointer
   5: Internal inconsistency
   6: Undefined symbols remain
   7: main function returned error code
   8: Could not dlclose the previous linkage
   9: Too many dynamically linked modules
   10: File not found
   
\end{verbatim}
\end{tscreen}





\subsection{linsolve\label{linsolve}}

\begin{tscreen}
\begin{verbatim}
[x] = linsolve(A,b)
 linsolve(A,b) solves the linear system A**x == b.
   If A is square, the result x is roughly the same as
   computing inv(A)**b (however, using linsolve is
   faster and numerically more accurate). If A is not
   square, a least-square problem is solved. If the system
   is overdetermined, the solution x minimizes the quantity
   |A**x - b|. If the system is underdetermined, the
   solution x minimizes |x| among all x that satisfy
   A**x==b.
   The second argument may be a vector or a matrix.
   If it is a matrix, several linear systems are effectively
   solved simultaneously.
\end{verbatim}

See also: \ref{inv} {(inv)}, \ref{LU} {(LU)}, \ref{eig} {(eig)}, \ref{SVD} {(SVD)}.
\begin{verbatim}
   Error codes:
   -1: First input arg is not an array
   -2: First input arg is not a matrix
   -3: Second input arg is not an array
   -4: Second input arg is not a vector or matrix
   -5: Incompatible dimensions in first/second args
   -6: Matrix must be square
   -7: Singular matrix
   -8: Internal error
   
\end{verbatim}
\end{tscreen}





\subsection{load\label{load}}

\begin{tscreen}
\begin{verbatim}
[] = load(filename)
 load("file") loads the contents of "file" in Tela workspace.
   "file" must have been previously created using the 'save'
   command; it must be in a certain HDF format.
   
   Filename conventions:
   If the filename starts with "/", "./" or "..",
   it is considered absolute. Otherwise it is searched
   along TELAPATH. This applies to other file
   operations as well.
   
   The counterpart of load is save.
   To read more general HDF files and ASCII files, see import1.
   To load more general HDF files and MATLAB binary files,
   see import.
   
\end{verbatim}

See also: \ref{save} {(save)}, \ref{import} {(import)}, \ref{import1} {(import1)}, \ref{export_matlab} {(export\_matlab)}.
\begin{verbatim}
   Error codes:
   1: Argument not string or char
   2: Argument is not an HDF file
   3: File not found
   4: Argument is not an HDF or netCDF file
   5: Cannot load this HDF file
   6: HDF file load incomplete
   
\end{verbatim}
\end{tscreen}





\subsection{log\label{log}}

\begin{tscreen}
\begin{verbatim}
[y] = log(x)
 y = log(x) computes the natural logarithm of x.
   If x is complex, the result is complex. If x is real or
   integer, but negative, the result is complex (purely
   imaginary). If x is real or integer and non-negative,
   the result is real.
   If x is an array, the operation is applied componentwise.
   If some of the components are negative, all components
   of the result are complex. 
\end{verbatim}
\end{tscreen}





\subsection{logfact\label{logfact}}

\begin{tscreen}
\begin{verbatim}
[y] = logfact(x)
 logfact(x) returns log(x!) for any numerical x, including complex.
   The result for real or int x is usually real, but may be complex
   if x is negative. For arrays, the function is applied componentwise.
   In the array case, the result is complex if any of the imaginary
   parts becomes nonzero, otherwise real.
\end{verbatim}

See also: \ref{fact} {(fact)}, \ref{Gamma} {(Gamma)}, \ref{logGamma} {(logGamma)}.
\begin{verbatim}
   Error codes:
   -1: Argument not numerical
 
\end{verbatim}
\end{tscreen}





\subsection{map\label{map}}

\begin{tscreen}
\begin{verbatim}
[B] = map(fn,A,darg)
 [B] = map(fn,A,d) maps function fn along d'th dimension
   in array A. Fn is a functional argument. It must return a scalar
   or a vector when called with one vector argument. The type and length
   of the returned value must not change from call to call.

   For example, map(mean,A,1) computes the columnwise means
   of matrix A, returning a vector. map(sort,A,2) sorts all rows
   of matrix A in ascending order. Notice than many standard
   functions, including mean, have a builtin mapping capabability;
   using map in these cases is unnecessary.
\end{verbatim}

See also: \ref{mapmin} {(mapmin)}, \ref{mapmax} {(mapmax)}, \ref{flip} {(flip)}.
\begin{verbatim}
   Error codes:
   -1: First arg not a function
   -2: Second arg not a numerical array
   -3: Third arg not an integer scalar
   -4: Third arg (the dimension) out of range
   -5: First arg (function) did not return a numerical object
   -6: First arg (function) unexpectedly changed its return type
   -7: First arg is an intrinsic function; it is no good
   -8: Function returned rank>1 array when first called
\end{verbatim}
\end{tscreen}





\subsection{mapmax\label{mapmax}}

\begin{tscreen}
\begin{verbatim}
[y;P] = mapmax(x;d)
 mapmax(x,d) finds maximum along d'th dimension in array x.
   The result is an array with rank one less than rank(x).
   The array may not be complex.
   [M,p] = mapmax(x,d) returns the maximum positions p along with
   the maximums m. The array p is of the same shape as M, but is
   integer-valued.

   mapmax(x) is a flattened form which returns a scalar result.
   It is equivalent to max(x). [M,p] = mapmax(x) also works.

   Using mapmax is faster than using map and max together.
   In the latter case you would also have to define another function:
       function y=max1(x) {y=max(x)}
   because being intrinsic function, max can not be passed
   to map directly. 
   
\end{verbatim}

See also: \ref{mapmin} {(mapmin)}, \ref{map} {(map)}, \ref{limit} {(limit)}.
\begin{verbatim}
   Error codes:
   -1: First arg not a numerical array
   -2: Second arg not an integer scalar
   -3: Second arg (the dimension) out of range
   -4: First arg is complex 
\end{verbatim}
\end{tscreen}





\subsection{mapmin\label{mapmin}}

\begin{tscreen}
\begin{verbatim}
[y;P] = mapmin(x;d)
 mapmin(x,d) finds minimum along d'th dimension in array x.
   The result is an array with rank one less than rank(x).
   The array x may not be complex.
   [m,p] = mapmin(x,d) returns the minimum positions p along with
   the minimum values m. The array p is of the same shape as m,
   but is integer-valued.

   mapmin(x) is a flattened form which returns a scalar result.
   It is equivalent to min(x). [m,p] = mapmin(x) also works.

   Using mapmin is faster than using map and min together.
   In the latter case you would also have to define another function:
       function y=min1(x) {y=min(x)}
   because being intrinsic function, min can not be passed
   to map directly. 
   
\end{verbatim}

See also: \ref{mapmax} {(mapmax)}, \ref{map} {(map)}, \ref{limit} {(limit)}.
\begin{verbatim}
   Error codes:
   -1: First arg not a numerical array
   -2: Second arg not an integer scalar
   -3: Second arg (the dimension) out of range
   -4: First arg is complex 
\end{verbatim}
\end{tscreen}





\subsection{matprod\label{matprod}}

\begin{tscreen}
\begin{verbatim}
[C] = matprod(A,B; Aflag,Bflag)
 matprod(A,B) returns the matrix product of A and B.
   If at least one of A and B is scalar, matprod(A,B) is the
   same as their ordinary product A*B. If both A and B
   are arrays, their "inner" dimensions must agree.
   That is, the last dimension of A must equal the first
   dimension of B.
   You can abbreviate matprod(A,B) as A**B.

   Optional args: matprod(A,B,aflag,bflag) can be used to
   transpose or Hermitian-conjugate the factors before the
   product. 'n' means no operation, 't' means transpose and
   'h' means Hermitian conjugate. For example,

   matprod(A,B,'h') = A'**B = herm(A)**B
   matprod(A,B,'n','t') = A**B.' = A**transpose(B)

   Normally you need not use matprod explicitly, but you
   can use the operator **, which is internally translated
   to matprod. Hermitian conjugates and transposes in
   connection with ** produce the corresponding 'h' and
   't' options in matprod. For example,

   A'**B        generates       matprod(A,B,'h')
   A.'**B'      generates       matprod(A,B,'t','h')
   A**B.'       generates       matprod(A,B,'n','t')

   and so on. The runtime is optimal for all these operations.
\end{verbatim}

See also: \ref{inv} {(inv)}.
\begin{verbatim}
   
   Error codes:
   -1: Inner dimensions do not agree
   -2: Resulting array would have too high rank
   -3: Third arg not one of 'n', 't', 'h'
   -4: Fourth arg not one of 'n', 't', 'h'
   
\end{verbatim}
\end{tscreen}





\subsection{meminuse\label{meminuse}}

\begin{tscreen}
\begin{verbatim}
[m] = meminuse()
 meminuse() returns the size of the current Tela process
   in bytes. The result is a real number.

   Implementation: It uses the getrusage function if it is available
   and if it seems to be working properly. Otherwise it calls sbrk(0),
   which should return the data segment size in bytes. On Cray UNICOS
   it multiplies the sbrk(0) value by 8 (Cray is a word machine).
   If sbrk is not available, it returns always 0.
   - Surprisingly, many Unix systems have the getrusage function but
   the relevant field in struct rusage is always returned as zero.
   In this case the sbrk method is used.
   - On some machines the result is unreliable.

   If Tela has been compiled with TELASPECIFIC_NEWDELETE defined,
   Tela's own new/delete operators are in use. In this case meminuse
   will report the memory asked from the OS with malloc from C++ code
   only. Any use of malloc from C code (GNU readline library, HDF library ...)
   will thus not be included, as also not the data segment and stack size.
   The result can nevertheless be a useful measure of the user memory.
   You can tell if TELASPECIFIC_NEWDELETE is on by looking at the output
   of the info() function.
\end{verbatim}

See also: \ref{info} {(info)}.
\begin{verbatim}
\end{verbatim}
\end{tscreen}





\subsection{menu\label{menu}}

\begin{tscreen}
\begin{verbatim}
[result] = menu(title...)
 choice = menu("title","choice1","choice2",...) displays
   a menu of choices and returns the number entered by
   the user.
\end{verbatim}

See also: \ref{smenu} {(smenu)}.
\begin{verbatim}
   Error codes:
   -1: Less than two input args 
\end{verbatim}
\end{tscreen}





\subsection{mesh\label{mesh}}

\begin{tscreen}
\begin{verbatim}
[] = mesh(z...)
 mesh(z[,options]) plots the matrix z as a 3D mesh.
   mesh(z,"xgrid",<x-grid-vector>,"ygrid",<y-grid-vector>[,other opts])
   specifies a nonuniform grid in X and Y.
\end{verbatim}

See also: \ref{annotate} {(annotate)}, \ref{plot} {(plot)}, \ref{contour} {(contour)}, \ref{pcolor} {(pcolor)}, \ref{vplot} {(vplot)}.
\begin{verbatim}
   Error codes:
   1: Could not open temporary MTV file
   2: First argument is not a numeric 2D array
   3: Syntax error in graph options
   4: Write error in MTV file - file system full? 
\end{verbatim}
\end{tscreen}





\subsection{ones\label{ones}}

\begin{tscreen}
\begin{verbatim}
[y] = ones(...)
 ones(n,m...) returns an integer array with all elements
   equal to 1 of size n x m x ... .

   ones(V) where V is an integer vector, and thus
   ones(size(A)), works also.
\end{verbatim}

See also: \ref{rand} {(rand)}, \ref{eye} {(eye)}.
\begin{verbatim}
   
   Error codes:
   -1: Input argument not an integer or IntVector
   -2: Rank of requested tensor array exceeds MAXRANK
   -3: Non-positive input argument
   -4: Negative input argument
   -5: Integer array rank not 1
   
\end{verbatim}
\end{tscreen}





\subsection{panel\label{panel}}

\begin{tscreen}
\begin{verbatim}
[] = panel(...)
 panel() finishes the current SPPC panel by drawing the pending
   curve and pseudo objects.
   Do 'help sppcopts' for list of valid options.
\end{verbatim}

See also: \ref{curve} {(curve)}, \ref{sppcmark} {(sppcmark)}, \ref{pseudo} {(pseudo)}, \ref{endpanels} {(endpanels)}, \ref{sppcopts} {(sppcopts)}, \ref{xlabels} {(xlabels)}.
\begin{verbatim}
   Error codes:
   1: Calling panel before calling curve or pseudo
   5: Odd number of arguments
   6: Option name not a string
   7: Unknown option
   8: Bad option value
\end{verbatim}
\end{tscreen}





\subsection{parseEPOCH\label{parseEPOCH}}

\begin{tscreen}
\begin{verbatim}
[epoch] = parseEPOCH(s)
 parseEPOCH(epochstring) parses an epoch string
   and returns the epoch value (number of milliseconds since
   01-Jan-0000 00:00:00.000). The epoch string must be of the
   form dd-mmm-yyyy hh:mm:ss.ccc where
     dd   is the day of month (1-31)
         mmm  is the month (Jan,Feb,Mar,Apr,May,Jun,Jul,Aug,Sep,Oct,Nov,Dec)
         yyyy is the year
         hh   is the hour (0-23)
     mm   is the minute (0-59)
         sec  is the second (0-59)
         ccc  is the millisecond (0-999)
   If an invalid string is given, -1.0 is returned.
\end{verbatim}

See also: \ref{computeEPOCH} {(computeEPOCH)}, \ref{EPOCHbreakdown} {(EPOCHbreakdown)}, \ref{encodeEPOCH} {(encodeEPOCH)}.
\begin{verbatim}
   Error codes:
   -1: Argument not a string
\end{verbatim}
\end{tscreen}





\subsection{pause\label{pause}}

\begin{tscreen}
\begin{verbatim}
[] = pause(;seconds)
 pause() will wait for a keypress on keyboard.
   pause(n) will pause for n seconds and then continue.
   The argument n may be integer or real.

   Note: some systems implicitly round a real argument
   to nearest whole number.
   Error codes:
   1: Argument not a real scalar
   2: Argument is negative
   3: This system does not support pausing for n seconds 
\end{verbatim}
\end{tscreen}





\subsection{pcolor\label{pcolor}}

\begin{tscreen}
\begin{verbatim}
[] = pcolor(z...)
 pcolor(z[,options]) plots the matrix z as a pseudocolor density plot.
   pcolor(z,"xgrid",<x-grid-vector>,"ygrid",<y-grid-vector>[,other opts])
   specifies a nonuniform grid in X and Y.
\end{verbatim}

See also: \ref{annotate} {(annotate)}, \ref{plot} {(plot)}, \ref{contour} {(contour)}, \ref{mesh} {(mesh)}, \ref{vplot} {(vplot)}.
\begin{verbatim}
   Error codes:
   1: Could not open temporary MTV file
   2: First argument is not a numeric 2D array
   3: Syntax error in graph options
   4: Write error in MTV file - file system full? 
\end{verbatim}
\end{tscreen}





\subsection{perf\label{perf}}

\begin{tscreen}
\begin{verbatim}
[v] = perf()
 perf() returns an array containing all maintained operation
   counters. To measure performance of a code segment, do

   p0=perf(); mycode(); p=perf()-p0;

   Now you can apply various performance-related functions to p:
   for example cputime(p), Mflops(p). 
\end{verbatim}
\end{tscreen}





\subsection{pixmap\label{pixmap}}

\begin{tscreen}
\begin{verbatim}
[] = pixmap(flag)
 pixmap(off) tells PlotMTV not to use pixmaps for faster redraw.
   pixmap(on) turns the pixmap mode on, which is the default.
   
   If your X server uses backing store, you can save memory
   by turning pixmap(off) without hurting performance.
   Error codes:
   1: Argument is not an integer 
\end{verbatim}
\end{tscreen}





\subsection{plot\label{plot}}

\begin{tscreen}
\begin{verbatim}
[] = plot(...)
 plot(x1,y1,[options1], x2,y2,[options2],...) is the basic 2D plot function.
   Each vector yi is plotted versus the corresponding xi. All curves yi are
   displayed in the same figure. The option sequences must consist of keyword-
   value pairs. Example:

       x = 0:0.1:4*pi;
       plot(x,sin(x), "linewidth",3,"linecolor",2);

   The abscissa x may be missing, in which case the default of 1:length(y)
   is used. The ordinates y may be matrices; then each row produces one
   curve. If also abscissa x is matrix, the x-value may be different for each
   curve.
\end{verbatim}

See also: \ref{plot3} {(plot3)}, \ref{annotate} {(annotate)}, \ref{plotopt} {(plotopt)}, \ref{mesh} {(mesh)}, \ref{contour} {(contour)}, \ref{pcolor} {(pcolor)}, \ref{vplot} {(vplot)}, \ref{colormap} {(colormap)}.
\begin{verbatim}
   Error codes:
   1: Could not open temporary MTV file
   2: Nonnumeric or complex data argument
   3: Syntax error in graph options
   4: The abscissa ("x") must be a vector or a matrix
   5: The ordinate ("y") must be a vector or a matrix
   6: x and y dimensions disagree
   
\end{verbatim}
\end{tscreen}





\subsection{plot3\label{plot3}}

\begin{tscreen}
\begin{verbatim}
[] = plot3(x,y,z...)
 plot3(x,y,z[,options]) produces parametric space curves.
   The quantities x,y,z must have equal ranks, and they can
   be either vectors or matrices. If they are vectors, only
   one space curve is drawn. If they are matrices, the number
   of curves produces equals the number of rows.
\end{verbatim}

See also: \ref{plot} {(plot)}, \ref{annotate} {(annotate)}.
\begin{verbatim}
   Error codes:
   1: Could not open temporary MTV file
   2: y dimensionality disagrees with x dimensionality
   3: z dimensionality disagrees with x dimensionality
   4: Input arrays must be integer or real arrays
   5: Input arrays must have rank equal to 1 or 2
   6: Syntax error in graph options
   
\end{verbatim}
\end{tscreen}





\subsection{plotopt\label{plotopt}}

\begin{tscreen}
\begin{verbatim}
[] = plotopt(s)
 plotopt("-3d -colorps -landscape...") sets a set of PlotMTV command
   line options for subsequent graphics commands (global setting).
\end{verbatim}

See also: \ref{plot} {(plot)}, \ref{annotate} {(annotate)}.
\begin{verbatim}

   NOTICE: plotopt is usually not required. You can pass the option string
   to all plot commands directly, for example:

   plot(x,sin(x),"-3d -landscape");

   These options affect only the current (or next outputted, if hold is used)
   plot. All graphics function optional string args which start with minus sign
   are assumed to be PlotMTV command line options.
   
   Error codes:
   1: Argument not a string 
\end{verbatim}
\end{tscreen}





\subsection{poissonrand\label{poissonrand}}

\begin{tscreen}
\begin{verbatim}
[y] = poissonrand(x)
 poissonrand(x) returns a Poisson-distributed random integer
   whose expectation value is x (x >= 0). The probability that
   the result is n (n>=0) is x^n*exp(-x)/n!.
   If (x < 0) the result is always 0.
   If x is an array, the result is the same shape as x.
   The result is always integer-valued.
\end{verbatim}

See also: \ref{rand} {(rand)}, \ref{srand} {(srand)}, \ref{gaussrand} {(gaussrand)}.
\begin{verbatim}
   Error codes:
   -1: Argument not real scalar or array
\end{verbatim}
\end{tscreen}





\subsection{printf\label{printf}}

\begin{tscreen}
\begin{verbatim}
[] = printf(formatstr...)
 printf("format-string",arg1,arg2,...) is an interface to the C
   printf function. The format string should have a percent slot
   for every arg. The args may be integer or real scalars or strings.
\end{verbatim}

See also: \ref{fprintf} {(fprintf)}, \ref{sprintf} {(sprintf)}, \ref{format} {(format)}.
\begin{verbatim}
   Error codes:
   1: Bad argument type
   2: First arg not a string
   
\end{verbatim}
\end{tscreen}





\subsection{prod\label{prod}}

\begin{tscreen}
\begin{verbatim}
[y] = prod(x;d)
 prod(x) multiplies all the elements of x, if x is an array.
   prod(x,d) takes the product along d'th dimension only, returning
   an array of rank one less than rank(x).
   If x is scalar, it is returned as such.
   If I is integer array, prod(I) will be of type real if the product
   would cause integer overflow. In all other cases, including prod(I,d),
   the type of y equals the component type of x.
\end{verbatim}

See also: \ref{sum} {(sum)}, \ref{cumprod} {(cumprod)}.
\begin{verbatim}
   Error codes:
   -1: Nonnumeric input arg
   -2: Second argument not an integer
   -3: Second argument (dimension spec) out of range 
\end{verbatim}
\end{tscreen}





\subsection{pseudo\label{pseudo}}

\begin{tscreen}
\begin{verbatim}
[] = pseudo(x,y,z...)
 pseudo(x,y,z) plots a pseudocolor plot in the current SPPC panel.
   SPPC option-value pairs can be added as further arguments,
   for example, pseudo(x,y,z,"interp","true) uses smooth rendering.
   Do 'help sppcopts' for list of valid options.
\end{verbatim}

See also: \ref{curve} {(curve)}, \ref{sppcmark} {(sppcmark)}, \ref{panel} {(panel)}, \ref{endpanels} {(endpanels)}, \ref{sppcopts} {(sppcopts)}, \ref{xlabels} {(xlabels)}.
\begin{verbatim}
   Error codes:
   1: Could not open temporary MTV file
   2: First argument (x) not a real vector
   3: Second argument (y) not a real vector
   4: Third argument (z) not a real matrix
   5: Odd number of arguments
   6: Option name not a string
   7: Unknown option
   8: Bad option value
   9: Third argument (z) dimensions do not match x,y lengths
\end{verbatim}
\end{tscreen}





\subsection{putenv\label{putenv}}

\begin{tscreen}
\begin{verbatim}
[] = putenv(varname,value)
 putenv("varname","value") sets an environment variable.
   The environment variables are inherited to Unix system programs
   invoked with system() and run(), as well as new PlotMTV windows.
   For example:
   putenv("MTV_PSFONT1","32");  // set large font
   plot(x,y,"toplabel","Large");
\end{verbatim}

See also: \ref{getenv} {(getenv)}.
\begin{verbatim}
   Error codes:
   -1: First argument not a string
   -2: Second argument not a string
   -3: First argument an empty string
   -4: Environment variable name may not contain the '=' character
   -5: Environment variable value may not contain the '=' character
   1: System call failed to set environment variable
\end{verbatim}
\end{tscreen}





\subsection{quit\label{quit}}

\begin{tscreen}
\begin{verbatim}
[] = quit(;rv)
 quit() stops Tela, returning zero (success) exit code
   to the calling process.

   quit(n) returns exit code n instead of zero.
   n must be an int scalar.

   Function quit is a synonym for exit.
\end{verbatim}

See also: \ref{error} {(error)}.
\begin{verbatim}
Error codes:
    1: expecting integer scalar for EXIT_CODE (arg 1).
\end{verbatim}
\end{tscreen}





\subsection{rand\label{rand}}

\begin{tscreen}
\begin{verbatim}
[x] = rand(...)
 rand() returns a uniformly distributed random real x, 0<=x<1.
   rand(N) (N positive integer) returns a real random vector of length N.
   rand(N,M) returns a random matrix, and so on.
\end{verbatim}

See also: \ref{srand} {(srand)}, \ref{gaussrand} {(gaussrand)}.
\begin{verbatim}
   Error codes:
   -1: Tried to create too high rank array
   -2: Argument not an integer
   -3: Non-positive integer argument 
\end{verbatim}
\end{tscreen}





\subsection{rank\label{rank}}

\begin{tscreen}
\begin{verbatim}
[y] = rank(x)
 rank(A) returns the number of dimensions of array A.
   The rank of a scalar is 0. The rank of a nonnumeric
   object, including undefined value, is -1. The rank function
   never generates an error.
\end{verbatim}

See also: \ref{length} {(length)}, \ref{size} {(size)}.
\end{tscreen}





\subsection{read\_CDF\label{read_CDF}}

\begin{tscreen}
\begin{verbatim}
[...] = read_CDF(fn...)
 read_CDF("fyle","varname") reads the variable "varname"
   in CDF file "fyle.cdf" and returns its value.
   You can leave out the suffix .cdf from filename.
   It can also read *.cdf.gz files, uncompressing them
   automatically.

   [out1,out2,...] = read_CDF("fyle","name1","name2",...)
   reads several variables and assigns them into the
   output parameters in order. There must be at least as
   many names as there are outputs; if there are more,
   a warning message is given.

   CDF 'record dimensions' are turned into the first dimension of the
   Tela arrays. Likewise, CDF 'number of elements', which is used for
   character strings, is turned into the last dimension. Thus you get
   all the data stored in the variables in all cases.

   Use read_CDF if you want to load only some variables and/or want
   to give them new names or load them into local variables.
   If it is OK to read the variables with their default names into
   global variables, it is easier to use import_CDF.
   
\end{verbatim}

See also: \ref{import_CDF} {(import\_CDF)}, \ref{export_CDF} {(export\_CDF)}.
\begin{verbatim}
   Error codes:
   1: Loading INT2/UINT2 CDF data, but this Tela kernel has sizeof(short int) != 2
   2: Loading INT4/UINT4 CDF data, but this Tela kernel has sizeof(int) != 4
   3: Loading FLOAT CDF data, but this Tela kernel has sizeof(float) != 4
   4: Loading DOUBLE CDF data, but this Tela kernel has sizeof(double) != 8
   5: Empty variable name in CDF file - ignoring
   6: Too high rank in CDF file for this Tela kernel (variable(s) skipped)
   10: More named variables than output args
   -1: First argument not a string
   -2: Fatal CDF library error
   -3: Unsupported data type
   -5: Function called with more than one argument and the rest are not all strings
   -6: Could not generate temporary file
   -10: Internal error
   -11: More output args than named variables
\end{verbatim}
\end{tscreen}





\subsection{realFFT\label{realFFT}}

\begin{tscreen}
\begin{verbatim}
[f] = realFFT(u; dim)
 realFFT(u) gives the Fast Fourier Transform of real array u.
   If u's rank is more than one, the transform is computed
   only along the first dimension (many independent 1D
   transforms).

   realFFT(u,dim) computes the FFT along the specified dimension.
   The first dimension is labeled 1 and so on.

   The result of realFFT() is the same as FFT() except that only
   nonnegative frequency components are returned. The result is
   always complex array. The first component (0 frequency) has always
   zero imaginary part. If the transform length is even, the last
   component has zero imaginary part as well. Notice that these
   conventions are different from some generally used C and Fortran
   library routines, which return a real array force-fitted
   in the same space as the input array by not storing the zero
   imaginary parts. The Tela convention allows you to manipulate the
   result in k-space more easily because it is already complex.

   realFFT is the most efficient when the transform length is
   a product of small primes.
   
\end{verbatim}

See also: \ref{invrealFFT} {(invrealFFT)}, \ref{FFT} {(FFT)}, \ref{sinFFT} {(sinFFT)}, \ref{cosFFT} {(cosFFT)}, \ref{sinqFFT} {(sinqFFT)}, \ref{cosqFFT} {(cosqFFT)}, \ref{Lomb} {(Lomb)}.
\begin{verbatim}
   Error codes:
   -1: First argument not a real array
   -2: Second argument not integer
   -3: Second argument out of range
   
\end{verbatim}
\end{tscreen}





\subsection{remove\label{remove}}

\begin{tscreen}
\begin{verbatim}
[] = remove(fn)
 remove("file") removes the named file.
   If the file does not exist or some other error occurs,
   no warning or error message is given.
   Error codes:
   1: Argument not a string
   
\end{verbatim}
\end{tscreen}





\subsection{reset\_alt\_display\label{reset_alt_display}}

\begin{tscreen}
\begin{verbatim}
[] = reset_alt_display()
 reset_alt_display() must be used after set_alt_display
   to return to normal settings (to direct X11 commands to
   DISPLAY environment variable again).
\end{verbatim}

See also: \ref{set_alt_display} {(set\_alt\_display)}.
\begin{verbatim}
\end{verbatim}
\end{tscreen}





\subsection{reshape\label{reshape}}

\begin{tscreen}
\begin{verbatim}
[B] = reshape(A...)
 reshape(A,n,m,...) returns the data in array A rearranged
   to have dimensionality n x m x ... . The product of the indices
   must equal the length of A.
   reshape(A,#(n,m...)) works also.

   Example: reshape(#(1,2,3,4,5,6), 2,3) returns
       #(1, 2, 3;
             4, 5, 6)  
   Error codes:
   -1: First argument not an array
   -2: Later argument not an integer
   -3: Product of dimensions does not equal the length of first argument
   -4: Number of input arguments exceeds MAXRANK
   -5: Second arg is array but not integer vector
   
\end{verbatim}
\end{tscreen}





\subsection{round\label{round}}

\begin{tscreen}
\begin{verbatim}
[y] = round(x)
 round(x) returns the nearest integer.
   x must be integer or real scalar or array. If it is an array,
   the operation is applied componentwise.
\end{verbatim}

See also: \ref{floor} {(floor)}, \ref{ceil} {(ceil)}, \ref{trunc} {(trunc)}.
\begin{verbatim}
   Error codes:
   -1: Complex or nonnumeric input argument 
\end{verbatim}
\end{tscreen}





\subsection{run\label{run}}

\begin{tscreen}
\begin{verbatim}
[output] = run(cmd; input)
 run("cmd","input") runs operating system (Unix) command
   "cmd", using contents of the second argument string as
   standard input. It returns the standard output of "cmd"
   as a string.

   The form run("cmd") may be used if the command does not
   read standard input. The command is executed by /bin/sh.
\end{verbatim}

See also: \ref{system} {(system)}.
\begin{verbatim}

   Error codes:
   -1: First argument not a string
   -2: Second argument not a string
   -3: Error with temporary file
   -4: Error with internal pipe
   -5: run not supported: OS does not provide unistd.h nor popen()
   7: wait(2) returned error
   -8: pipe(2) returned error
   -9: cannot fork(2) a child process
   
\end{verbatim}
\end{tscreen}





\subsection{save\label{save}}

\begin{tscreen}
\begin{verbatim}
[] = save(fn...)
 save("file") saves all variables in Tela workspace
   in "file". Any previous contents of "file" is
   overwritten. The data are written as Scientific
   Data Sets in HDF format. Hidden variables are not saved.
   
   save("file","var1","var2"...) saves only the
   specified variables. Notice that you have to give
   the variable names as strings.
   
   Limitations (bugs): It is not possible to save
   local variables, since they are not bound to
   symbols. If you try, the global one, if any,
   will be saved.
   
\end{verbatim}

See also: \ref{HDFNewMode} {(HDFNewMode)}, \ref{load} {(load)}, \ref{export_matlab} {(export\_matlab)}, \ref{export_netCDF} {(export\_netCDF)}.
\begin{verbatim}
   Error codes:
   1: Too few arguments
   2: Argument not a string or char
   3: Unexpected HDF error
   4: Cannot open new-style HDF SD file
\end{verbatim}
\end{tscreen}





\subsection{sec\label{sec}}

\begin{tscreen}
\begin{verbatim}
[y] = sec(x)
 y = sec(x) is the secant function
   sec(x) = 1/cos(x). 
\end{verbatim}
\end{tscreen}





\subsection{set\_alt\_display\label{set_alt_display}}

\begin{tscreen}
\begin{verbatim}
[] = set_alt_display(s)
 set_alt_display("X display name") informs Tela that
   subsequent PlotMTV commands use the given X display name
   instead of the default (DISPLAY environment variable).
   If you want to temporarily redirect your display to somewhere
   else, be it mydisplay:1 do the following:

   closefig(1); ...     // close all figures that you will manipulate below
   set_alt_display("mydisplay:1");
   plotopt("-display mydisplay:1 ...");
   //  Do your graphics commands here ...
   reset_alt_display(); // return to normal setting
   plotopt("...");      // i.e  without -display argument

   You MUST ensure that the display given to PlotMTV via
   p|otopt (or directly, using a plot command option) is the same
   as that used in set_alt_display.
\end{verbatim}

See also: \ref{plot} {(plot)}, \ref{reset_alt_display} {(reset\_alt\_display)}.
\begin{verbatim}
   Error codes:
   -1: Argument not a string
\end{verbatim}
\end{tscreen}





\subsection{sformat\label{sformat}}

\begin{tscreen}
\begin{verbatim}
[s] = sformat(formatstr...)
 sformat("format-string",arg1,arg2,...) is similar to format,
   except that it does not output to stdout but returns a string
   variable.
\end{verbatim}

See also: \ref{format} {(format)}, \ref{fformat} {(fformat)}, \ref{sprintf} {(sprintf)}.
\begin{verbatim}
   Error codes:
   -1: First argument not a string or char 
\end{verbatim}
\end{tscreen}





\subsection{showcompiled\label{showcompiled}}

\begin{tscreen}
\begin{verbatim}
[] = showcompiled(filename...)
 showcompiled("filename.ct",f1,f2,...) compiles functions
   f1,f2,... to C-tela code, creating "filename.ct".
   If no suffix is given in "filename", the suffix
   ".ct" will be assumed.
   showcompiled(f1,f2,...) displays on standard output.
\end{verbatim}

See also: \ref{t2ct} {(t2ct)}.
\begin{verbatim}
   Error codes:
   1: One of the args is not a Tela-function
   2: Cannot open output file
   
\end{verbatim}
\end{tscreen}





\subsection{sign\label{sign}}

\begin{tscreen}
\begin{verbatim}
[y] = sign(x)
 y = sign(x) returns 1 if x>0, 0 if x==0, and -1 if x<0.
   x must be real. If x is array, the operation is applied componentwise.
\end{verbatim}

See also: \ref{HeavisideTheta} {(HeavisideTheta)}.
\begin{verbatim}
   Error codes:
   -1: Complex or nonnumeric input argument 
\end{verbatim}
\end{tscreen}





\subsection{sin\label{sin}}

\begin{tscreen}
\begin{verbatim}
[y] = sin(x)
 y = sin(x) computes the sine function of x.
   If x is complex, the result is complex, otherwise real.
   The argument must be in radians.
   If x is an array, the operation is applied componentwise.
\end{verbatim}
\end{tscreen}





\subsection{sinFFT\label{sinFFT}}

\begin{tscreen}
\begin{verbatim}
[f] = sinFFT(u; dim)
 sinFFT(u) gives the sine Fast Fourier Transform of array u.
   If u's rank is more than one, the transform is computed
   only along the first dimension (many independent 1D
   transforms).

   sinFFT(u,dim) computes the FFT along the specified dimension.
   The first dimension is labeled 1 and so on.

   For vector u, f=sinFFT(u) is equivalent with

   n = length(u); f = zeros(n);
   for (j=1; j<=n; j++)
       f[j] = 2*sum(u*sin((1:n)*j*pi/(n+1)));

   Note that sinFFT is the most efficient when n+1 is a product of
   small primes, where n is the transform length.
           
\end{verbatim}

See also: \ref{invsinFFT} {(invsinFFT)}, \ref{cosFFT} {(cosFFT)}, \ref{sinqFFT} {(sinqFFT)}, \ref{cosqFFT} {(cosqFFT)}, \ref{realFFT} {(realFFT)}, \ref{FFT} {(FFT)}.
\begin{verbatim}
   Error codes:
   -1: First argument not a real array
   -2: Second argument not integer
   -3: Second argument out of range
   
\end{verbatim}
\end{tscreen}





\subsection{sinh\label{sinh}}

\begin{tscreen}
\begin{verbatim}
[y] = sinh(x)
 y = sinh(x) computes the hyperbolic sine function of x.
   If x is complex, the result is complex, otherwise real.
   If x is an array, the operation is applied componentwise.
\end{verbatim}
\end{tscreen}





\subsection{sinqFFT\label{sinqFFT}}

\begin{tscreen}
\begin{verbatim}
[f] = sinqFFT(u; dim)
 sinqFFT computes the quarter-wave sine Fourier transform of array u.
   Except for the quarter-wave sine character, it works similarly to sinFFT.
   
   For vector u, f=sinqFFT(u) is equivalent with

   n = length(u); f = zeros(n);
   for (j=1; j<=n; j++)
       f[j] = (-1)^(j-1)*u[n] + 2*sum(u[1:n-1]*sin((2*j-1)*(1:n-1)*pi/(2*n)));

   sinqFFT is most efficient when the transform length is a product
   of small primes.
           
\end{verbatim}

See also: \ref{invsinqFFT} {(invsinqFFT)}, \ref{realFFT} {(realFFT)}, \ref{cosqFFT} {(cosqFFT)}, \ref{FFT} {(FFT)}.
\begin{verbatim}
   Error codes:
   -1: First argument not a real array
   -2: Second argument not integer
   -3: Second argument out of range
   
\end{verbatim}
\end{tscreen}





\subsection{size\label{size}}

\begin{tscreen}
\begin{verbatim}
[...] = size(x)
 [n,m,...] = size(A) finds out the dimensions of array A.
   The number of n,m... must not exceed rank(A). If rank(A)==0
   (that is, A is scalar), n=size(A) sets 1 to n.
   V = size(A) assigns the dimension vector [n,m,..] to V.
   If A is scalar, V is set to 1, if A is vector, V becomes
   a one-element vector.
\end{verbatim}

See also: \ref{length} {(length)}, \ref{rank} {(rank)}.
\begin{verbatim}
   Error codes:
   -1: No output arguments
   -2: Argument has undefined value
   1: More than one output arg but non-array input arg
   2: Too many output args relative to input arg rank 
\end{verbatim}
\end{tscreen}





\subsection{smenu\label{smenu}}

\begin{tscreen}
\begin{verbatim}
[result] = smenu(title...)
 choice = smenu("title","choice1","choice2",...) displays
   a menu of choices and returns the "choice" string corresponding
   to the number entered by the user.
\end{verbatim}

See also: \ref{menu} {(menu)}.
\begin{verbatim}
   Error codes:
   -1: Less than two input args 
\end{verbatim}
\end{tscreen}





\subsection{sort\label{sort}}

\begin{tscreen}
\begin{verbatim}
[y,I] = sort(abs(z));

Then y=z[I] is the wanted result. This method, when applicable,
is most probably faster than supplying a user-defined comparison
function.

sort() is a rewritten function which should be backward compatible
with the old one. In this Tela version the old sort function is
still available under sort_old().
Error codes:
   1: Internal error
   2: Expecting Tela or CTela function in CMP (arg 2)
   3: CMP signature is not b = CMP(x1, x2)
\end{verbatim}
\end{tscreen}





\subsection{sort\_old\label{sort_old}}

\begin{tscreen}
\begin{verbatim}
[y;I] = sort_old(x)
 sort_old is the old sort routine. Use it only if you
   encounter problems with sort(), sort_old will be
   removed in a later version. The new sort() function
   accepts a second input argument (user-defined comparison)
   but is otherwise identical in semantics.
   
   sort_old(x) returns array x sorted in ascending order.
   If x is complex, it is sorted by the real parts.
   If x is not an array, it is returned as is.
   [y,I] = sort_old(x) returns also an index vector I such that
   y == x[I].

   To sort with user-defined comparisons, do the following.
   For example, if you want to sort a complex vector z by
   absolute value as Matlab does, first sort a vector of
   absolute values saving the index information:

   [y,I] = sort_old(abs(z));

   Then y=z[I] is the wanted result.

   If x is multidimensional, it is implicitly flattened.
   Use the map function to get around this problem.
\end{verbatim}
\end{tscreen}





\subsection{source\label{source}}

\begin{tscreen}
\begin{verbatim}
[] = source(fn)
 source("file.t") loads the tela code from given file.
\end{verbatim}

See also: \ref{source_silent} {(source\_silent)}, \ref{autosource} {(autosource)}, \ref{load} {(load)}.
\begin{verbatim}
   Error codes:
   1: Operation did not succeed
   2: Argument not a string 
\end{verbatim}
\end{tscreen}





\subsection{source\_silent\label{source_silent}}

\begin{tscreen}
\begin{verbatim}
[] = source_silent(fn)
 source_silent("file.t") is similar to source("file.t"),
   but it does not complain if e.g. the file does not exist.
\end{verbatim}

See also: \ref{source} {(source)}, \ref{autosource} {(autosource)}, \ref{load} {(load)}.
\begin{verbatim}
   Error codes:
   1: Argument not a string 
\end{verbatim}
\end{tscreen}





\subsection{sppcmark\label{sppcmark}}

\begin{tscreen}
\begin{verbatim}
[] = sppcmark(x,y...)
 sppcmark(x,y) sets a marker in the current SPPC panel.
   The following options can be given, in addition to those
   listed under 'help sppcopts':

   Option      Type       Explanation
   ------      ----       -----------
   label       string     Text to draw at (x,y)
   markertype  integer    0=none,1=dot,2=+,3=X,4=box,5=filled box,
                            6=diamond, 7=filled diamond, 8=triangle,
                            9=filled triangle, 10=inverted triangle,
                            11=filled inverted triangle, 12=circle,
                            13=filled circle
   markersize  real       in points (1/72 inch), default 6
   markercolor 3-vector   Color of marker and text
   endpoint_x  real       Draws a line from (x,y)
   endpoint_y  real          to (endpoint_x,endpoint_y).
                             Current linewidth, linetype, linecolor
                             are used.
   arrow       true/false Whether to draw arrowhead at line end                                                  
   
\end{verbatim}

See also: \ref{curve} {(curve)}, \ref{pseudo} {(pseudo)}, \ref{panel} {(panel)}, \ref{endpanels} {(endpanels)}, \ref{sppcopts} {(sppcopts)}, \ref{xlabels} {(xlabels)}.
\begin{verbatim}
   Error codes:
   1: Bad first argument x (not scalar)
   2: Bad second argument y (not scalar)
   5: Odd number of arguments
   6: Option name not a string
   7: Unknown option
   8: Bad option value
\end{verbatim}
\end{tscreen}





\subsection{sppcopts\label{sppcopts}}

\begin{tscreen}
\begin{verbatim}
[] = sppcopts(...)
 sppcopts([,"option",value,...]) provides a method for giving
   SPPC options without using any SPPC primitive. The following
   options can be given. These can also be, and usually are,
   attached to the plotting functions themselves (curve, pseudo, etc.).
   Options marked with (*) are global, others are reset after
   each panel:

   Option name       Type         Explanation
   ------------------------------------------
  *title             string       Overall title of page
   leftlabel         string       Left label of this panel
   rightlabel        string       Right label of this panel
   xrightlabel       string       Right label on bottom horiz axis
   xleftlabel        string       Left label on bottom horiz axis
   linetype          integer      0=none,1=solid,2=dashed,3=dotted,
                                    4=dotdashed
   linecolor         3-vector     Curve color, RGB-values in 0..1 range
   fillcolor         3-vector     Fill color (for filled curves only)
  *foreground        3-vector     Color of axes, labels etc.
  *background        3-vector     Background color of page
  *framewidth        real or int  Frame and axis line width, in 1/72-inch
   linewidth         real or int  Curve line width, in 1/72-inch
  *xmin,*xmax,ymin,               Manually given axis limits,
     ymax,zmin,zmax  real           z refers to pseudocolor scaling
   ymin_right,ymax_right real     Right vertical axis min,max
   rightcurve        true/false   This curve uses the right vertical axis.
                                    Reset after each curve.
   houraxis          true/false   This xlabels() is hours, render as hh:mm
   xlabelskip        integer      Tick marks between each xlabel (default 1)
  *houraxis_x        true/false   Horizontal axis is hour axis
  *grid              true/false   Whether to draw dotted grid lines
  *interp            true/false   If true, draws pseudocolor surfs smoothly
  *palette           3x256 matrix Color palette, Int matrix, values in 0..255
                     "default"    Rainbow palette
  *xlog              true/false   Logarithmic horizontal axis
   ylog              true/false   Logarithmic vertical axis
   zlog              true/false   Logarithmic color scaling for pseudos
  *showneg           true/false   If log scale: show <= 0 values as smallest
  *titlefontsize     real         Size of title font in 1/72-inch units
  *labelfontsize     real         Size of side label font
  *axisfontsize      real         Size of axis tickmark label font
  *annotfontsize     real         Size of marker label font
  *usebinary         true/false   Default true; if false, uses ASCII in tmpfile
   panelheight       real         Relative height of panel, default 1.0

   Using SPPC in batch mode: see endpanels.
  
\end{verbatim}

See also: \ref{curve} {(curve)}, \ref{sppcmark} {(sppcmark)}, \ref{pseudo} {(pseudo)}, \ref{panel} {(panel)}, \ref{xlabels} {(xlabels)}, \ref{endpanels} {(endpanels)}.
\begin{verbatim}
   Error codes:
   5: Odd number of arguments
   6: Option name not a string
   7: Unknown option
   8: Bad option value
\end{verbatim}
\end{tscreen}





\subsection{sprintf\label{sprintf}}

\begin{tscreen}
\begin{verbatim}
[s] = sprintf(formatstr,arg)
 sprintf("format-string",arg1,arg2,...) is an interface to the C
   sprintf function. The format string should have a percent
   slot for every arg. The args may be integer or real scalars
   or strings.
\end{verbatim}

See also: \ref{sformat} {(sformat)}.
\begin{verbatim}

   LIMITATIONS:
       This implementation allows only one arg (arg1).
       The resulting string may not become larger than
           500 chars or Tela may crash.
   Error codes:
   -1: First arg not a string
   -2: Args may only be scalar ints or reals, or strings
   
\end{verbatim}
\end{tscreen}





\subsection{sqrt\label{sqrt}}

\begin{tscreen}
\begin{verbatim}
[y] = sqrt(x)
 y = sqrt(x) computes the square root of x.
   If x is complex, the result is complex. If x is real or
   integer, but negative, the result is complex (purely
   imaginary). If x is real or integer and non-negative,
   the result is real.
   If x is an array, the operation is applied componentwise.
   If some of the components are negative, all components
   of the result are complex.
\end{verbatim}
\end{tscreen}





\subsection{srand\label{srand}}

\begin{tscreen}
\begin{verbatim}
[] = srand(seed)
 srand(seed) seeds the random number generator.
   The same seed will always produce the same random
   number sequence. The argument must be an integer.
\end{verbatim}

See also: \ref{rand} {(rand)}.
\begin{verbatim}
   Error codes:
   1: Argument not an integer 
\end{verbatim}
\end{tscreen}





\subsection{stencil2d\_4\label{stencil2d_4}}

\begin{tscreen}
\begin{verbatim}
[Lu] = stencil2d_4(u,ap0,am0,a0p,a0m)
 stencil2d_4(u,ap0,am0,a0p,a0m) computes the two-dimensional
   five-point "molecule" where the coefficient of the central
   term is unity:

   Lu = u[i,j]
      + ap0*u[i+1,j] + am0*u[i-1,j]
          + a0p*u[i,j+1] + a0m*u[i,j-1];

   where the indices i and j run from 2..nx and 2..ny where
   [nx,ny] = size(u). The size of ap0,am0,a0p,a0m must be two
   less than the size of u in both directions.
   Error codes:
   -1: One of the arguments is not a real matrix
   -2: One of the coefficient args has bad size
\end{verbatim}
\end{tscreen}





\subsection{str2num\label{str2num}}

\begin{tscreen}
\begin{verbatim}
[y] = str2num(s)
 str2num(s) converts a string to a number.
   The string must represent a scalar. If an error
   occurs, str2num returns a void value.
   Error codes:
   1: Argument not a string 
\end{verbatim}
\end{tscreen}





\subsection{streq\label{streq}}

\begin{tscreen}
\begin{verbatim}
[y] = streq(s1,s2)
 streq("string1","string2") returns 1 if the argument
   strings are exactly equal and 0 otherwise. If one of
   the args is not a string, the result is also 0.
\end{verbatim}

See also: \ref{strstarteq} {(strstarteq)}.
\end{tscreen}





\subsection{strmat\label{strmat}}

\begin{tscreen}
\begin{verbatim}
[x] = strmat(...)
 strmat("string1","string2",...) makes a string matrix
   ouf of individual strings. The strings need not be same length,
   they are padded with zeros (invisible) if they are not.

   Usage example:
   S = strmat("string1","longer_string");
   s = tostring(S[1,:]);
   after which streq(s,"string1") will return true (1).
   The function tostring is needed to get rid of possible
   extra zeros at the end of the string.
   
\end{verbatim}

See also: \ref{strmat2} {(strmat2)}, \ref{tostring} {(tostring)}, \ref{streq} {(streq)}.
\begin{verbatim}
   Error codes:
   -1: Argument not a string
\end{verbatim}
\end{tscreen}





\subsection{strmat2\label{strmat2}}

\begin{tscreen}
\begin{verbatim}
[x] = strmat2(str; sep)
 strmat2("string") creates a string matrix from "string"
   interpreting the newline character as row ending marker.
   strmat2("string",sep) uses separator sep instead of
   newline char; sep may be either character or string.
   If sep is a string, any character that is a member of sep
   is taken to be a separator. If the rows have unequal lengths,
   they are padded with zeros.
\end{verbatim}

See also: \ref{strmat} {(strmat)}.
\begin{verbatim}
   Error codes:
   -1: First arg not a string
   -2: Second arg not a char or string
   
\end{verbatim}
\end{tscreen}





\subsection{strstarteq\label{strstarteq}}

\begin{tscreen}
\begin{verbatim}
[y] = strstarteq(s1,s2)
 strstarteq("string1","string2") returns 1 if the argument
   strings are equal on the first min(length(s1),length(s2))
   characters and 0 otherwise.
   If one of the the args is not a string, the result is also 0.
\end{verbatim}

See also: \ref{streq} {(streq)}.
\end{tscreen}





\subsection{sum\label{sum}}

\begin{tscreen}
\begin{verbatim}
[y] = sum(x;d)
 sum(x) sums all the elements of x, if x is an array.
   The result type is always the same as the component type
   of x. If x is scalar, it is returned as such.
   sum(x,d) sums only along d'th dimension, returning array
   of rank one less than rank(x).
\end{verbatim}

See also: \ref{cumsum} {(cumsum)}, \ref{prod} {(prod)}, \ref{map} {(map)}.
\begin{verbatim}
   Error codes:
   -1: Nonnumeric input arg
   -2: Second argument not an integer
   -3: Second argument (dimension spec) out of range 
\end{verbatim}
\end{tscreen}





\subsection{system\label{system}}

\begin{tscreen}
\begin{verbatim}
[] = system(s)
 system("string") executes string as an external operating system
   command.
\end{verbatim}

See also: \ref{system_with_returnvalue} {(system\_with\_returnvalue)}, \ref{run} {(run)}.
\begin{verbatim}

   Error codes:
   1: Argument not a string 
\end{verbatim}
\end{tscreen}





\subsection{system\_with\_returnvalue\label{system_with_returnvalue}}

\begin{tscreen}
\begin{verbatim}
[;rv] = system_with_returnvalue(s)
 system_with_returnvalue("string") executes string as an external operating system
   command and returns the return value rv of the command.
\end{verbatim}

See also: \ref{system} {(system)}, \ref{run} {(run)}.
\begin{verbatim}

   Error codes:
   1: Argument not a string 
\end{verbatim}
\end{tscreen}





\subsection{t2ct\label{t2ct}}

\begin{tscreen}
\begin{verbatim}
[] = t2ct(fn)
 t2ct("filename.t") translates t-code to ct-code.
   This makes it possible to compile t-files into
   faster code.

   To compile a t-file "fyle.t", do the following:

   (1) Generate "fyle.ct" by t2ct("fyle.t") from Tela.
   You can also do this from the shell prompt or Makefile by

   tela -sbe 't2ct("fyle.t")'

   (use single quotes around the argument but cannot show it here, sorry).
   
   (2) Compile it using

   telakka -c fyle.ct

   from the shell or the Makefile. (From within Tela,
   you can use the system function, or precede the command
   line by "!" in interactive mode.)

   (3) The resulting object file fyle.o can be linked directly
   to Tela on some systems (e.g. Linux using the DLD library):

   link("fyle.o")

   On systems using the DSO library (SGI, Sun ...) you must first
   make a shared object file:

   ld -o fyle.so -shared fyle.o

   The DSO file can now be linked to Tela:

   link("fyle.so")

   (When using the DLD library you can emulate the previous step
   by ld -o fyle.so -r fyle.o, this makes it possible to write machine-
   independent Makefiles using gmake.)

   (3b) If you do not want dynamic linking, or if your system does
   not support it (e.g. UNICOS), you can always link your stuff in
   a new Tela kernel by simply using the telakka command:

   telakka -o newtela fyle.o

   When you start the binary 'newtela' the functions defined in
   fyle.t are available.
   
\end{verbatim}

See also: \ref{avoid_constructors} {(avoid\_constructors)}, \ref{system} {(system)}.
\begin{verbatim}
   Error codes:
   1: Operation did not succeed
   2: Argument not a string
   3: Could not open output file
   
\end{verbatim}
\end{tscreen}





\subsection{tan\label{tan}}

\begin{tscreen}
\begin{verbatim}
[y] = tan(x)
 y = tan(x) computes the tangent function of x.
   If x is complex, the result is complex, otherwise real.
   The argument must be in radians.
   If x is an array, the operation is applied componentwise.
\end{verbatim}
\end{tscreen}





\subsection{tanh\label{tanh}}

\begin{tscreen}
\begin{verbatim}
[y] = tanh(x)
 y = tanh(x) computes the hyperbolic tangent function of x.
   If x is complex, the result is complex, otherwise real.
   If x is an array, the operation is applied componentwise.
\end{verbatim}
\end{tscreen}





\subsection{telapath\label{telapath}}

\begin{tscreen}
\begin{verbatim}
[currentpath; maxelements] = telapath(;newpath)
 telapath() returns the current Tela path as a string.
   telapath(NEWPATH) sets the Tela search path to NEWPATH and returns
   the *old* search path.
   [oldpath,maxelems] = telapath()
   also returns the maximum number of members in the
   search path in maxelems (currently 100).
   Error codes:
   -1: expecting string NEWPATH (arg 1)
    1: too many elements in NEWPATH (arg 1); ignoring trailing elements
\end{verbatim}
\end{tscreen}





\subsection{tic\label{tic}}

\begin{tscreen}
\begin{verbatim}
[] = tic()
 tic() marks the CPU time at which it was invoked.
   To measure CPU time, use tic() and toc().
\end{verbatim}

See also: \ref{cputime} {(cputime)}, \ref{toc} {(toc)}.
\begin{verbatim}
   Example:

      a = rand(100,100); tic(); b=inv(a); toc()

  This would measure the CPU time in inverting a 100x100
  random real matrix. See also: toc. 
\end{verbatim}
\end{tscreen}





\subsection{toc\label{toc}}

\begin{tscreen}
\begin{verbatim}
[t] = toc()
 toc() gives the CPU seconds used since the last call to tic().
\end{verbatim}

See also: \ref{tic} {(tic)}, \ref{cputime} {(cputime)}.
\end{tscreen}





\subsection{tostring\label{tostring}}

\begin{tscreen}
\begin{verbatim}
[y] = tostring(x)
 tostring(x) converts an integer vector to a string.
   Transfer of characters is stopped if zero element
   is encountered.
   tostring(A) where A is geneeral integer array copies
   A and sets the string flag.
   Error codes:
   -1: Argument not an integer array
   
\end{verbatim}
\end{tscreen}





\subsection{transpose\label{transpose}}

\begin{tscreen}
\begin{verbatim}
[B] = transpose(A; P)
 B = transpose(A) returns a transpose of array A: B[i,j,k...l] = A[l...k,j,i].
   B = transpose(A,P) where P is integer vector transposes the indices according
   to the permutation defined by P.
   For example if A has rank 3, B = transpose(A,[2,1,3]) causes the assignment
   B[j,i,k] = A[i,j,k] to be carried out. B = transpose(A) would in this case
   correspond to B[k,j,i] = A[i,j,k].

   The second argument is meaningful only if rank(A) is greater than 2.
   
   You can abbreviate "transpose(A)" by "A.'".
\end{verbatim}

See also: \ref{herm} {(herm)}, \ref{flip} {(flip)}.
\begin{verbatim}
   Error codes:
   -1: Permutation argument not integer array
   -2: Permutation argument of bad rank or size
   -3: Permutation argument contains invalid integers 
\end{verbatim}
\end{tscreen}





\subsection{trunc\label{trunc}}

\begin{tscreen}
\begin{verbatim}
[y] = trunc(x)
 trunc(x) returns the nearest integer not larger in absolute value.
   x must be integer or real scalar or array. If it is an array,
   the operation is applied componentwise.
\end{verbatim}

See also: \ref{floor} {(floor)}, \ref{ceil} {(ceil)}, \ref{round} {(round)}.
\begin{verbatim}
   Error codes:
   -1: Complex or nonnumeric input argument 
\end{verbatim}
\end{tscreen}





\subsection{ungetc\label{ungetc}}

\begin{tscreen}
\begin{verbatim}
[] = ungetc(ch,fnum)
 ungetc(ch,fnum) puts the character ch back to the file defined
   by identification number fnum. The success of the operation
   depends on the implementation underlying C library ungetc call;
   ANSI C guarantees only one character of pushback but many libraries
   allow more.
   Ungetc does not return a value. If the operation is unsuccessful,
   a warning message is generated.
   Error codes:
   -1: Bad first argument: not integer
   -2: Bad first argument: outside range
   -3: File was not open
   -4: First argument not an integer
   1: Cannot push back character
\end{verbatim}
\end{tscreen}





\subsection{unhide\label{unhide}}

\begin{tscreen}
\begin{verbatim}
[] = unhide(...)
 hide("sym-name",...) unsets the 'hidden' attribute to
   specified symbols.
\end{verbatim}

See also: \ref{whos} {(whos)}, \ref{hide} {(hide)}.
\begin{verbatim}
   Error codes:
   1: Argument not a string
   2: Argument does not name a symbol
   
\end{verbatim}
\end{tscreen}





\subsection{version\label{version}}

\begin{tscreen}
\begin{verbatim}
[x] = version()
 version() returns the Tela version number (real) currently
   in use. 
\end{verbatim}
\end{tscreen}





\subsection{vplot\label{vplot}}

\begin{tscreen}
\begin{verbatim}
[] = vplot(x,y,vx,vy...)
 vplot(x,y,vx,vy[,options]) produces a 2D vector plot of the vector
   field (vx,vy). All arguments x,y,vx and vy must be 2D integer or
   real arrays and of the same size. Each 2D vector will be positioned
   at (x[i,j],y[i,j]) and its direction will be (vx[i,j],vy[i,j]) where
   (i,j) run over rows and columns of the matrices.
\end{verbatim}

See also: \ref{annotate} {(annotate)}, \ref{plot} {(plot)}, \ref{mesh} {(mesh)}, \ref{contour} {(contour)}, \ref{pcolor} {(pcolor)}.
\begin{verbatim}
   Error codes:
   1: Could not open temporary MTV file
   2: One of first four args is not a numeric array
   3: One of first four args has rank not equal to 2
   4: Dimensions of first four args disagree
   5: Syntax error in graph options 
\end{verbatim}
\end{tscreen}





\subsection{where\label{where}}

\begin{tscreen}
\begin{verbatim}
[y] = where(c,a,b)
 where(c,a,b) returns a where c is nonzero and b where c is zero.
   Usually the arguments are arrays, and if they are, they must be
   of similar dimensions.
   Error codes:
   -1: First argument not integer-valued
   -2: Second arg non-numeric
   -3: Second arg is incompatible array
   -4: Third arg non-numeric
   -5: Third arg incompatible array
\end{verbatim}
\end{tscreen}





\subsection{whos\label{whos}}

\begin{tscreen}
\begin{verbatim}
[] = whos(;hidden)
 whos() displays names of variables together with their
   types and values, if short. 'Hidden' symbols are not shown.
   whos("hidden") shows also hidden symbols.
\end{verbatim}

See also: \ref{hide} {(hide)}, \ref{unhide} {(unhide)}.
\begin{verbatim}
   Error codes:
   1: Bad argument
   -1: Cannot open pipe to sort
   
\end{verbatim}
\end{tscreen}





\subsection{xlabels\label{xlabels}}

\begin{tscreen}
\begin{verbatim}
[] = xlabels(x,y...)
 xlabels(x,y) puts a xlabel vector to SPPC page.
   xlabels(x,y,"houraxis","true") assumes y represents hours
   and renders the labels as hh:mm.
   Do 'help sppcopts' to see other valid options (especially
   rightlabel and leftlabel).
\end{verbatim}

See also: \ref{curve} {(curve)}, \ref{sppcmark} {(sppcmark)}, \ref{pseudo} {(pseudo)}, \ref{endpanels} {(endpanels)}, \ref{sppcopts} {(sppcopts)}, \ref{panel} {(panel)}.
\begin{verbatim}
   Error codes:
   1: Could not open temporary MTV file
   2: First argument (x) not a real vector
   3: Second argument (y) not a real vector
   4: First two arguments of different lengths
   5: Odd number of arguments
   6: Option name not a string
   7: Unknown option
   8: Bad option value
\end{verbatim}
\end{tscreen}




\end{document}
